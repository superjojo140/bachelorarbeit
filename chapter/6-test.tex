\chapter{Gestaltungslösungen evaluieren}
\label{chapter:evaluation}

\subsubsection{Einleitung Usertest}
Im vorherigen Kapitel wurden einzelne Aspekte der konkreten Implementierung der
Gestaltungslösungen vorgestellt. Im nächsten Schritt sollen die so entwickelten
Prototypen mit einem Nutzer getestet werden. Dies entspricht dem vierten
Schritt \textit{Gestaltungslösungen aus der Benutzerperspektive evaluieren} im
menschzentrierten Gestaltungsprozess der ISO Norm \cite{ISO9241}. Christian
Moser erklärt den möglichen Ablauf eines solchen Usertests wie folgt:
\glqq{}Bei Tests mit Benutzern wird den Testteilnehmern der Kontext anhand des
Szenarios erklärt und dann werden die Aufgaben gestellt. Dabei wird beobachtet,
wie gut sie diese mit Hilfe des Prototyps lösen können.\grqq{}\cite{moserTesting}. In diesem Fall wird der Test gemeinsam mit \ipName von der allgemeinen Studienberatung der Universität Kassel durchgeführt. Gemeinsam haben wir uns eine Stunde Zeit genommen um das neu entwickelte Modul zur Terminvereinbarung zu testen. Wir sitzen gemeinsam im Büro von \ipName und können die Software so direkt an seinem Dienstrechner testen um den Nutzungskontext möglichst authentisch zu gestalten.

\section{Usertest}
\subsubsection{Monatsübersicht}
Der Usertest beginnt damit, dass \ipName die Software Stubegru im Browser
aufruft und den View mit dem neuen Kalendermodul zur Terminvergabe öffnet. Sein
erster Blick fällt auf die Monatsübersicht der Termine im November.

\begin{figure}[H]
    \caption{Monatsübersicht der Termine im November.}
    \centering
    \includegraphics[width=0.9\textwidth]{screen_new_month_view.png}
\end{figure}

In dieser Übersicht kann er einsehen, welche Termine für den aktuellen Monat
eingetragen sind. Anhand der farbigen Punkte \todo{Listing mit color attr
    anpassen} kann \ipName auf einen Blick erkennen, welche Termine noch frei sind
und welche bereits an ratsuchende Kunden vergeben wurden. Neben dem farbigen
Punkt kann für jeden Termin eingesehen werden, bei welchem Studienberatenden
dieser Termin stattfindet. \ipName merkt an, dass es im Vergleich zur alten
Version möglich ist direkt in der Übersicht zu sehen, zu welchem Zeitpunkt die
Termine stattfinden und wie viele Termine an einem Tag angeboten werden. Er
testet was passiert, wenn viele Termine am gleichen Tag angelegt werden.
Zufrieden stellt er fest, dass die Zelle für den entsprechenden Tag automatisch
größer wird, wenn viele Termine eingetragen werden. \ipName überlegt weiterhin,
dass die Monatsübersicht dadurch eventuell so lang werden könnte, dass nicht
mehr alle Zeilen gleichzeitig auf den Bildschirm passen könnten. Wir
diskutieren wie viele Termine üblicherweise pro Tag angeboten werden und
einigen uns darauf die aktuelle Ansicht zunächst beizubehalten und in der
Praxis zu testen.

\begin{figure}[H]
    \caption{Toggles im Dropdown Menü um die angezeigten Termine zu filtern. Hier werden nur freie Termine dargestellt}
    \centering
    \includegraphics[width=0.9\textwidth]{screen_new_filter.png}
\end{figure}

Ich weise \ipName darauf in, dass er über den Button mit dem Zahnrad weitere
Filtermöglichkeiten hat, um eine überfüllte Monatsübersicht auf die
wesentlichen Termine zu reduzieren. Den Button hätte er zunächst nicht bemerkt,
nachdem er ihn einmal gefunden hat versteht er die Bedienung über die beiden
Toggles aber schnell. Er merkt positiv an, dass Änderungen an den
Filtereinstellungen direkt übernommen werden und die Auswirkung der Filter
somit leicht zu begreifen ist. Ihm fällt auf, dass der Filter für eigene
Termine für die Hilfskräfte wenig Sinn ergibt, da diese keine Berechtigung
haben um selber Termine zu erstellen. Sie können lediglich eingestellte
Zeitslots an Ratsuchende vergeben. Wir diskutieren über die Möglichkeit diesen
Filter für Accounts von Hilfskräften auszublenden, entscheiden uns aber dafür,
dass dies nicht notwendig ist, da der Filter für diese Nutzergruppe zwar
überflüssig, aber auch nicht störend ist.

\subsubsection{Detailansicht}

Als nächstes gebe ich \ipName die Aufgabe einen neuen Zeitslot zu erstellen.
Ich verrate nicht weiter, wo er diese Funktion finden kann und lasse ihn
ausprobieren. Nach kurzer Zeit entdeckt \ipName den Button \textit{Neuer
    Termin} und es öffnet sich das Modal zum Erstellen eines freien Zeitslots.
\ipName erklärt seine Gedanken: \glqq{}Ich kenne es von anderer Software so,
dass man auf den entsprechenden Tag in der Monatsübersicht klickt um einen
neuen Termin anzulegen. Aber aus der alten Version von Stubegru erinnere ich
mich noch an den grünen Knopf zum Erstellen eines Termins. Früher war der Knopf
unter der Monatsübersicht, aber so fügt er sich elegant in die Leiste mit der
Suche und den neuen Filtern ein.\grqq{}

\begin{figure}[H]
    \caption{Navigationsleiste über der Monatsansicht. Über den grünen Button kann eine neuer Zeitslot erstellt werden.}
    \centering
    \includegraphics[width=0.9\textwidth]{screen_new_create_button.png}
\end{figure}

In dem leeren Formular füllt \ipName Schritt für Schritt die Input Felder aus.
Nach einem Klick auf das Feld für die Startzeit ist er enttäuscht. In der alten
Version öffnete sich hier ein Timepicker mit dessen Hilfe man die Uhrzeit ohne
Tastatur nur mit der Maus eingeben konnte. Auch ich bin überrascht, denn in
meinen Tests hat sich hier auch ein Timepicker geöffnet. Nach kurzer REcherche
stellt sich heraus, dass das verwendete Html5 Input Feld für Uhrzeiten in jedem
Browser unterschiedlich dargestellt wird. Während der Implementierung habe ich
die Ergebnisse immer in Google Chrome getestet, dort erscheint nach dem Klick
in ein Uhrzeit-Input ein Timepicker \todo{Timepicker ins Glossar}. Die
Abteilung der Studienberatung verwendet in der Regel Firefox als Browser, da
dieser vom ITS standardmäßig vorinstalliert ist. Ich verspreche mich dem
Problem anzunehmen und einen zusätzlichen Timepicker einzubinden, der
unabhängig vom verwendeten Browser funktioniert.

\begin{figure}[H]
    \caption{Erstellen eines neuen Termins und Ausfüllen der Eingabefelder. In diese Fall wird in Google Chrome ein Timepicker angezeigt.}
    \centering
    \includegraphics[width=0.9\textwidth]{screen_new_create_with_picker.png}
\end{figure}

Alle anderen Formularfelder füllt \ipName zügig aus. Da die abgefragten Daten
und die Anordnung der Eingabefelder genauso wie in der alten Version gestaltet
sind, merkt man hier, dass \ipName bereits viel Erfahrung mit der alten
Softwareversion von Stubegru gesammelt hat. Den unten rechts platzierten grünen
Button zum Speichern der Daten findet er sofort. Der Datensatz wird in der
Datenbank gespeichert und eine kleine grüne Meldung am linken unteren Rand
verkündet, dass der Termin erfolgreich gespeichert wurde. \ipName ist trotzdem
verunsichert, ob der Vorgang erfolgreich war. In der alten Version, hat sich
das Modal nach dem Speichern automatisch geschlossen. \ipName merkt an, dass in
der neuen Version nun jedes Mal ein zusätzlicher Klick auf den Button
\textit{Schließen} notwendig ist. Wenn das Modal sich nach dem Speichern direkt
schließen würde, wäre es für ihn einfacher. Ich erkläre, dass das Modal
geöffnet bleibt, damit ein neu erstellter Termin direkt an einen Kunden
vergeben werden kann. Nach dem Speichern erscheint hierfür der große blaue
Button mit der Aufschrift \textit{Termin vergeben}. \ipName erklärt, dass die
Termine eigentlich immer von den Hilfskräften der Erstinformation vergeben
werden und es für die Studienberatenden nicht nötig sei, den erstellten Termin
direkt vergeben zu können. Wir einigen uns darauf, dass ein Klick auf den
Speichern Button das Modal, nach erfolgreichem Sichern der Daten, schließt.

Und noch eine weitere Sache fällt \ipName auf: Wenn er bereits einen Termin
eingetragen hat und dann einen zweiten Termin erstellen möchte, werden alle
Eingabefelder zurückgesetzt und sind wieder leer. \glqq{}Das war in der alten
Version praktischer\grqq{}, sagt \ipName \glqq{}Oftmals trage ich viele Termine
fpr den gleichen Tag mit den gleichen Attributen ein. Das einzige was ich
bisher anpassen musste war dann die Uhrzeit. Nun brauche ich viel mehr Klicks
weil ich jedes Mal wieder den Raum und das Mailtemplate auswählen muss.\grqq{}.
Es stellt sich heraus, das es in der alten Softwareversion einen Bug gab,
wodurch in eingen Fällen das zurücksetzen des Formulars nicht aufgerufen wurde.
Dieser Bug hat sich in der Praxis dann allerdings als sehr praktisch etabliert.
Die Formularfelder nie zurückzusetzen kann allerdings inkonsistente Status
erzeugen, wenn zuvor beispielsweise ein bereits vergebener Termin im Modal
angezeigt wurde. Wir einigen uns auf eine gemeinsame Idee: Wenn das Modal zum
Erstellen eines Termins angezeigt wird, erscheint in der Fußleiste ein weiterer
Button mit der Aufschrift: \textit{Speichern und nächster}. Dieser Button
speichert den aktuellen Termin und lässt das Modal ohne zurücksetzen der
Formularfelder geöffnet. Somit kann direkt ein neuer Termin mit ähnlichen
Attributen eingetragen werden. \todo{Direkte Zitate von \ipName kennzeichnen?}

\begin{figure}[H]
    \caption{Ein freier Termin. Studienberatende können Änderungen am Termin vornehmen oder einen Kunden einbuchen.}
    \centering
    \includegraphics[width=0.9\textwidth]{screen_new_free_meeting.png}
\end{figure}

Als nächstes soll \ipName einen erstellten Zeitslot an einen Kunden vergeben.
Intuitiv klickt er hierfür auf den entsprechenden Termin in der
Monatsübersicht. Es öffnet sich das Modal mit der Detailansicht der
Termindaten. Sehr schnell findet \ipName den großen blauen Knopf zur
Terminvergabe. Wie er es bereits aus der alten Softwareversion kennt, öffnet
sich das Formular um die weiteren Kundendaten einzutragen. \ipName prüft was
passiert, wenn er hier relevante Felder einfach leer lässt. Beruhigt stellt er
fest, das beim Versuch die Kundendaten zu speichern ein Hinweis am
entsprechenden leeren Eingabefeld auftaucht und das Speichern unvollständiger
Daten verhindert.

Nachdem alle Felder vollständig ausgefüllt sind, speichert \ipName die
Kundendaten und sieht eine leicht veränderte Detailansicht des Termins:

\begin{figure}[H]
    \caption{Ein vergebener Termin. Terminattribute und neue Kundendaten können erst nach Löschen der aktuellen Kundendaten hinterlegt werden.}
    \centering
    \includegraphics[width=0.9\textwidth]{screen_new_assigned.png}
\end{figure}

Die Terminattribute und Kundendaten können nicht weiter bearbeitet werden. Dies
erkennt \ipName schnell daran, dass die entsprechenden Eingabefelder ausgegraut
angezeigt werden. Zusätzlich verrät der Hinweis in der blauen Infobox, dass
Änderungen nur vorgenommen werden können, wenn zuvor die Kundendaten gelöscht
werden. \ipName realisiert, dass diese Einschränkung Sinn macht, da somit keine
Änderungen am Termin vorgenommen werden können, ohne dass der Kunde durch ein
erneutes Einbuchen darüber informiert wird.

\ipName freut sich, dass die Telefonnummer nun in übersichtlichen Blöcken von jeweils vier Ziffern angezeigt wird. Das Eintippen ins Telefon sei somit deutlich einfacher. Allerdings merkt er an, dass die Telefonnummer nur in dem Account des zugeordneten Beratenden angezeigt werden soll. So kann ein höherer Datenschutz gewährleistet werden und Hilfskräfte haben keinen Zugriff auf die Kontaktdaten aller ratsuchenden Kunden. Ich erkläre, dass ich diese Funktion bereits für das Beratungsanliegen umgesetzt habe. Wir loggen uns mit einem anderen Account ein und sehen, dass das Beratungsanliegen zensiert dargestellt wird. Dies Funktionalität soll in Zukunft auch für die Telefonnummer und Mailadresse der Kunden eingesetzt werden.

Abschließend betrachtet \ipName nochmals die Monatsübersicht. \glqq{} Wäre es
technisch aufwendig, bei den vergebenen, roten Terminen statt dem Namen des
Beratenden den Namen des Kunden anzuzeigen? So könnten die Hilfskräfte den
entsprechenden Termin eines Kunden schneller finden.\grqq{}, fragt er mich. Ich
antworte, dass dies technisch kein großer Aufwand wäre. Allerdings könnte es
Verwirrung stiften, wenn bei manchen Terminen der Name des Beratenden und bei
anderen der Name des Kunden angezeigt wird. Für den Fall, dass der Termin eines
bestimmten Kunden schnell gefunden werden muss, wurde außerdem die Suchfunktion
entwickelt. \ipName erinnert sich an die Suchfunktion und will diese direkt
ausprobieren.

\begin{figure}[H]
    \caption{Suchfunktion um bereits vergebene Termine anhand des eingebuchten Kunden zu finden. Die Suchanfrage \glqq{}jo\grqq{} ergibt drei Treffer.}
    \centering
    \includegraphics[width=0.9\textwidth]{screen_new_search.png}
\end{figure}

Das Eingabefeld für die Suchanfrage findet er schnell in der oberen
Navigationsleiste. Er gibt einige Buchstaben ein und direkt werden passende
Suchvorschläge angezeigt. \ipName ist überzeugt, dass dies eine gute Lösung
ist, um direkt nach Namen von ratsuchenden Kunden zu suchen. Er betont, dass es
wichtig ist, dass passenden Suchvorschläge bereits beim Tippen der ersten
Buchstaben angezeigt werden und kein Bestätigen der Suche mit Enter notwendig
ist.

\subsubsection{Abschluss Usertest}
Der Usertest ist nun beendet und \ipName bedankt sich für die
entgegengebrachte Aufmerksamkeit. Er ist sichtlich begeistert, dass die
Nutzenden des Systems in diesem Designprozess im Fokus stehen und freut sich
auf die Einführung der neuen Softwareversion: \glqq{} Wir müssen dann
sicherlich eine Schulung für die Hilfskräfte machen um sie mit den neuen
Features vertraut zu machen. Aber eigentlich ist ja fast alles so, wie sie es
schon gewohnt sind.\grqq{}, beendet \ipName unser Gespräch. Ich bedanke mich
ebenfalls für die Zeit und die konstruktive Kritik, die eine Diskussion über
Verbesserungspotenziale erst möglich gemacht hat.