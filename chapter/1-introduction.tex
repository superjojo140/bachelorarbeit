\chapter{Einleitung}

\section{Zusammenfassung}

Diese Arbeit beschäftigt sich mit dem Entwicklungsprozess einer Software zur
Terminvergabe von Beratungsterminen. Diese Software soll in der Studienberatung
an der Universität Kassel eingesetzt werden. Der Entwicklungsprozess wird mit
Methoden des \textit{Human Centered Design} begleitet und umgesetzt. Diese
Arbeit erklärt die verwendet Methoden, sowie die dahinterliegende Theorie. Im
Rahmen dieser Arbeit wird die Software zur Terminvereinbarung vollständig
implementiert. Dabei werden alle Schritte des Gestaltungsprozesses beschrieben,
dokumentiert und ausgewertet.

Die allgemeine Studienberatung der Universität Kassel setzt die Software
Stubegru bereits seit 6 Jahren zum Management ihrer Beratungstermine,
Abwesenheiten und als Wissensdatenbank ein. Das Modul zur Vergabe der
Beratungstermine soll nun grundlegend neu implementiert werden und noch besser
an die Bedürfnisse der Mitarbeitenden angepasst werden. In Zusammenarbeit mit
einem Ansprechpartner aus der Abteilung Studium und Lehre, wird dieses neue
Modul entwickelt, designt und implementiert.\\ Die Bachelorarbeit begleitet
diesen praktischen Prozess, indem theoretische Grundlagen erläutert werden und
alle Entwicklungsschritte beschrieben werden. Die Theorie des Human Centered
Design gibt vor, wie ein Gestaltungsprozess von Software aussehen kann. Neben
grundlegenden Herangehensweisen bringt diese Theorie einige Methoden mit, die
im Laufe des Gestaltungsprozesses angewandt werden. Hierzu gehört
beispielsweise das \textit{Interview im Kontext}, das in dieser Arbeit
ebenfalls erklärt wird und dessen praktische Umsetzung dokumentiert,
ausgewertet und kritisch hinterfragt wird.

\section{Motivation}

\subsubsection{Historische Entwicklung}
Wir leben in einem Zeitalter, in dem immer mehr Prozesse des täglichen Lebens
durch technische Systeme wie Software begleitet werden. In dieser Situation ist
es von hoher Relevanz, dass alle Menschen unserer Gesellschaft die Möglichkeit
haben, diese Systeme gleichermaßen zu nutzen und zu verstehen. In den Anfängen
der Softwareentwicklung wurde Software in der Regel aus Sicht der
verarbeitenden Maschine gedacht: Wie kann die Maschine die gegebenen Aufgaben
möglichst schnell, fehlerfrei und effizient abarbeiten? Die Nutzenden dieser
Maschinen war meist ein eingeschränkter, speziell dafür ausgebildeter
Personenkreis. Diese Situation hat sich gänzlich verändert. Fast jeder Mensch
agiert heute viele Male am Tag mit technischen, softwaregetriebenen Systemen:
Bankautomaten, Smartphones, Supermarktkassen oder Computer. Außerdem wird der
Umgang mit solchen Systemen durch neue Entwicklungen wie Sprachsteuerung oder
Touchscreens immer interaktiver und unmittelbarer. Somit ist es enorm wichtig,
dass alle diese Systeme eine klare und nutzungsfreundliche Schnittstelle für
die Nutzenden bieten\cite{moserTesting}.

\subsubsection{Rolle des Human Centered Design}
Der Ansatz des Human Centered Design setzt genau an dieser Stelle an, indem er
Grundkonzepte und Methoden vermittelt, die das Design von
Benutzerschnittstellen thematisieren\cite{hcd}. Das Kernkonzept des Human
Centered Design ist es, den Blickwinkel aus der Perspektive der Menschen zu
wählen, die mit den technischen Systemen täglich interagieren müssen. Die zu
entwickelnden Systeme sollen Schnittstellen anbieten, die intuitiv und einfach
zu verstehen sind. Prozessabläufe sollen aus Sicht des Nutzenden abgebildet
werden und möglichst wenig durch technische Hintergründe eingeschränkt
sein\cite{HMI-HCD}.

\subsubsection{Praktische Anwendung}
Der Gestaltungsprozess nach den Methoden des Human Centered Design soll in
dieser Bachelorarbeit genauer untersucht werden. Zum einen, indem die
theoretischen Hintergründe genauer erläutert und Methoden präzise erklärt
werden. Den wesentlichen Teil dieser Arbeit stellt aber die Dokumentation der
praktischen Anwendung all dieser Methoden und Konzepte dar. Dies soll am
Beispiel einer Software zur Terminvergabe in der Studienberatung an der
Universität Kassel durchgeführt werden. Das entsprechende Modul für diese
Software soll im Rahmen dieser Arbeit geplant, gestaltet, implementiert und
getestet werden. All diese Schritte werden schriftlich aufgearbeitet,
festgehalten und hinterfragt. Das Ziel dieser Arbeit soll es also sein, den
Designzyklus des Human Centered Design am praktischen Beispiel einer
Terminvergabe-Software anzuwenden. Die bestehenden wissenschaftlichen
Erkenntnisse im Bereich der menschzentrierten Gestaltung sollen somit
erarbeitet, vertieft und auf Praxistauglichkeit geprüft werden. Ein weiterer
wichtiger Bestandteil ist es schließlich die theoretischen und methodischen
Grundlagen im Hinblick auf die gewonnenen praktischen Erfahrungen zu
hinterfragen und zu diskutieren.

\section{Aufbau}
Zu Beginn der Arbeit werden in Kapitel \ref{chapter:methods} zunächst die
verwendeten Grundlagen und Methoden des Human Centered Design genauer erläutert
und klar definiert. Exemplarisch für eine Methode wird hier das Konzept des
Interviews im Kontext genauer betrachtet und eingeführt. Die
Kapitel \ref{chapter:user-context} bis \ref{chapter:evaluation} beschäftigen sich
mit der praktischen Entwicklung des Softwaremoduls zur Terminvereinbarung
anhand des vorgestellten Designzyklus. Um dem Lesenden einen Eindruck über den
praktischen Einsatz der betrachteten Software an der Universität zu geben, wird
in Kapitel \ref{chapter:user-context} zunächst die Situation in der allgemeinen
Studienberatung erläutert. Anschließend wird die eingesetzte Software genauer
vorgestellt und deren Entwicklungsgeschichte aufgegriffen. Im weiteren Verlauf
soll thematisiert werden, aus welchen Gründen eine Überarbeitung dieser
Software notwendig ist und welches Ziel die allgemeine Studienberatung damit
beabsichtigt. Im Kapitel  \ref{chapter:user-requirements} wird ein Interview im
Kontext mit einem Mitarbeitenden der Studienberatung durchgeführt, dokumentiert
und ausgewertet. Die aus den Wünschen und Bedürfnissen der Nutzenden
entstehenden Anforderungen sollen ebenfalls zusammengefasst und dargestellt
werden. Als nächster Schritt soll in Kapitel \ref{chapter:implementation} die
praktische Umsetzung der technischen Anforderungen an die Software skizziert
werden. Implementierungsdetails, Screenshots und Codeschnipsel sollen in diesem
Teil die Dokumentation der Umsetzung veranschaulichen. Als letzter Aspekt der
praktischen Durchführung soll in Kapitel \ref{chapter:evaluation} das Feedback
der Nutzenden bei ersten Usertests festgehalten und aufbereitet werden. In
Kapitel \ref{chapter:conclusion} wird schließlich das Ergebnis des
durchgeführten Entwicklungsprozesses dargestellt. Die eingesetzten Methoden
werden nochmals aufgegriffen und kritisch hinterfragt: An welchen Stellen
können die theoretischen Grundlagen des Human Centered Design den
Entwicklungsprozess in der Praxis tatsächlich sinnvoll unterstützen? Gab es
eventuell auch Methoden, die in der praktischen Umsetzung problematisch waren
oder noch optimiert werden könnten? Der letzte Teil dieser Arbeit gewährt
schlussendlich einen Ausblick auf weitere Forschungsfelder des Human Centered
Design und die weitere Entwicklung der Software für die Studienberatung der
Universität Kassel.