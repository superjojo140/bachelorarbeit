\chapter{Nutzungsanforderungen}

\section{Interview im Kontext}
\label{subsection:IIK}

Das Modul zur Terminvereinbarung der Software Stubegru soll auf den
Arbeitsalltag der allgemeinen Studienberatung der Universität Kassel angepasst
werden. Dies soll mit Methoden des Human Centered Design umgesetzt werden. Ein
zentraler Bestandteil des Human Centered Design ist der enge und stetige
Austausch mit den Nutzenden des Softwaresystems \cite{hci}. Um den Änderungsbedarf eines bestehenden Softwaresystems einschätzen
zu können wird im Human Centered Design häufig die Methode des
\textit{Interviews im Kontext} gewählt.\cite{contextualDesign} Diese Methode
eignet sich besonders zu Beginn des Entwicklungsprozesses, da wenig
Vorkenntnisse über die eingesetzte Software und das Umfeld, in dem Software
eingesetzt wird, bekannt sein muss. Die Softwareentwickler können so einen
guten Einstieg finden, um einen Überblick zu gewinnen, welche Funktionen die
fertige Software am Ende unterstützen muss. Auch lässt sich durch ein genaues
Beobachten beim Interview herausarbeiten, in welchem Kontext die Software im
tatsächlichen Arbeitsalltag genutzt wird und welche weiteren Faktoren die
Nutzenden der Systeme beeinflussen.


\subsection*{Rahmenbedingungen IiK}
Als erster Schritt wurde ein Termin für ein Interview im Kontext mit \ipName vereinbart. \ipName ist einer von drei
Mitarbeitenden der allgemeinen Studienberatung der Universität Kassel. Zu
seinen Aufgaben gehört die Betreuung der Software Stubegru und deren Einsatz in
der Abteilung Studium und Lehre. Seit über sechs Jahren arbeitet \ipName
bereits gemeinsam mit Hilfskräften an dem Aufbau und der Optimierung der
Software Stubegru um den täglichen Arbeitsalltag seines Teams optimal zu
unterstützen. Ich habe mich persönlich mit \ipName in seinem Büro im Campus
Center der Universität getroffen. Dort hat er mir an seinem Schreibtisch
gezeigt, wie er mit der alten Version der Software Beratungstermine erstellt
und vergeben kann. \ipName saß vor mir und hatte Maus und Tastatur in der Hand.
Ich saß hinter ihm auf einem Stuhl und habe auf einem iPad Notizen
mitgeschrieben. Für die Dauer von einer Stunde hat \ipName mir gezeigt, wie er
die Software aktuelle nutzt, welche Features für ihn sehr wichtig sind und an
welchen Stellen noch Verbesserungspotenzial besteht.

\subsection*{Detaillierter Ablauf IiK}
Am Anfang habe ich \ipName gebeten, mir einmal zu zeigen, wie er einen
Beratungstermin in der Software anlegen und vergeben kann. Dies ist der
Workflow, der im Arbeitsalltag am häufigsten vorkommt und daher eine hohe
Priorität im Designprozess hat. \ipName klickte sich durch die verschiedenen
Eingabefelder um einen freien Zeitslot für einen Beratungstermin anzulegen.
Hierbei erwähnte er, dass es ganz wichtig ist, dass Datum und Uhrzeit des
Beratungstermins mit wenigen Klicks über ein Date-/Timepicker mit der Maus
eingeben werden können. Eine Datumseingabe über die Tastatur würde er nicht
bevorzugen.

\begin{figure}[H]
    \caption{Datepicker im Formular zur Erstellung eines Zeitslots}
    \centering
    \includegraphics[width=0.9\textwidth]{screen_old_datepicker.png}
\end{figure}

Beim Eintragen mehrere Termine wäre es auch besonders praktisch, dass das zuvor
eingegebene Datum stehen bleibt und direkt ein weiterer Zeitslot für den
gleichen Tag angelegt werden kann, ohne dass er nochmal extra das Datum
auswählen muss. Diem meisten der weiteren Felder sind Dropdown Menüs, mit
wenigen Elemente., Die Auswahl der richtigen Werte kann \ipName schnell
vornehmen. Bei der Auswahl der verknüpften Räume werden beispielsweise die
Räume, die mit seinem Nutzeraccount verknüpft sind, ganz oben in der
Auswahlliste angezeigt. Da eine Beratung in der Regel in den eigen Räumen
stattfindet, ist hier eine schnelle Auswahl für den Normalfall möglich. In
einer Spezialsituation, in der ein größerer Beratungstermin beispielsweise in
einem gemeinsamen Gruppenraum stattfinden, ist aber auch solch eine Auswahl
möglich.

\begin{figure}[H]
    \caption{Dropdown zur Auswahl des Beratungsraums. Der eigene Raum wird immer als oberstes angezeigt}
    \centering
    \includegraphics[width=0.9\textwidth]{screen_old_roomdropdown.png}
\end{figure}

Nachdem der Zeitslot für den Termin angelegt ist, wird der entsprechende Tag in
der Kalenderübersicht nun grün hinterlegt. Dies ist ein Zeichen für die
Hilfskräfte der Erstinformation, dass an diesem Tag noch freie Zeitslots
verfügbar sind.

\begin{figure}[H]
    \caption{Kalenderübersicht. Grüne gefärbte Tage zeigen noch freie Zeitslots an. Rot gefärbte Tage weisen auf vergeben Zeitslots hin}
    \centering
    \includegraphics[width=0.9\textwidth]{screen_old_module.png}
\end{figure}

Durch ein Mouseover über den entsprechenden Tag in der Monatsübersicht kann man
die genauen Termine mit Informationen über die Uhrzeit, den zuständigen
Beratenden und die Anzahl der freien Plätze sehen. \ipName erklärt mir, dass
die kompakte Monatsansicht mit den farblich hervorgehobenen Terminslots bereits
eine sehr gute Lösung ist, damit die Hilfskräfte auf einen Blick erfassen
können, an welche Tagen sie den Kunden noch Beratungsgespräche anbieten können.
Sobald alle Plätze der Beratungstermine an einem Tag vergeben sind, wird dieser
im Kalender rot markiert. \glqq So sehen Hilfskräfte mit einem Blick sofort,
dass sie hier keinen Termin mehr vergeben werden können\grqq, erklärt \ipName
\cite{claves}.

\begin{figure}[H]
    \caption{Bewegt man den Mauszeiger über einen Tag, erscheinen weiteren Informationen zu den Zeitslots an diesem Tag}
    \centering
    \includegraphics[width=0.9\textwidth]{screen_old_hover.png}
\end{figure}

Soll nun ein Zeitslot tatsächlich vergeben werden, klickt man auf den
entsprechenden Tag in der Monatsansicht und es öffnet sich ein Modal. Dies ist
ein Fenster, welches sich über den anderen Bildschirminhalt legt und dem Nutzer
somit deutlich anzeigt, dass hier eine Aktion im neu geöffnet Fenster notwendig
ist. \ipName zeigt mir, wie die Mitarbeitenden der Erstinformation in diesem
Detail-View die freien Zeitslots an die ratsuchenden Personen vergeben können.
In einer Liste werden, nach Uhrzeit sortiert, alle Termine untereinander
angezeigt. Neben jedem freien Termin steht ein Button zum Vergabe dieses
Zeitslots zur Verfügung.

\begin{figure}[H]
    \caption{Der Detail-View: Eine Liste mit drei freien Zeitslots am entsprechenden Datum}
    \centering
    \includegraphics[width=0.9\textwidth]{screen_old_daylist.png}
\end{figure}

\ipName zeigt mir wie eine Hilfskraft der Erstinformation nun einen solchen
Zeitslot vergeben könnte. Nach Klick auf den "Vergabe-Button" klappt ein
Formular auf, indem Name, Kontaktdaten und Anliegen der Ratsuchenden erfasst
werden können.

\begin{figure}[H]
    \caption{Formular zum vergeben eines Zeitslots an eine ratsuchende Person}
    \centering
    \includegraphics[width=0.9\textwidth]{screen_old_clientdata.png}
\end{figure}

Nachdem alle personenbezogenen Daten korrekt erfasst wurden kann der Termin nun
endgültig gebucht werden. Hierzu klicken die Hilfskräfte auf den Button
"Bestätigen". \ipName erklärt mir, dass dies ein sehr wichtiger Schritt ist:
Solange eine Mitarbeitender der Erstinformation das Formular zum Erfassen der
persönlichen Daten des Ratsuchenden geöffnet hat, wird dieser Zeitslot mit
einer Sperre versehen. So wird verhindert, dass dieser Zeitslot von einem
Kollegen vergeben werden kann, während man selbst gerade mit dem Ratsuchenden
beispielsweise am Telefon die persönlichen Daten und das Anliegen bespricht.
Sollte nach dem Aufklappen des Formulars der entsprechende Zeitslot doch nicht
vergeben werden, ist es deshalb notwendig, dass die terminvergebende Person auf
"Abbrechen" klickt, um die Sperre dieses Zeitslots aufzuheben und ihn somit für
die Kollegen wieder freizugeben. \ipName betont, dass dieser Schritt manchmal
nicht ganz intuitiv ist, und für die Hilfskräfte daher in Einführungsschulungen
immer besonders hervorgehoben wird. Es wäre allerdings deutlich schlimmer einen
Termin doppelt zu vergeben und somit mindestens einer ratsuchenden Person
wieder absagen zu müssen, als einen Zeitslot versehentlich zu sperren.

\begin{figure}[H]
    \caption{Detail-View: Ein Zeitslot wurde nun vergeben und ist für den entsprechenden Kunden reserviert. Hilfskräfte können nur den Namen des Ratsuchenden einsehen}
    \centering
    \includegraphics[width=0.9\textwidth]{screen_old_assigned_hiwi.png}
\end{figure}

Ist der Termin nun erfolgreich vergeben, können alle Nutzenden der Software
einsehen an welche Person dieser Termin vergeben wurde. Meldet sich ein
Ratsuchender beispielsweise einige Tage später noch einmal bei der
Erstinformation und möchte wissen, wann sein Beratungstermin stattfindet,
können die Mitarbeitenden der Erstinformation diese Auskunft aus der Software
ablesen. Aus Datenschutzgründen können allerdings keine weiteren
personenbezogenen Daten des Beratungstermins ausgelesen werden. Lediglich der
Studienberatende, bei dem der Termin stattfindet, bekommt beim Aufruf des
Detail-Views weitere Details wie Kontaktdaten und Anliegen der ratsuchenden
Person angezeigt.

\begin{figure}[H]
    \caption{Detail-View: Der verantwortliche Beratende kann weitere personenbezogene Details einsehen}
    \centering
    \includegraphics[width=0.9\textwidth]{screen_old_assigned.png}
\end{figure}

\ipName hat nun den zweistufigen Workflow zur Terminvergabe einmal komplett
durchgespielt und mich auf viele Details hingewiesen. Während \ipName mir
gezeigt und erzählt hat, wie die Terminvergabe in der aktuellen Softwareversion
abläuft, habe ich in Stichworten mitgeschrieben, welche Bemerkungen und
Auffälligkeiten er besonders betont hat.

\section{Auswertung des Interviews}

\subsection*{Einleitung Auswertung}
Während bisher der detaillierte Ablauf des Interviews im Kontext geschildert
wurde, sollen im Folgenden die wesentlichen Kernaspekte nochmals
zusammengefasst werden, die während des Interviews notiert wurden. Das
Augenmerk liegt hierbei auf Beobachtungen, die Konsequenzen für den
Designprozess des überarbeiteten Kalendermoduls zur Terminvergabe
hervorbringen.

\subsection*{Methode der Auswertung}
Während dem Interview habe ich mir alle relevant erscheinenden Aussagen von
\ipName auf einem iPad notiert. Wurden im weiteren Gesprächsverlauf noch
ergänzenden Informationen zu den einzelnen Punkten deutlich, habe ich diese in
den Notizen stichpunktartig an die entsprechenden Themen angefügt. Im Nachgang
des Interviews mussten diese Notizen nun sorgfältig analysiert und ausgewertet
werden. Hierzu bin ich die einzelnen Themen durchgegangen und habe die
entsprechenden Ansichten und Klickpfade in der Software nochmals nachgespielt.
In einem neuen Dokument habe ich nun die herausgearbeiteten Problematiken
zusammengefasst um die zu Grunde liegenden Zusammenhänge klarzustellen und zu
spezifizieren. Dies entspricht dem zweiten Schritt des iterativen Design Zyklus
des Human Centered Design nach ISO 9241 \cite{iso9241}

\subsection*{Spannende Erkenntnisse}
\label{subsection:SpannendeErkenntnisse}

Im Folgenden werden nun drei Punkte exemplarisch vorgestellt, die während des
Interviews aufgefallen sind. Anhand dieser drei verschiedenen bestehenden
Probleme wird der Designprozess des Human centered Design beispielhaft
durchlaufen.

\subsubsection{Kompakte Ansicht Kalender (mit Farben)}
In der alten Softwareversion, die an der Uni Kassel bisher zum Einsatz kam,
werden alle freien und vergeben Zeitslots der Beratungstermine in einer
tabellarischen Monatsansicht dargestellt.

\begin{figure}[H]
    \caption{Tabellarische Ansicht der Zeitslots mit Einfärbungen der einzelnen Tage}
    \centering
    \includegraphics[width=0.9\textwidth]{screen_old_module.png}
\end{figure}

Durch die farblichen Markierungen der einzelnen Tage können Nutzenden auf einen
Blick erfassen, ob an diesem Tag Beratungsslots eingetragen wurden und ob unter
den eingetragenen Zeitslots noch freie Termine vorhanden sind. Ein grün
markierter Tag bedeutet, dass an diesem Tag noch mindestens ein freier
Beratungsslot vorhanden ist. Ein rot markierter Tag bedeutet, dass an diesem
Tag Beratungstermine stattfinden, diese allerdings bereits alle an ratsuchende
Personen vergeben sind. In der überarbeiteten Version der Stubegru Software,
die in Zusammenarbeit mit der Hochschule Bremen entstanden ist, wurde diese
kompakte tabellarische Übersicht durch eine größere umfangreiche Ansicht
ausgetauscht, die durch die Javascript Bibliothek \textit{full
    calendar}\cite{fullCalendarWeb} gerendert wird.

\begin{figure}[H]
    \caption{Monatsübersicht der Beratungstermine in der Bremer Version}
    \centering
    \includegraphics[width=0.9\textwidth]{screen_bremen_month_view.png}
\end{figure}

Dieses neue Ansicht ermöglicht auf den ersten Blick zu sehen, zu welcher
Uhrzeit die Termine stattfinden und einzelne Termine aus der Monatsübersicht
direkt anzuklicken. Allerdings bietet diese Ansicht keine Möglichkeit, Tage je
nach freien Plätzen rot oder grün darzustellen. Dies ist jedoch ein wichtiges
Feature für die zweistufige Terminvergabe an der zentralen Studienberatung der
Universität Kassel. An dieser Stelle braucht es eine Idee um den Hilfskräften
der Erstinformation auf den ersten Blick anzuzeigen, ob sie an diesem Tag noch
freie Terminslots vergeben können.

\subsubsection{Suche nach Teilnehmern}
Manchmal kommt es vor, dass Ratsuchende, die bereits einen Beratungstermin
vereinbart haben, nochmals in Kontakt mit der Erstinformation treten, um
weitere Fragen zum Termin zu stellen. Auch kommt es vor, dass das genaue Datum
oder die Uhrzeit vergessen wurden. In diesem Fall sollen die Hilfskräfte der
Erstinformation möglichst schnell Auskunft über die angefragten Details geben
können. Hierfür immer alle vergebenen Beratungstermine manuell durchzulesen,
ist zeitlich ein großer Aufwand. Es braucht also ein Feature, sodass die
Mitarbeitenden der Erstinformationen direkt nach Terminen und weiteren
organisatorischen Daten dieser Termine suchen können. Wenn Ratsuchende
beispielsweise am Telefon ihren Namen nennen, werden sie manchmal nicht
einwandfrei verstanden. Ein Suche nach Teilnehmernamen der Termine sollte also
auch funktionieren, wenn der Name nicht exakt in der gleichen Schreibweise
eingegeben wird, wie er im Datensatz des Beratungstermins in der Datenbank
hinterlegt ist.

\subsubsection{Telefonnummer Anzeige ("Silbentrennung")}
In der Regel wird bei einer Terminvergabe die Telefonnummer der ratsuchenden
Person erfasst. Der zuständige Studienberatende kann den Datensatz bei Bedarf
aufrufen und diese Telefonnummer einsehen. Dies passiert in der Regel, wenn der
Berater vor einem Beratungstermin nochmals telefonisch Details mit der
ratsuchenden Person abklären möchte. Der Berater wählt also die angezeigt
Telefonnummer in seinem Telefon. Während des Interviews im Kontext zeigt sich,
dass die Eingabe längerer Telefonnummern manchmal Fehler mit sich bringt, da
Ziffern vertauscht oder vergessen werden. Den Beratenden wäre hier eine
wertvolle Hilfe an die Hand gegeben, wenn eine Darstellung langer
Telefonnummern möglich wäre, die ein direktes und intuitives eintippen in die
Telefontastatur erleichtern.

\section{Gestaltungslösungen entwickeln}

Nachdem nun die Problematiken und Herausforderungen des neuen Softwaremoduls
verdeutlicht wurden, sollen im nächsten Schritt konkrete Ideen entwickelt
werden, wie die erkannten Problematiken und Anforderungen in der Praxis
umgesetzt werden können. Alan Dix betitelt diese Phase in "Human Computer
Interaction" als "Requirements specification" und betont, dass der Fokus in
diesem Schritt darauf liegt, die notwendigen Funktionalitäten und Features der
Software grob zu beschreiben. Von besonderer Bedeutung in diesem Schritt des
Designzyklus sind Zusammenhänge und Abhängigkeiten zwischen einzelnen
Komponenten. Exakte Implementierungsdetails hingegen sind in dieser Phase noch
nicht von großer Bedeutung und sollten erst im nächste Schritt genauer
betrachtet werden.

\subsection*{Methode der Erarbeitung}
Durch die Auswertung des Interviews im Kontext sind Nutzungsanforderungen an
das neue Modul zur Terminvereinbarung entstanden. Um diese lose formulierten
Nutzungsanforderungen später implementieren zu können, werden sie in diesem
Schritt weiter konkretisiert. Es sollen erste Ideen entstehen, wie die
Bedürfnisse der Nutzenden durch einzelne Komponenten der Software umgesetzt
werden können. In diesem Fall wird mit Skizzen der einzelnen Views und
Formulare gearbeitet. Für jedes Szenario, dass Nutzende beim späteren Verwenden
der Software durchlaufen, wird eine digital gezeichnete Skizze erstellt.
Hierbei werden bereits wichtige Elemente wie Buttons, Formularfelder und
Hinweisboxen skizziert. Durch Markierungen und Notizen an der Skizze werden die
Funktionen dieser Elemente definiert.

\subsubsection{Kompakte Ansicht Kalender (mit Farben)}

Die Übersicht aller Termine eines Monats ist die Ansicht, die Nutzende beim
Aufruf der Software als erstes sehen. Den größten Raum nimmt die tabellarische
Ansicht der einzelnen Tage des Monats ein. In den einzelnen Feldern werden
Terminslots, nach Uhrzeit sortiert, aufgelistet. Neben der Uhrzeit des Termins
wird der Titel eines jeden Termins angezeigt. Die einzelnen Termine werden
farblich entweder grün oder rot eingefärbt, um auf den ersten Blick zu
kennzeichnen, ob es sich um einen freien Terminslot (grün) oder um einen
bereits vergebene Termin (rot) handelt. Wenn an einem Tag viele Zeitslots
angelegt werden, wird das Feld für diesen Tag automatisch größer, sodass alle
Termine Platz finden. Sollten an jedem Tag sehr viel Termine angelegt werden,
könnte die tabellarische Monatsansicht so lang werden, dass sie unter Umständen
nicht mehr vollständig auf den Bildschirm passt. Dies wäre unpraktisch, da dann
nicht mehr alle Termine eines Monats auf einen Blick erfasst werden könnten. In
der Phase der Evaluation sollte Diese Problematik berücksichtigt werden und
eine Abschätzung getroffen werden, wie viele Termine im praktische Einsatz
tatsächlich pro Tag angelegt werden.

\begin{figure}[H]
    \caption{Monatsübersicht der Beratungstermine mit farblichen Markierungen}
    \centering
    \includegraphics[width=0.9\textwidth]{doodle_month_view.jpeg}
\end{figure}

Über der tabellarischen Ansicht der Tage befindet sich eine horizontale Leiste, die den aktuell angezeigten Monat betitelt und Kontrollelemente beinhaltet um in den vorherigen bzw nächsten Monat zu wechseln. Ein Button, um nach einigem hin- und herblättern wieder den aktuelle Monat anzuzeigen, könnte in einigen Anwendungsszenarien viele Klicks ersparen. Über der Leiste mit dem Monat befindet sich eine weitere Kontrollleiste. Diese enthält einen Button um einen neuen Zeitslot anzulegen. Dieser Button sollte nur für Nutzeraccounts von Beratenden sichtbar sein. Hilfskräfte der Erstinformation sollen Zeitslots nur vergeben, aber nicht selbst anlegen können. Daneben befindet sich eine Suchleiste um schnell nach Namen von ratsuchenden Personen suchen zu können. Ganz rechts gibt es schließlich noch einen Button um weitere Einstellungen vorzunehmen. Durch einen Klick auf diesen Button mit einem Zahnrad Symbol soll ein Dropdown-Menü aufklappen, in dem Filter für die Ansicht der Termine gesetzt werden können.

\begin{figure}[H]
    \caption{Filtereinstellungen der Kalenderansicht. Das Dropdown Menü öffnet sich durch Klick auf den Zahnrad Button}
    \centering
    \includegraphics[width=0.9\textwidth]{example-image-a}
\end{figure}

In diesem Menü kann über Toggles eingestellt werden, ob nur eigene Termine oder
auch fremde Termine in der Monatsansicht dargestellt werden sollen. Mit
\textit{eigenen Terminen} sind Termine gemeint, die den eigenen Benutzeraccount
als zuständigen Beratenden hinterlegt haben. Außerdem kann ein Filter gesetzt
werden um ausschließlich freie Termine anzuzeigen. Dies kann besonders für
Hilfskräfte bei der Vergabe freier Termine relevant sein, da bereits vergeben
Zeitslots in diesem Fall irrelevante Informationen sind, die von freien
Zeitslots ablenken.

\subsubsection{Suche nach Teilnehmern}

Die Suchfunktion ist ein weiterer Aspekt, dem in dieser Ausarbeitung besonderer
Aufmerksamkeit gewidmet ist. Über das Freitextfeld in der oberen Kontrollleiste
können Nutzende nach Namen von Ratsuchenden suchen, an die bereits Termine
vergeben wurden. Tippt man einige Buchstaben in das Suchfeld ein, klappt eine
Box mit Ergebnisvorschlägen unter der Suchleiste auf und schiebt den restlichen
Inhalt (die tabellarische Monatsansicht) nach unten. In diese Box werden zur
Suchanfrage passenden Termine dargestellt. Für jeden Termin wird in einer Zeile
der Titel, der Name des Ratsuchenden, der Name des Beratenden sowie Datum und
Uhrzeit aufgelistet. Neben jedem Datensatz erscheint ein Button mit einem
Augensymbol. Durch einen Klick darauf wird der entsprechende Termin in der
Detailansicht geöffnet.

\begin{figure}[H]
    \caption{Suche nach Terminen eines Ratsuchenden mit Ergebnisliste}
    \centering
    \includegraphics[width=0.9\textwidth]{doodle_search_view.jpeg}
\end{figure}

Wichtig für die Suchfunktion ist, dass passende Ergebnisse auch angezeigt
werden, wenn die Eingabe in der Suchleiste eventuell Fehler enthält oder noch
nicht vollständig ist. Durch solche automatischen Ergebnisvorschläge wird das
Suchen für die Nutzenden erleichtert und Fehlerquellen minimiert. Dadurch, dass
Nutzenden schon während dem Tippen der ersten Buchstaben ein aktives und
konstruktives Feedback erhalten, fühlt sich die Nutzung der Software
dynamischer und flüssiger an. \cite{autoCompletion} Wenn Mitarbeitende der
Erstinformation ihre Kunden am Telefon beispielsweise nicht ganz genau
verstehen, können sie mit diesen automatischen Ergebnisvorschlägen trotzdem den
passenden Termin finden. Allerdings muss bei solchen automatisiertenVorschlägen
darauf geachtet werden, dass nicht zu viele unnötige oder unpassende Vorschläge
angezeigt werden. Diese würden Nutzende von den eigentlich gesuchten
Ergebnissen ablenken und sich somit nachteilig auf die User-Experience
auswirken. \cite{autosuggModeration}

\subsubsection{Telefonnummeranzeige ("Silbentrennung")}

Durch einen Klick auf den Termin in der Monatsübersicht öffnet sich die
Detailansicht des zugehörigen Termins und weitere Eigenschaften des Datensatzes
werden angezeigt. Alternativ kann ein Termin auch über die Suchfunktion
gefunden und dann über den Button mit dem Augensymbol in der Detailansicht
aufgerufen werden. In dieser Ansicht können Nutzeraccounts mit der
entsprechenden Berechtigung nochmals Details des Termins bearbeiten oder den
Termin löschen.

\begin{figure}[H]
    \caption{Detailansicht eines Termins, der bereits an eine ratsuchende Person vergeben wurde}
    \centering
    \includegraphics[width=0.9\textwidth]{doodle_client_details.jpeg}
\end{figure}

Wenn Beratende nach der Vereinbarung eines Termins nochmals auf telefonischem
Weg Absprachen oder Vorgespräche mit den Ratsuchenden erledigen möchten, können
sie die Telefonnummer der entsprechenden Person in der Detailansicht eines
Beratungstermins einsehen. Die Beobachtung des Nutzungsverhaltens während des
Interviews im Kontext hat gezeigt, dass es umständlich ist, lange
Telefonnummern zu erkennen und korrekt in die Tastatur des Telefons einzugeben.
Im Gespräch mit \ipName kam der Wunsch auf, Telefonnummern an dieser Stelle so
zu formatieren, dass sie intuitiver erfasst und abgetippt werden können. Der
Standard für das Formatieren von Telefonnummern in Deutschland wird durch DIN
5008 geregelt. Diese Norm beschäftigt sich mit Formatierungsstandards für
Briefe und Anschreiben. Hier wird das Trennen der Vorwahl vom Rest der Nummer
durch ein Leerzeichen vorgeschrieben. Weitere Formatierung, wie beispielsweise
das aufteilen der Ziffern in kleinere Blöcke wird hier nicht thematisiert.
\cite{din5008}

\begin{figure}[H]
    caption{So werden nationale Festnetz- und Mobilfunknummern nach DIN 5008 richtig geschrieben. Quelle: \cite{phoneFormatBlog}}
    \centering
    \includegraphics[width=0.9\textwidth]{grafik-telefonnummer-national.png}
\end{figure}

Wissenschaftliche Untersuchungen zeigen, dass das Eingeben und Ablesen von
Telefonnummern in Interaktion mit den entsprechenden Maschinen ein relevantes
Details ist. Dieser Prozess sollte durch die technischen Systeme möglichst
intuitiv und nutzerfreundlich gestaltet werden. \cite{humCompPhoneNumbers}.
Eine Unterteilung der Ziffern in kleinere Blöcke, von beispielsweise vier
Ziffern pro Block, erhört die Lesbarkeit deutlich und ermöglicht es dem
menschlichen Gehirn einen Ziffernblock in einem Blick direkt zu erfassen und
auf die Telefontastatur zu übertragen. \cite{phoneFormatBlog}
\cite{numberRecognition} \cite{numberRepres}