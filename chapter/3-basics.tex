\chapter{Nutzungskontext}
\label{chapter:user-context}

Um den Bedarf und Entstehungsprozess der Software besser einordnen zu können,
werden nun sowohl der Kontext als auch die Prozessabläufe der Nutzergruppe kurz
skizziert. Die Beschreibung der Situation im Team hilft zu erkennen, wie sich
die aktuelle Softwarelösung in den Arbeitsalltag eingliedert, welche
Prozessabläufe bereits gut durch Software begleitet werden, und an welchen
Stellen noch Optimierungsbedarf besteht.

\section{Studienberatung der Universität Kassel}
Als Nutzergruppe in dieser Bachelorarbeit werden die Mitarbeitenden der
Abteilung \textit{Studium und Lehre} der Hochschulverwaltung an der Universität
Kassel dienen. Zu deren alltäglichen Aufgaben gehört es, alle erdenklichen
Organisationen zu übernehmen, die Studierenden und Lehrenden ein erfolgreiches
Zusammenarbeiten an der Universität ermöglichen. Hierzu gehören beispielsweise
das Einschreiben und Exmatrikulieren von Studierenden, die Durchführung des
Bewerbungsverfahrens, das Betreiben der Information Studium und die allgemeine
Studienberatung. In dieser Arbeit wird der Fokus auf die Mitarbeitenden der
Studienberatung als Teilgruppe der Abteilung Studium und Lehre gesetzt.

Die allgemeine Studienberatung der Universität Kassel berät Studierende zu
allen Fragen rund um das Studium. Insbesondere bei persönlichen Problemen mit
der Fertigstellung des eigenen Studiums hilft die Studienberatung mit einem
persönlichen Lösungsgespräch und kann an weitere fachspezifische
Beratungsstellen weiter vermitteln. Des Weiteren bietet die allgemeine
Studienberatung verschiedene Workshops und Seminare an. Hierbei können sich
Studierende mit Fokus auf bestimmten Fragestellungen austauschen und
Qualifikationen im Umgang mit herausfordernden Studiensituationen erlangen.
Auch Schnupperkurse für Schüler werden von der allgemeinen Studienberatung
angeboten um den Studieninteressierten einen möglichst unmittelbaren Einblick
in den Studienalltag zu gewähren \cite{studBeratungKsWeb}.

Eines der zentralen Themen im Alltag der Studienberatung sind Beratungstermine.
Studienberatende müssen Termine mit den Ratsuchenden vereinbaren und abstimmen.
Beratungstermine können über verschiedene Kontaktkanäle stattfinden: Es ist
eine telefonische Beratung oder auch eine Beratung über eine Videokonferenz
möglich. Ebenso ist es auch möglich einen persönlichen Beratungskontakt vor Ort
zu vereinbaren. Über all diese Termine muss jeder Studienberatende den
Überblick behalten und gleichzeitig neue Terminanfragen schnell beantworten
können. Um diesen Prozess zu erleichtern und mögliche Fehler, wie
beispielsweise Terminüberschneidungen, zu minimieren, wird hierfür die Software
\gls{Stubegru} eingesetzt.

\section{Aktuelle Softwarelösung: Stubegru}
Stubegru ist ein umfangreiches Softwarepaket für akademische Beratungsstellen.
Die webbasierte Groupware begleitet viele Arbeitsabläufe im Alltag einer
Beratungsstelle an einer Hochschule. In einem Softwaresystem vereint Stubegru
eine Wissensdatenbank in Form eines Wikis, sowie ein Dashboard mit vielen
Modulen für spezifische Workflows. So können über Stubegru an der Universität
Kassel beispielsweise Abwesenheiten der Abteilung, Telefonnotizen und
Beratungskontakte verwaltet werden. Jeder Mitarbeitende der Abteilung hat über
einen eigenen Account Zugriff auf die Software, die er im Browser aufrufen
kann. Die Software hilft dabei, tagesaktuelle Informationen schnell und
übersichtlich allen Mitarbeitenden zur Verfügung zu stellen und bei Bedarf
langfristige und ausführliche Informationen mit wenigen Klicks zur Verfügung zu
stellen \cite{stubegruWebsite}.

Das wichtigste Modul der eingesetzten Software für die Studienberatung ist der
Kalender zur Terminvereinbarung von Beratungsterminen. Über dieses Modul können
in einem zweistufigen Prozess Termine für Ratsuchende freigegeben und an die
entsprechenden Studierenden und Studieninteressierten vermittelt werden.

\subsection*{Zeitslots erstellen}
Im ersten Schritt können die Studienberatenden freie Zeitslots für ihre
Beratungstermine anlegen. Diese Zeitslots zeigen an, dass der entsprechende
Beratende in der eingestellten Zeitspanne potenziell Zeit für ein
Beratungsgespräch hat. Bei der Erstellung der Zeitslots können weitere
Attribute wie der Beratungskanal (Online Meeting, Telefongespräch oder
Präsenztermin) konfiguriert werden. Außerdem können Mail Templates verknüpft
werden, die im Falle einer Terminvergabe den Ratsuchenden per Email über alle
wichtigen Informationen zum Termin informieren.

\subsection*{Terminvergabe durch Erstinformation}
Im zweiten Schritt werden die eingestellten Zeitslots durch Hilfskräfte der
Erstinformation an Ratsuchende vergeben. Die Erstinformation der Universität
Kassel berät Studierende und Studieninteressierte zu allen Fragen rund ums
Studium übers Telefon, Email und an einer Servicetheke vor Ort. Bei
tiefgehenden Fragen und spezifischen Anliegen verweisen die Mitarbeitenden an
die entsprechenden Sachbearbeitenden oder Beratungsstellen. Die Erstinformation
ist auch für das Vereinbaren von Beratungsterminen mit der allgemeinen
Studienberatung verantwortlich. Sind die Mitarbeitenden der Erstinformation in
Kontakt mit einem Kunden, der einen Termin in Anspruch nehmen möchte, können
sie in der Software alle freien Zeitslots der Beratenden einsehen und einen
passenden Termin mit dem Kunden vereinbaren. Wenn ein freier Zeitslot vergeben
wird und fest mit einem Kunden verknüpft ist, wird eine Email an den Beratenden
versendet, die über alle Details wie Adresse, Kontaktinformationen und Anliegen
der Ratsuchenden informiert. Des Weiteren wird eine Mail an den Kunden
versendet, die auf dem zuvor verknüpften Mailtemplate aufbaut und dynamisch
terminrelevante Informationen einsetzt, wie beispielsweise Datum und Uhrzeit
des Termins, oder eine Wegbeschreibung zum Beratungsraum.

\subsection*{Auskunft bei Terminabsage}
Eine weitere Verwendung des Kalendermoduls tritt ein, wenn Kunden der
Erstinformationen Fragen zu einem bereits vereinbarten Termin haben oder diesen
absagen möchten. In diesem Fall können die Hilfskräfte der Erstinformation über
eine Suchfunktion gezielt nach den vereinbarten Terminen des Kunden suchen und
weitere Auskünfte geben.

\subsection*{Datenschutz}
Da Datenschutz in Beratungsszenarien eine wichtige Rolle einnimmt, kann
lediglich der verantwortliche Beratende das Anliegen der ratsuchenden Person
einsehen. Zu Auskunftszwecken können aber alle Mitarbeitenden der Abteilung
sehen, wann ein Beratungstermin mit welchem Beratenden vereinbart wurde.
Datensätze zu vergangenen Beratungsterminen werden täglich gelöscht, sodass
möglichst wenig personenbezogenen Daten in der Datenbank gespeichert werden
müssen. Über ein differenziertes Berechtigungssystem der Software Stubegru kann genau gesteuert werden, welche Nutzungsgruppen Beratungstermine anlegen
und vergeben dürfen.

\section{Historische Entwicklung der Software Stubegru}
Die Software Stubegru wurde ursprünglich von mir als Hilfskraft der Abteilung
Studium und Lehre an der Universität Kassel erstellt und betreut. Da die
Software nun langfristig an der Uni Kassel und auch an anderen deutschen
Hochschulen eingesetzt werden soll, wurde ein Prozess gestartet, um eine
Professionalisierung und nachhaltige Betreuung der Software zu gewährleisten.
In diesem Rahmen wurde die Software auch für andere Hochschulen zur Verfügung
gestellt und unter einer OpenSource Lizenz veröffentlicht. Im Zuge dieser
Veröffentlichung wurde in Zusammenarbeit mit der Hochschule Bremen eine
grundlegend überarbeitete Variante der Software Stubegru erstellt, die im
Vergleich zu der bisherigen Version deutlich flexibler ist und mehr
Anpassungsmöglichkeiten bietet. Wie Erich Gamma in \textit{Elemente
    wiederverwendbarer objektorientierter Software} betont, bieten wiederverwendbar
gestaltete Softwarestrukturen und abstrakte Implementierungsansätze die Option,
Softwaremodule in verschiedenen Kontexten zu verwenden. Dies erfordert im
Softwareentwurf allerdings eine vorausschauende Planung und einen hohen
Abstraktionsgrad \cite{wiederverwSoftware}. Somit ist der Einsatz der Software
Stubegru nun an verschiedenen Hochschulen mit verschiedenen Arbeitsabläufen
realisierbar. Um von dieser neuen, überarbeiteten Softwareversion auch an der
Universität Kassel zu profitieren, ist wiederum eine grundlegende
Überarbeitung des Moduls zur Terminvergabe der Beratungstermine für die
allgemeinen Studienberatung notwendig. Dieser Prozess soll im Rahmen dieser
Bachelorarbeit begleitet, wissenschaftlich untermauert und evaluiert werden.

\subsection*{Fehlende Features}
In der bisherigen Softwareversion gibt es einige Features, die noch nicht
vollständig funktionieren oder nicht optimal auf den tatsächlichen
Arbeitsalltag zugeschnitten sind. An diesen Stellen soll das neue Modul zur
Terminvereinbarung verbessert und noch weiter an die Bedürfnisse der Nutzenden
angepasst werden. Im Rahmen der Softwareüberarbeitung mit der Hochschule Bremen
wurde für die neue Version von Stubegru bereits ein Kalendermodul zur Vergabe
von Beratungsterminen entwickelt. Dieses Modul weist allerdings noch einige
Probleme auf, um reibungslos im Arbeitsalltag der allgemeinen Studienberatung
an der Universität Kassel eingesetzt zu werden. In der \textit{Bremer Version} läuft
das Erstellen und Vergeben eines Beratungstermins in einem einzigen Schritt ab.
Ein zentraler Punkt, um die überarbeitete Software auch in der allgemeinen
Studienberatung der Universität Kassel nutzen zu können, ist der zweistufige
Prozess der Terminvergabe. Hier müssen Beratende die Möglichkeit haben, zuerst
freie Terminslots freizugeben, die dann in einem getrennten zweiten Schritt
durch Mitarbeitende der Erstinformation an Ratsuchende vergeben werden können.

Welche Anpassungen im Detail notwendig sind, um die Software optimal in der
Studienberatung einsetzen zu können, soll in den folgenden Kapiteln methodisch
herausgearbeitet werden und durch Dokumentation von praktisch durchgeführten
Nutzerstudien und Gesprächen mit den verantwortlichen Personen ergänzt werden.
Diese Bachelorarbeit soll insbesondere den Designprozess strukturiert begleiten
und wissenschaftliche Methoden aufzeigen, um Softwareentwicklern und Nutzenden
eine möglichst gute Zusammenarbeit zu ermöglichen.