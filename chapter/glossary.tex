\makeglossaries

\newglossaryentry{Bootstrap}
{
    name=Bootstrap,
    description={Bootstrap ist ein ursprünglich von Twitter entwickeltes \gls{Css} Framework, das standardisierte \gls{Html} Elemente definiert. Somit können Softwaredesignende häufig benötigte Gestaltungselemente direkt verwenden und können gleichzeitig eine einheitliche und vertraute User-Experience bieten.\\ \\
            \textit{In dieser Arbeit verwendete Version: \textbf{Bootstrap v3.3.4}}
            \cite{Bootstrap}}
    \\\textit{--Begriff kommt vor auf Seite: }
}

\newglossaryentry{Tech-Stack}
{
    name=Tech Stack,
    description={Ein Tech Stack beschreibt die Technologien, die zur Entwicklung einer spezifischen Software zum Einsatz kommen. Dazu zählen beispielsweise die verwendeten Programmiersprachen, Bibliotheken, Frameworks oder auch Entwicklungswerkzeuge.}
    \\\textit{--Begriff kommt vor auf Seite: }
}

\newglossaryentry{Stubegru}
{
    name=Stubegru,
    description={Stubegru ist ein Softwarepaket für akademische Beratungsstellen. Als webbasierte Groupware kann Stubegru typische Workflows in einer Studienberatung begleiten: Beratungstermine, Wissensdatenbank, Abwesenheitsmanagement und vieles mehr
            \cite{stubegruWebsite}. Die Software Stubegru wurde gemeinsam von Johannes Schnirring und der Universität Kassel entwickelt. In Abschnitt \ref{section:stubegru-refs} finden sich weitere Links und Referenzen zur Software.}
    \\\textit{--Begriff kommt vor auf Seite: }
}

\newglossaryentry{Usertest}
{
    name=Usertest,
    description={Fertige Software oder Prototypen werden den Nutzenden vorgestellt. Diese sollen vorgegebene Prozesse in der Software durchlaufen. Softwareentwickelnde und Gestaltende beobachten wie die Nutzenden mit dem System interagieren. Anschließend kann das Feedback der Nutzenden eingeholt werden.}
    \\\textit{--Begriff kommt vor auf Seite: }
}

\newglossaryentry{Timepicker}
{
    name=Timepicker,
    description={Kleines grafisches Popup, um Uhrzeiten elegant mit der Maus eingeben zu können. Ähnlich wie ein Datepicker, der allerdings zur grafischen Eingabe eines Datum genutzt werden kann.
            \cite{datepicker}}
    \\\textit{--Begriff kommt vor auf Seite: }
}

\newglossaryentry{UML}
{
    name=UML,
    description={Abkürzung für Unified Modeling Language (vereinheitlichte Modellierungssprache).UML beschreibt Konventionen um Datenmodelle und Workflows von Software und technischen Systemen grafische zu modellieren.
            \cite{UML}}
    \\\textit{--Begriff kommt vor auf Seite: }
}

\newglossaryentry{View}
{
    name=View,
    description={Ein View beschreibt eine Ansicht, die Nutzende zu einem Zeitpunkt beim Verwenden einer Software auf dem Bildschirm grafisch angezeigt bekommen. Eine tabellarische Übersicht von Datensätzen oder eine Eingabemaske für einen neuen Datensatz sind Beispiele für einzelne Views.}
    \\\textit{--Begriff kommt vor auf Seite: }
}

\newglossaryentry{Modal}
{
    name=Modal,
    description={Im \gls{Css} Framework \gls{Bootstrap} bezeichnet ein Modal eine Art Popup Fenster. Dieses Fenster legt sich über den bisherigen Bildschirminhalt und zieht somit die Aufmerksamkeit der Nutzenden auf sich. Aufforderungen für Nutzereingaben, oder Detailansichten eines Datensatzes werden oftmals mit Modals umgesetzt.}
    \\\textit{--Begriff kommt vor auf Seite: }
}

\newglossaryentry{Javascript}
{
    name=Javascript,
    description={JavaScript (Abk. JS) ist eine Skriptsprache, welche für interaktive Inhalte in Webbrowsern entwickelt wurde. Mit Javascript können Inhalte von \gls{Html} Seiten angepasst oder dynamisch nachgeladen werden. Außerdem können Interaktionen vom Nutzenden (z.B. Klicken, Scrollen, Tastatureingaben) verarbeitet und ausgewertet werden.\\ \\
            \textit{In dieser Arbeit verwendete Version: \textbf{ECMAScript 2022 (ES13)}}
            \cite{Javascript}}
    \\\textit{--Begriff kommt vor auf Seite: }
}

\newglossaryentry{Html}
{
    name=Html,
    description={Abkürzung für Hypertext Markup Language. Html ist eine textbasierte Sprache um Dokumente, insbesondere Internetseiten zu Strukturieren. Über Html kann der Text auf einer Internetseite dargestellt, formatiert und positioniert werden. Außerdem können grafische Elemente wie Hyperlinks, Bilder oder Videos eingebunden werden.\\ \\
            \textit{In dieser Arbeit verwendete Version: \textbf{HTML 5}}
            \cite{Html}}
    \\\textit{--Begriff kommt vor auf Seite: }
}

\newglossaryentry{Php}
{
    name=Php,
    description={Php ist eine Programmiersprache um serverseitige Befehle auszuführen. Php Skripte werden häufig verwendet um Datenbankoperationen auf einem Webserver auszuführen und die Ergebnisse an einen Webbrowser zu schicken.\\ \\
            \textit{In dieser Arbeit verwendete Version: \textbf{PHP 8}}
            \cite{Php}}
    \\\textit{--Begriff kommt vor auf Seite: }
}

\newglossaryentry{Http}
{
    name=Http,
    description={Die Abkürzung Http steht für Hypertext Transfer Protocol. Dieses Protokoll wurde von Tim Berners-Lee entwickelt um \gls{Html} Dokumente über das World Wide Web verfügbar zu machen. Das Hypertext Transfer Protocol definiert Standards, wie eine Website vom Server zum Client übertragen wird.
            \cite{http}}
    \\\textit{--Begriff kommt vor auf Seite: }
}

\newglossaryentry{MySQL}
{
    name=MySQL,
    description={MySQL ist ein Datenbankverwaltungssystem, das von Oracle entwickelt und maintained wird. MySQL ist OpenSource und kann somit kostenfrei von Softwarentwickelnden genutzt werden um relationale Datenbanken aufzusetzen.\\ \\
            \textit{In dieser Arbeit verwendete Version: \textbf{10.9.4-MariaDB}}
            \cite{mariadb}}
    \\\textit{--Begriff kommt vor auf Seite: }
}

\newglossaryentry{OpenSource}
{
    name=OpenSource,
    description={Der Begriff OpenSource beschreibt Softwareprojekte deren Code für alle Menschen frei zugänglich ist. Somit kann die Software nicht nur kostenfrei genutzt werden, sondern auch von jedem Einzelnen verändert, weiterentwickelt und verteilt werden.}
    \\\textit{--Begriff kommt vor auf Seite: }
}

\newglossaryentry{JSON}
{
    name=JSON,
    description={JSON steht für JavaScript Object Notation und ist eine Beschreibungssprache. Mithilfe von JSON können Datensätze in textueller Form dargestellt werden. Diese Darstellung kann sowohl von Menschen, als auch von Computern einfach gelesen werden. Daher wird JSON inzwischen nicht nur im Zusammenhang mit \gls{Javascript} verwendet, sondern hat sich als Standard für den Datenaustausch im Internet durchgesetzt.
            \cite{json}}
    \\\textit{--Begriff kommt vor auf Seite: }
}

\newglossaryentry{Document-Object-Model}
{
    name=Document Object Model,
    description={Das Document Object Model (kurz DOM) definiert eine Programmierschnittstelle um \gls{Html} Dokumente als Baumstruktur darzustellen. Über das DOM kann zum Beispiel mit \gls{Javascript} auf einzelne Html Elemente zugegriffen werden, um diese zu verändern oder auszulesen,}
    \\\textit{--Begriff kommt vor auf Seite: }
}

\newglossaryentry{PDO}
{
    name=Php Data Objects,
    description={Die Php Data Objects Erweiterung (PDO) definiert eine Schnittstelle, um in \gls{Php} Skripten auf Datenbanken zuzugreifen. \gls{MySQL} definiert beispielsweise eine PDO-Schnittstelle, die es ermöglicht aus Php Skripten heraus auf Datensätze in der Datenbank zuzugreifen und diese zu bearbeiten.}
    \\\textit{--Begriff kommt vor auf Seite: }
}

\newglossaryentry{fetch-api}
{
    name=fetch API,
    description={Die fetch API ist eine Schnittstelle zum dynamischen Abrufen und Versenden von Daten über \gls{Http} Requests und kann in \gls{Javascript} standardmäßig verwendet werden. Im Vergleich zu den früher gebräuchlichen XMLHttpRequests bietet die fetch API eine deutliche einfachere Syntax sowie mächtige Funktionalitäten um dem asynchronen Charakter von Anfragen an einen entfernten Server gerecht zu werden
            \cite{fetchAPI}.}
    \\\textit{--Begriff kommt vor auf Seite: }
}

\newglossaryentry{Css}
{
    name=Css,
    description={Abkürzung für Cascading Style Sheets. Css ist eine textbasierte Stylesheet-Sprache und wird in Verbindung mit \gls{Html} genutzt um die Position, Farbe und Größe der einzelnen Html Elemente zu definieren.\\ \\
            \textit{In dieser Arbeit verwendete Version: \textbf{CSS 3}}
            \cite{Css}}
    \\\textit{--Begriff kommt vor auf Seite: }
}

\newglossaryentry{fullcalendar}
{
    name=fullcalendar,
    description={Fullcalendar ist eine \gls{Javascript} Bibliothek und stellt fertige Funktionen zur Verfügung um Kalenderansichten und Termindaten einfach und übersichtlich in einem \gls{Html} Dokument anzuzeigen.\\ \\
            \textit{In dieser Arbeit verwendete Version: \textbf{FullCalendar v5.8.0}}
            \cite{fullCalendarWeb}}
    \\\textit{--Begriff kommt vor auf Seite: }
}

\newglossaryentry{jQuery}
{
    name=jQuery,
    description={jQuery ist eine \gls{Javascript} Bibliothek und stellt fertige Funktionen zur Verfügung um \gls{Html} Elemente dynamisch und effizient auszulesen, zu verändern oder einzublenden.\\ \\
            \textit{In dieser Arbeit verwendete Version: \textbf{jQuery v3.6.0}}
            \cite{jQuery}}
    \\\textit{--Begriff kommt vor auf Seite: }
}