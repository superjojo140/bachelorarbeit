\makeglossaries



\newglossaryentry{Usertest}
{
    name=Usertest,
    description={Fertige Software oder Prototypen werden den Nutzenden vorgestellt. Diese sollen vorgegebene Prozesse in der Software durchlaufen. Softwareentwickelnde und Gestaltende beobachten wie die Nutzenden mit dem System interagieren. Anschließend kann das Feedback der Nutzenden eingeholt werden.}
}

\newglossaryentry{Timepicker}
{
    name=Timepicker,
    description={Kleines grafisches Popup, um Uhrzeiten elegant mit der Maus eingeben zu können. Ähnlich wie ein Datepicker, der allerdings zur grafischen Eingabe eines Datum genutzt werden kann. \cite{datepicker}}
}

\newglossaryentry{UML}
{
    name=UML,
    description={Abkürzung für Unified Modeling Language (vereinheitlichte Modellierungssprache).UML beschreibt Konventionen um Datenmodelle und Workflows von Software und technischen Systemen grafische zu modellieren.\cite{UML}}
}

\newglossaryentry{View}
{
    name=View,
    description={Ein View beschreibt eine Ansicht, die Nutzende zu einem Zeitpunkt beim Verwenden einer Software auf dem Bildschirm grafisch angezeigt bekommen. Eine tabellarische Übersicht von Datensätzen oder eine Eingabemaske für einen neuen Datensatz sind Beispiele für einzelne Views.}
}

\newglossaryentry{Modal}
{
    name=Modal,
    description={Im \gls{Css} Framework \gls{Bootstrap} bezeichnet ein Modal eine Art Popup Fenster. Dieses Fenster legt sich über den bisherigen Bildschirminhalt und zieht somit die Aufmerksamkeit der Nutzenden auf sich. Aufforderungen für Nutzereingaben, oder Detailansichten eines Datensatzes werden oftmals mit Modals umgesetzt.}
}

\newglossaryentry{Javascript}
{
    name=Javascript,
    description={JavaScript (Abk. JS) ist eine Skriptsprache, welche für interaktive Inhalte in Webbrowsern entwickelt wurde. Mit Javascript können Inhalte von \gls{Html} Seiten angepasst oder dynamisch nachgeladen werden. Außerdem können Interaktionen vom Nutzenden (z.B. Klicken, Scrollen, Tastatureingaben) verarbeitet und ausgewertet werden.\cite{Javascript}}
}

\newglossaryentry{Html}
{
    name=Html,
    description={Abkürzung für Hypertext Markup Language. Html ist eine textbasierte Sprache um Dokumente, insbesondere Internetseiten zu Strukturieren. Über Html kann der Text auf einer Internetseite dargestellt, formatiert und positioniert werden. Außerdem können grafische Elemente wie Hyperlinks, Bilder oder Videos eingebunden werden.\cite{Html}}
}

\newglossaryentry{Php}
{
    name=Php,
    description={Php ist eine Programmiersprache um serverseitige Befehle auszuführen. Php Skripte werden häufig verwendet um Datenbankoperationen auf einem Webserver auszuführen und die Ergebnisse an einen Webbrowser zu schicken.\cite{Php}}
}

\newglossaryentry{Css}
{
    name=Css,
    description={Abkürzung für Cascading Style Sheets. Css ist eine textbasierte Stylesheet-Sprache und wird in Verbindung mit \gls{Html} genutzt um die Position, Farbe und Größe der einzelnen Html Elemente zu definieren.\cite{Css}}
}

\newglossaryentry{fullcalendar}
{
    name=fullcalendar,
    description={Fullcalendar ist eine \gls{Javascript} Bibliothek und stellt fertige Funktionen zur Verfügung um Kalenderansichten und Termindaten einfach und übersichtlich in einem \gls{Html} Dokument anzuzeigen.\cite{fullCalendarWeb}}
}

\newglossaryentry{Bootstrap}
{
    name=Bootstrap,
    description={Bootstrap ist ein ursprünglich von Twitter entwickeltes \gls{Css} Framework, dass standardisierte \gls{Html} Elemente definiert. Somit können Softwaredesignende häufig benötigte Gestaltungselemente direkt verwenden und könne gleichzeitig eine einheitliche und vertraute User-Experience bieten.\cite{Bootstrap}}
}

\newglossaryentry{jQuery}
{
    name=jQuery,
    description={jQuery ist eine \gls{Javascript} Bibliothek und stellt fertige Funktionen zur Verfügung um \gls{Html} Elemente dynamisch und effizient auszulesen, zu verändern oder einzublenden.\cite{jQuery}}
}
