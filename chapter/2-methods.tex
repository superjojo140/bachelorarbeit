% Hier wird die Problemstellung im Kontext
% einer Anwendung dargestellt und der Inhalt der Arbeit kurz vorweggenommen. Welches Problem wurde
% gelöst, warum ist das relevant? Wie ist die Vorgehensweise? Was wurde thematisch eingegrenzt, was
% wurde ausgegrenzt? Wichtig ist es, den eigenen Beitrag in wenigen prägnanten Sätzen herauszuarbeiten. (Das Fazit soll sich zum Schluss auf diese Beiträge beziehen, um der Arbeit eine erzählerische
% Klammer zu geben.) Die Einleitung endet mit einem Überblick über die Arbeit – hierbei werden die
% Inhalte der einzelnen Kapitel knapp umschrieben

\chapter{Methoden}

\section{Human Centered Design}
Human Centered Design ist eine Methode die den Entwicklungsprozess von
interaktiven Systemen und Software beschreibt. Die Kernthese des Human Centered
Design ist, dass die Nutzenden der Systeme im Mittelpunkt aller Entwicklungs-
und Designprozesse stehen.

Die Entwicklung von Software war in den Anfängen geprägt von den technisch sehr
eingeschränkte Möglichkeiten des Computers. Software wurde also größtenteils so
entwickelt, dass sie möglichst problemlos und effizient auf dem entsprechenden
Computer ausgeführt werden kann. Alle Entwicklungsprozesse wurden also aus
Perspektive der ausführenden Maschine gedacht. Der Ansatz des Human Centered
Design setzt dieser maschinenzentrierten Vorgehensweise eine klare Alternative
entgegen. Martin Ludwig Hofmann beschreibt den Grundgedanken in \glqq{} Human
Centered Design\grqq{} so: \glqq{} [Es geht] nicht darum, mit Geräten zu
denken, sondern mit Menschen und ihren Weltanschauungen\grqq{}. \cite{hcd} Wie
im Kapitel Motivation ausführlich erläutert gewinnt dieser Ansatz in der
heutigen Zeit immer größere Bedeutung. Wie Alan Dix in \glqq{} Human Computer
Interaction\grqq{} erklärt, hat sich die Art und Weise, wie Menschen und
Computer interagieren, in den letzten Jahrzehnten stark verändert. Während
Computer früher meist nur mathematische Berechnungen durchführten, sind
Softwaresysteme heute hochgradig interaktiv und sollen von allen Teilen der
Gesellschaft ungehindert genutzt werden können. \cite{hci}

Um die Methodik des Human Centered Design weiter zu spezifizieren ist zunächst
festzuhalten, dass der Designprozess von Software als solches eine wichtige
Stellung einnimmt. Es reicht nicht aus, dass eine Software korrekt
funktioniert, sich also alle Berechnungen und Prozesse fehlerfrei ausführen
lassen. Im Human Centered Design liegt der Fokus auf der Interaktion zwischen
dem Nutzenden und der Maschine. Man betrachtet, wie die Nutzenden mit dem
Computer interagieren. Wie stoßen sie Prozesse an? Welche Reaktionen erwarten
sie von dem System? Die Nutzenden sollen mit den Systemen möglichst intuitiv
interagieren können. Die Benutzeroberfläche der Software soll selbst erklären,
welche Funktionen auf welchem Weg erreicht werden können. Ein essentieller
Bestandteil um die Interaktion zwischen den Nutzenden und der Systeme
optimieren zu können, ist zu verstehen, wie die Systeme genutzt werden.
Relevante Fragestellung sind hier: In welchem Kontext werden die Systeme
eingesetzt? Welche Menschen arbeiten mit den Systemen? Welche Informationen
müssen die Nutzenden schnell erfassen können? Um diese Fragen sinnvoll
beantworten zu können müssen die Nutzenden während der Designphase
kontinuierlich in den Entwicklungsprozess einbezogen werden. Ihre gesamte
Situation, sowie ihre Bedürfnisse müssen im Designprozess untersucht und direkt
in das Produktdesign integriert werden.\cite{hci}

Wie M. Kurosu erwähnt, ist ein weitere wichtiger Aspekt des Human Centered
Design das Durchlaufen mehrere Iterationen. Ein Produkt ist nach einem ersten
Entwicklungszyklus selten schon perfekt. Oftmals ist es wichtig den Nutzenden
erste Ergebnisse zu präsentieren oder ihnen Prototypen zu zeigen. Das Feedback
das Nutzende hierbei geben muss dokumentiert und für weitere Iterationen der
Entwicklung aufgearbeitet werden. Wie Kurosu weiter erwähnt, steht der
zyklische Ablauf des Human Centered Design auch in der entsprechende ISO Norm
im Fokus.\cite{kurosuHCI}

\begin{figure}[h]
    \caption{Iteratives Vorgehen im Human Centered Design nach ISO 9241 \cite{iso9241}}
    \centering
    \includegraphics[width=10cm]{HCD.png}
\end{figure}

Das Schaubild der Iso Norm verdeutlicht den Ablauf eines Designprozesses in
mehreren Phasen. Nachdem geplant ist, was für ein Projekt umgesetzt werden soll
beginnt der zyklische Kreislauf des Entwicklungsprozesses. Im ersten Schritt
liegt der Fokus auf dem Analysieren der Situation, in der das zu entwickelnde
System später eingesetzt werden soll. Nur wer den Kontext eines interaktiven
Systems kennt, kann die Schnittstelle zwischen Maschine und Mensch adäquat
gestalten. Im zweiten Schritt geht es darum aus den geplanten Erwartungen an
das System und der Analyse des Nutzungskontextes konkrete Anforderungen zu
formulieren. Hierbei geht es noch nicht nicht darum, wie diese Anforderungen
technisch umgesetzt werden könnten, sondern darum, welche Anforderungen das
fertige System überhaupt erfüllen soll. Im dritten Schritt geht es nun darum
diese Anforderungen tatsächlich umzusetzen und technische Lösungen zu
implementieren. Gibt es erste Ergebnisse wird der vierte Schritt relevant: Die
Umgesetzte Lösungen müssen im tatsächlichen Nutzungskontext evaluiert werden.
Das setzt wieder eine intensiven Austausch mit den Nutzenden des Systems
voraus. Mit dem so gewonnen Feedback kann der Prozess wieder von vorne
beginnen. Fehlende Features können nachgebessert, missverständliche
Schnittstellen klarer gestaltet und Probleme im praktischen Einsatz minimiert
werden.\cite{iso9241}

G.A. Boy berichtet in \glqq{}The Handbook of Human-Machine Interaction\grqq{}
wie wichtig es ist, dem späteren Nutzenden der System erste Prototypen und
Ideen zu präsentieren. Oftmals wissen die Nutzenden gar nicht welche
technischen Möglichkeiten es gibt oder welche alternativen Bedienkonzepte in
ihrem Kontext besonders gut funktionieren könnten. Neben dem Zuhören und
Eingehen auf die Nutzenden und ihr Umfeld ist also auch das Einbringen und zur
Diskussion stellen neuer, für die Nutzenden unbekannter Lösungsansätze, von
hoher Bedeutung.\cite{HMI-HCD}

\section{Interview im Kontext}

Ein wesentlicher Bestandteil im Entwicklungsprozess nach den Methoden des Human
Centered Design ist der intensive Austausch mit den Nutzenden der Systeme.
Hierfür ist es wichtig, dass Softwareentwickler und Designer von
Benutzerschnittstellen in den direkten Austausch mit den Nutzenden der Systeme
gehen. Um diesen Prozess strukturiert und zielführend zu durchlaufen, kann
hierfür die Methode des \textbf{Interviews im Kontext} benutzt werden.

Beim Interview im Kontext geht es darum, die Nutzenden der Systeme in dem
Umfeld zu beobachten, in dem die Software tatsächlich auch im praktischen
Alltag eingesetzt wird. Der Fokus liegt also auf dem Umfeld der Interaktion mit
technischen Systemen. In herkömmliche Interviews begegnen sich Interviewer und
Interviewter oftmals in einer neutralen Umgebung, wie beispielsweise einem
Konferenzraum und sprechen über die Prozesse für die sich der Interviewende
interessiert. Beim Interview im Kontext geht es nicht darum \textbf{über} einen
Prozess zu sprechen, sondern den Prozess als Interviewender \textbf{direkt
    mitzuerleben}.\cite{contextualDesign} In \textit{Manual on Human-Computer
    Interaction} legen die Autoren den Fokus auf die genaue Beobachtung der
NuTzenden in der Interaktion mit den Systemen. Mit dieser Methode könne man
noch viel mehr praxisnahe Details erfassen, als wenn man mit den Nutzenden nur
über das System spricht. Spricht man mit Nutzenden beispielsweise darüber wie
sie eine Software bedienen, gibt es möglicherweise viele Dinge, die sie nicht
erwähnen, weil sie vergessen wurden, nicht relevant erscheinen oder Nutzende
nicht wissen, wie sie genau darüber sprechen sollen.\cite{hciHandbook}

Dies kann in der Praxis bedeuten, dass man sich mit den Nutzenden der Systeme
in ihren Büroräumen trifft und für mehrere Stunden mit dabei ist, wenn sie
ihrer Arbeit nachgehen und mit den technischen Systemen interagieren. Die Rolle
des Interviewenden ist dabei aus einer beobachtenden Perspektive zu verstehen.
Der Interviewte soll die Richtung bestimmen, in die sich das Interview
entwickelt. Der Interviewende hört hauptsächlich zu und beobachtet wie die
Personen in ihrer Umgebung, ihrem Kontext agieren.\cite{hciHandbook} Dazu kann
der Interviewende Notizen machen um eine spätere Auswertung des Interviews zu
erleichtern. An passenden Stellen kann der Interviewende gezielte Nachfragen
stellen um beispielsweise Hürden bei der Interaktion mit den Systemen zu
provozieren. Eine weitere Intention für Nachfragen kann aber auch sein, die
Gedanken und Emotionen des Nutzenden zu erfragen, welche er beim Benutzen des
Systems empfindet.

\begin{figure}[h]
    \caption{Während eines Interviews im Kontext können viele Aspekte beobachtet werden die über das benutzte System selbst hinausgehen\cite{johannesGrafik}}
    \centering
    \includegraphics[width=0.9\textwidth]{Interview im Kontext Schaubild.pdf}
\end{figure}

Das Ergebnis eines Interviews im Kontext ist also eine ausführliche Beobachtung
der Interaktion von Nutzenden mit den entsprechenden Systemen im alltäglichen
Kontext. Diese Beobachtungen können schriftlich festgehalten werden oder auch
durch Video- und Tonaufzeichnungen ergänzt werden. Im Nachgang des Interviews
müssen die Beobachtungen ausgewertet und analysiert werden. Ziel ist es dabei
aus den Beobachtungen des Interviews konkrete Nutzungsanforderung an das zu
entwickelnde System zu formulieren. Hier können beispielsweise Featurelisten,
Diagramme oder User-Scenarios zum Einsatz kommen.\cite{HMI-HCD}