\section{Reflektion und Fazit}
\paragraph{Einleitung Kapitel Fazit}

Nachdem im vorherigen Kapitel die Durchführung des praktischen Designprozesses
ausführlich beschrieben wurde, soll dieses Kapitel die Ausarbeitung
abschließen, indem es die Ergebnisse zusammenfasst und hinterfragt. Zunächst
wird im Abschnitt \ref{subsection:resultDescription} das Ergebnis des
Gestaltungs- und Entwicklungsprozesses resümiert. Hier wird nochmals darauf
eingegangen, wie der Nutzungskontext erfasst wurde, die Nutzungsanforderungen
spezifiziert, die entsprechenden Gestaltungslösungen umgesetzt und evaluiert
wurden. Somit wird der in Kapitel \ref{chapterPractice} ausführlich
dokumentierte Entwicklungsprozess nochmals anhand der vier Phasen des Human
Centered Design aufgegriffen und eingeordnet\cite{iso9241}. Zusätzlich wird
kurz präsentiert, welche Änderungen an der Software nach der Auswertung des
Nutzertests noch vorgenommen wurden. Als nächstes werden die eingesetzten
Methoden im Abschnitt \ref{subsection:reflection} aufgegriffen und kritisch
hinterfragt. Der Fokus liegt dabei auf der Methode des Interview im Kontext und
dem Designzyklus des Human Centered Design nach ISO9241. Schließlich wird in
Abschnitt \ref{subsection:conclusion} an die Fragestellung aus der Einleitung
angeknüpft und mit den gewonnenen Resultaten in Verbindung gebracht. Als
Letztes wird in Abschnitt \ref{subsection:outlook} noch ein Ausblick gegeben,
wie die entwickelte Software nun in der Praxis eingesetzt werden wird und
welche Schritte dazu noch notwendig sind. Außerdem wird auch skizziert, welche
weiteren wissenschaftlichen Methoden und Prozesse man noch anwenden könnte, um
die hier gewonnen Ergebnisse zu vertiefen und weiter zu verwenden.

\subsection{Beschreibung der Ergebnisse}
\label{subsection:resultDescription}

\paragraph{Was war geplant, was wurde umgesetzt?}
Ziel dieses Abschnittes soll es sein, den Bogen zu spannen zwischen den
anfänglich erarbeiteten Nutzungsanforderungen und den am Ende tatsächlich
umgesetzten Ergebnissen.

Wenn man sich die Auswertung des Interviews im Kontext in Abschnitt
\ref{subsection:IIK} ansieht, wird klar, dass eines der wichtigsten Bedürfnisse
der Nutzenden stets das schnelle und intuitive Nutzen der Software ist. Es
sollten möglichst viele Informationen auf einen Blick sichtbar sein, ohne die
Nutzenden von den wesentlichen Dingen abzulenken. Wenn Workflows durch die
Nutzenden abzuarbeiten sind, sollten diese möglichst schnell und mit möglichst
wenigen Klicks zu erledigen sein. Ein großer Detailgrad der Auswahl- und
Konfigurationsmöglichkeiten sind dagegen in diesem Kontext nicht relevant. Wo
die Software eine halbwegs sinnvolle Vorauswahl treffen kann, sollte dies
automatisch passieren und den Nutzenden somit unnötige Eintragungen ersparen.

Um hierfür konkrete Anhaltspunkte zu geben, werden im Folgenden die drei
exemplarisch vorgestellten Nutzungsanforderungen aus Abschnitt
\ref{subsection:SpannendeErkenntnisse} nochmals aufgegriffen.

Gefordert wurde da eine kompakte Kalenderansicht, die durch farbliche
Markierungen auf den ersten Blick darstellt, an welchen Tagen noch freie
Termine verfügbar sind. Diese Anforderung konnte durch die tabellarische
Monatsansicht mithilfe der Javascript Bibliothek \textit{fullcalendar}
umgesetzt werden. Durch die roten und grünen Punkte neben jedem angezeigten
Termin, kann direkt auf den ersten Blick erkannt werden, ob es sich um einen
bereits vergebenen oder freien Termin handelt. Durch die verwendete Bibliothek
zum Darstellen der Monatsübersicht ist dieser tabellarische Überblick nicht
mehr ganz so kompakt wie in der alten Softwareversion. Dies ist auch \ipName
beim Nutzertest negativ aufgefallen. Ein positiver Ausgleich hierfür ist
allerdings, das in der neuen Übersicht direkt die einzelnen Termine eines Tages
dargestellt werden. In der alten Version war es nötig zunächst den Tag
anzuklicken und erst in dem danach erscheinenden Modal konnten die einzelnen
Termine und die entsprechenden Uhrzeiten eigensehen werden. Der etwas
ausführlichere neue View nimmt also mehr Platz auf dem Bildschirm ein und kann
bei sehr vielen Terminen an einzelnen Tagen sehr lang werden, bietet dafür aber
eine umfassende Übersicht über freie und vergebene Termine uns stellt
weiterführenden Informationen mit nur einem Klick in der Detailansicht zur
Verfügung.

\begin{figure}[H]
    \caption{Vergleich der Monatsübersicht in der alten und neuen Softwareversion}
    \centering
    \includegraphics[width=\textwidth]{screen_past_now_month_view.pdf}
\end{figure}

Als nächstes wurde beim Interview im Kontext deutlich, dass eine Funktion
notwendig ist, um im Nachhinein den Termin eines Ratsuchenden schnell und
unkompliziert finden zu können. Hierfür wurde die Suchfunktion neu
implementiert. Bereist nach dem Eingeben einiger Buchstaben aus dem Namen der
ratsuchenden Person schlägt diese neue Suchfunktion passende Ergebnisse vor und
bietet in einer tabellarischen Übersicht direkt die wichtigsten Daten zum
zugehörigen Beratungstermin. Durch einen Klick auf ein Suchergebnis kann der
Termin schnell und einfach in der Detailansicht geöffnet werden. Im Vergleich
zur alten Softwareversion sind die Suchergebnisse sehr viel ausführlicher und
strukturierter dargestellt. Dies wurde auch im Nutzertest positiv wahrgenommen.
In weiteren Tests erscheint der Eindruck, dass die alte Suchfunktion etwas
schneller gearbeitet hat. Dies kann aber erst objektiv bewertet werden, wenn
die neue Implementierung auch auf den gleichen Produktiv-Servern installiert
und im Einsatz ist.

\begin{figure}[H]
    \caption{Vergleich der Suchfunktion in der alten und neuen Softwareversion}
    \centering
    \includegraphics[width=\textwidth]{screen_past_now_search.pdf}
\end{figure}

Als letztes wurde im Anschnitt \ref{subsection:SpannendeErkenntnisse} das
elegante Darstellen der Telefonnummern gefordert. Dieses Feature ließ sich in
der Praxis leicht implementieren, indem in die, aus der Datenbank abgerufene,
Telefonnummer an jeder vierten Stelle ein Leerzeichen eingefügt wird. Somit
kann die Telefonnummer von den Studienberatenden bei Bedarf in leicht zu
merkenden Vierer-Blöcken in das Telefon eingeben werden. Für andere
Nutzeraccounts als den zuständigen Beratenden werden persönliche Details der
Ratsuchenden nicht angezeigt. Diese Funktion gab es auch in der alten Version
bereits, allerdings wird in der neuen Version statt den eigentlichen Daten ein
Hinweis angezeigt. In der alten Version wurde an dieser Stelle gar nichts
angezeigt, was manchmal zu Unsicherheiten geführt hat, ob hier überhaupt Daten
eingetragen und korrekt gespeichert wurden. Das Feedback aus dem Nutzertest
zeigt: Die neue Variante braucht etwas mehr Platz auf dem Bildschirm, ist dafür
aber klarer zu verstehen und intuitiver zu erfassen.

\begin{figure}[H]
    \caption{Vergleich der Detailansicht eines vergebenen Termins in der alten und neuen Softwareversion}
    \centering
    \includegraphics[width=\textwidth]{screen_past_now_client_details.pdf}
\end{figure}

\paragraph{Umsetzung von Feedback des Usertests}
\label{paragraph:weitereIteration}

In Kapitel \ref{chapterPractice} wurde das Durchlaufen des Designzyklus anhand
einer Iteration ausführlich beschrieben. Ein essentielles Grundkonzept des
Human Centered Design beinhaltet allerdings auch das Durchlaufen weiterer
Iterationen. Mit einem durchgeführten Nutzertest sollte der Entwicklungsprozess
noch nicht als abgeschlossen gelten. Die herausgearbeiteten Kritikpunkte und
Verbesserungsvorschläge, die sich aus dem Nutzertest ergeben haben, sollten
auch in den Gestaltungsprozess der Software mit einfließen. Dies ist im
Schaubild nach ISO9241 durch die gestrichelten Pfeile erkennbar. Nach der
Evaluierung sollte der Prozess erneut durchlaufen werden. Der Einstieg erfolgt
dabei je nach Feedback der Nutzenden in Phase 1 bis 3\cite{iso9241}. Diese
weiteren Iterationen ausführlich zu beschreiben würde den Rahmen dieser Arbeit
überstrapazieren. Daher werden im Folgenden lediglich die Ergebnisse eines
zweiten Durchlaufes nach dem Nutzertest exemplarisch präsentiert:

Ein neuer Button \textit{Speichern und Nächster} wird beim Erstellen eines
neuen Termins angezeigt. Somit kann mit einem Klick der aktuelle Termin
gespeichert werden und direkt ein neuer Datensatz eingetragen werden. Die
Besonderheit ist, dass die Formularfelder in diesem Fall nicht auf
Standardwerte zurückgesetzt werden. Somit können viele Termine mit ähnlichen
Attributen besonders schnell hintereinander Eingetragen werden.

\begin{figure}[H]
    \caption{Neuer Button: Mit \textit{Speichern und Nächster} kann direkt der nächste Termin eingetragen werden.}
    \centering
    \includegraphics[width=0.9\textwidth]{screen_feedback_save_next.png}
\end{figure}

\ipName hat während des Nutzertests angemerkt, dass nicht nur das Beratungsanliegen der ratsuchenden Personen besonderem Datenschutz unterliegt. Auch persönliche Kontaktdaten wie Mailadresse und Telefonnummer sollten nur dem zuständigen Beratenden angezeigt werden. In der nochmals überarbeiteten Version werden somit nun auch die Telefonnummer und die Mailadresse des Kunden zensiert.

\begin{figure}[H]
    \caption{Details eines vergebenen Termins aus Ansicht einer Hilfskraft.}
    \centering
    \includegraphics[width=0.9\textwidth]{screen_feedback_client_censorship.png}
\end{figure}

\subsection{Reflektion der eingesetzen Methoden}
\label{subsection:reflection}

Um dem wissenschaftlichen Anspruch dieser Arbeit Genüge zu tragen, sollen die
vorgestellten Methoden und Theorien nicht nur in der Praxis erprobt, sondern
auch kritisch hinterfragt werden. Im folgenden Abschnitt werden die Methoden
des Human Centered Design und des Interview im Kontext nochmals aufgegriffen
und deren Praxistauglichkeit reflektiert. Des Weiteren wird auch die Umsetzung
und Implementierung der Gestaltungslösungen in Frage gestellt und unter
Einbeziehung des Feedbacks aus dem Nutzertest analysiert.

\paragraph{Human Centered Design und IIK}
Die Theorie des Human Centered Design wurde zu Beginn dieser Arbeit motiviert
mit der technischen Entwicklung von interaktiven Systemen. Software ist im
Laufe der letzten Jahrzehnte immer interaktiver geworden. Der Schnittstelle
zwischen Mensch und Maschine kommt dabei eine immer größer werdenden Bedeutung
zu. Der Ansatz des Human Centered Design versucht an dieser Stelle alle
Beteiligten einzubeziehen und eine einfache, intuitive Bedienung der Systeme zu
ermöglichen, die ganz bewusst aus Perspektive der Nutzenden gestaltet wurde.
Diese Theorie wurde in diesem Fall anhand des Designzyklus nach ISO9241 in die
Praxis umgesetzt. Für die Durchführung der ersten Phase \textit{Den
    Nutzungskontext verstehen und Beschreiben} wurde hier die Methode des
Interviews im Kontext verwendet. Hierbei ging es darum alle Menschen, die mit
dem System zu tun haben einzubeziehen und Ihre Arbeitsabläufe zu verstehen. Wie
der Name der Methode bereits impliziert liegt der Fokus hierbei auch auf dem
Kontext in dem diese Arbeitsabläufe stattfinden. Das Interview im Kontext wurde
deshalb mit \ipName in seinem Büro an seinem Dienstrechner durchgeführt um eine
möglichst realistische Alltagssituation zu schaffen.

Der Einsatz dieser Methoden lässt sich im Großen und Ganzen als sehr
ertragreich beschreiben. Durch den engen Austausch mit \ipName konnte der
Nutzungskontext sehr gut verstanden und aufgearbeitet werden. Das Interview im
Kontext hat überraschend viele neue Ideen hervorgebracht, die wohl keiner der
Beteiligten im Vorhinein hätte formulieren können. Durch das gemeinsame
Benutzen der alten Softwareversion vor Ort wurde ganz klar deutlich, welche
Features wirklich wichtig sind für den tägliche Einsatz, an welchen Stellen es
schnell gehen muss und wo detaillierte Einstellungsmöglichkeiten unbedingt
notwendig sind. Die Erkenntnisse und Beobachtungen aus dem Interview im Kontext
wurden auf dem I-Pad unstrukturiert schriftlich festgehalten. Beim Auswerten
dieser Notizen ist aufgefallen, das viele Details aus dem Gespräch nicht so
schnell aufgeschrieben werden konnten. Dies Notizen waren somit nicht ganz
vollständig. Daher war es sehr wichtig die Auswertung des Interviews im Kontext
zeitlich sehr nah an der Durchführung zu planen. Somit konnten viele Details
aus der Erinnerung an das Interview selbst bei der Auswertung ergänzt werden.
Der Austausch mit Menschen aus allgemeinen Studienberatung hat sich auf den
Kontakt mit \ipName beschränkt. Dieser Austausch war sehr ertragreich, dennoch
wäre es interessant gewesen auch noch Nutzungsanforderungen anderer
Nutzergruppen wie beispielsweise der Hilfskräfte der Erstinformation mit
einzubeziehen. Die Erfahrungen und Ideen andere Nutzenden hätten noch weitere
hilfreiche Aspekte in den Gestaltungsprozess einbringen können. Somit hätte ein
System entstehen können, das noch inklusiver und reibungsloser alle beteiligten
Personen mit einbezieht. In der Praxis war es allerdings schon sehr aufwendig
Termine mit \ipName zu organisieren. Wären noch weitere Nutzungsgruppen mit
einbezogen worden, wäre der organisatorische und zeitliche Aufwand der Software
deutlich höher gewesen. Dies kann grade in gewinnorientierten Softwarehäusern
einen hemmenden Faktor für den Einsatz von Human Centered Design mit sich
bringen.

Um noch einmal Bezug auf die vier Phasen des Gestaltungsprozesses nach ISO9241
zu nehmen, kann man festhalten, dass die klare Zuordnung der Arbeitsschritte in
eine dieser Phasen sehr gut funktioniert hat. Somit wurde jede Arbeitsphase
klar strukturiert und war einem definierten Ergebnis zugeordnet. Ein
zielführendes und effizientes Arbeiten wurde somit erleichtert.

\paragraph{Implementierung und Usertests}

Im Folgenden soll nun auch die Implementierung der Software kritisch
hinterfragt werden. Die Ergebnisse der Nutzertests sollen hierbei mit
einbezogen werden und das Meinungsbild mit ersten Testerfahrungen bestärken.

% Einfache Umsetzung vs Nutzerfreundliche Lösung
Zunächst fällt allgemein auf, dass besonders nutzungsfreundliche Lösungen
oftmals technisch verhältnismäßig aufwendig umzusetzen sind. Wird Software von
Seite der technischen Möglichkeiten und Datenstrukturen her gestaltet, drängen
sich bestimme Herangehensweisen oftmals fast unumgänglich auf. Diese
Herangehensweisen sind dann meist sehr technisch geprägt und für unerfahrene
Nutzende manchmal schwierig zu verstehen. Wenn man diesen Prozess im Sinne des
Human Centered Design bewusst umdreht und die Gestaltung bei den Nutzenden und
ihrem Umfeld beginnt, können oftmals deutlich passendere und einfachere
Workflows und Views geschaffen werden. Diese sind in der Implementierung
allerdings oftmals aufwendig zu realisieren.

% Problem: Designpattern bei fertigem Softwareökosystem sehr eingeschränkt
% Objektorientierter Ansatz sinnvoll für start serverlastige Anwendung?
Als nächstes soll nochmal in Erinnerung gerufen werden, dass es sich bei dieser
Umsetzung um ein Modul in einer bereits bestehenden Software handelt. Es
mussten also alle bestehenden Schnittstellen und Programmierkonzepte der
bereits existierenden Module eingehalten werden. Beispielsweise wurde für die
Termindatensätze, sowie für Beratungsräume und Mailtemplates ein
objektorientierter Ansatz gewählt. Ziel war es damit die Beziehungen dieser
Datentypen untereinander klarer modellieren und elegant in Code umsetzen zu
können. Im Rahmen der gesamten Software Stubegru hat sich allerdings gezeigt,
das es eher wenig Sinn macht lokale Instanzen von Objekten zu erstellen und zu
referenzieren. Die restlichen Module funktionieren so, dass Daten meist nur vom
Server abgerufen und angezeigt werden. Weitere Referenzen oder Methodenaufrufe
auf diesen Daten sind nicht vorgesehen und machen in diesem Anwendungsfall auch
wenig Sinn. Die erhofften Vorteile der objektorientierten Umsetzung in diesem
Modul konnten somit nicht vollständig erreicht werden. Viele Methoden wurden
letztendlich als \textit{static} implementiert und werden somit unabhängig von
der Instanz einer Klasse aufgerufen.

% Prototypen wichtig für Feedback, Vorstellung
% Evtl frühere Tests
Im Nutzertest hat sich herausgestellt, dass es extrem hilfreich ist, wenn man
eine Gestaltungslösung vorzeigen und gemeinsam durchspielen kann. Beim
gemeinsamen Nutzen der implementierten Funktionen fallen oft wichtige Details
auf, die im Vorhinein bei der Planung noch nicht berücksichtigt wurden. Somit
war es also extrem hilfreich für den Nutzertest bereits das fast vollständig
fertig implementierte Modul zur Terminvereinbarung präsentieren zu können und
somit eine greifbare Gesprächsgrundlage zu bieten. Vielleicht wäre es hilfreich
gewesen, schon früher im Gestaltungsprozess erste Prototypen zu entwickeln, die
noch nicht voll funktionstüchtig sind, allerdings schon die wichtigsten
Nutzungsinteraktionen wiedergeben können. Dann hätte man schon früher im
Designprozess ein anschauliches Beispiel für konkrete Gestaltungslösungen
bieten können. Somit hätten etwaige Mängel schon schneller erkannt und direkt
korrigiert werden können. Da das Ergebnis des Nutzertests in diesem Fall
allerdings sehr positiv ausgefallen ist, gab es sowieso nicht mehr viel
Änderungsbedarf. Wären beim Nutzertest noch größere Unstimmigkeiten aufgefallen
hätte man vermutlich wesentliche Teile des Codes neu schreiben müssen. In solch
einem Fall hätte man mit früheren Prototypen viel doppelte Arbeit ersparen
können. Allerdings lässt sich abschließend auch festhalten, dass durch das
ausführliche Interview im Kontext bereits ein sehr guter Überblick entstanden
ist, wie die Software aussehen sollte, um die Nutzenden möglichst gut zu
unterstützen. Dies zeigt auf, dass eine ausführliche und gewissenhafte
Durchführung der ersten Phasen im Gestaltungsprozess, die Wahrscheinlichkeit
für Missverständnisse oder größere Änderungswünschen im Nutzertest senkt.

\subsection{Zusammenfassender Abschluss}
\label{subsection:conclusion}
% Eingehen auf Fragestellung der Einleitung
% An welchen Stellen können die theoretischen Grundlagen des Human Centered Design den Entwicklungsprozess in der Praxis tatsächlich sinnvoll unterstützen? 
% Gab es eventuell auch Methoden, die in der praktischen Umsetzung problematisch waren oder noch optimiert werden könnten?
% Erfüllt die Software die gewünschten Anforderungen?
% Abrundendes Schlusswort

\subsection{Ausblick}
\label{subsection:outlook}

Als abschließender Abschnitt dieser Ausarbeitung soll die weitere Perspektive
des Forschungsthemas betrachtet werden. Hierbei ist es interessant den weiteren
Einsatz des praktisch entwickelten Moduls zur Terminvereinbarung zu skizzieren.
Wie wird sich das Modul in das gesamte Softwarepaket Stubegru eingliedern? In
welchem Kontext kann es tatsächlich produktiven Einsatz finden und welche
Schritte sind dafür noch zu bedenken? Außerdem sollen die verwendeten Methoden
und Theorien nochmals aufgegriffen werden. An welchen Stellen lassen sich die
verwendeten Konzepte eventuell noch verbessern? In welcher Richtung verspricht
eine weitere wissenschaftliche Forschung spannende neue Erkenntnisse?

\paragraph{Werdegang der Software Stubegru}
Wie bereits in Abschnitt \ref{paragraph:weitereIteration} erwähnt sind noch
weitere Iterationen und Feedbackrunden mit den Nutzenden nötig um das neu
entwickelte Modul zur Terminvereinbarung tatsächlich sinnvoll im Arbeitsalltag
zu verwenden. Es ist konkret geplant diese weiteren Entwicklungszyklen zu
durchlaufen und eine neue Version der Software Stubegru in der Abteilung
Studium und Lehre produktiv einzusetzen. Ist das Modul zur Terminvereinbarung
fertig, werden noch andere Module und Workflows der Software überarbeitet
werden müssen. Wenn all diese Arbeiten abgeschlossen sind, wird es eine
intensive Einführungsphase geben. In den ersten Wochen im produktiven Betrieb
ist zu erwarten, dass noch einige Bugs entdeckt und behoben werden müssen.
Außerdem müssen die Mitarbeitenden der Abteilung auf die neuen Abläufe in der
überarbeiteten Softwareversion geschult werden. All diese Schritte sollen im
Rahmen eines Dienstleistungsvertrages mit der Universität Kassel weiterhin
betreut werden.

\paragraph{Forschungsausblick}
Die Durchführung und Auswertung dieser Arbeit hat gezeigt, dass die verwendeten
Theorien des Human Centered Design schon sehr ausgereift und vor allem sehr
praxisnah formuliert sind. Die Umsetzung der methodischen Grundkonzepte ließ
sich somit sehr gut in die Praxis übertragen. Spannend wäre es allerdings sich
noch tiefgreifender mit den verschiedenen Nutzungsgruppen einer Software
auseinander zu setzen. In dieser Arbeit bedeutete der Austausch mit den
Nutzenden größtenteils der Kontakt mit einem Studienberater der allgemeinen
Studienberatung. Dieser repräsentierte eine Nutzungsgruppe, die bereits
jahrelange Erfahrung mit den Prozessen und Abläufen in der Abteilung vorweisen
kann. Spannend wäre es, noch weitere Nutzungsgruppen in den Fokus zu rücken.
Hilfskräfte beispielsweise sind oftmals nur für einen kurzen Zeitraum mit
geringen Wochenstunden angestellt. Eine jahrelange Einarbeitung in die Prozesse
des Arbeitsalltags kann hier nicht vorausgesetzt werden. Wie könnte man also
diese Nutzungsgruppe besser in der Bedienung der verwendeten Software
unterstützen. Könnten zum Beispiel kleine interaktive Hilfetexte helfen, die
Arbeitsprozesse spielerisch bei der Verwendung der Software zu erlernen?
Hierfür wäre sicherlich ein intensiverer und direkter Austausch mit
Hilfskräften der Abteilung notwendig gewesen.

Des Weiteren wäre es relevant die Unterschiede in der direkten Interaktion mit
dem Computer der einzelnen Nutzenden genauer zu untersuchen. In den hier
verwendeten Grundlagen zum Human Centered Design lag der Fokus immer darauf
eine Gestaltungslösung zu finden die möglichst nutzungsfreundliche
Schnittstellen bietet.\cite{hcd} Oftmals bevorzugen verschiedene Nutzende
allerdings verschiedene Schnittstellen. Ich persönlich, als technisch sehr
erfahrener Nutzer, bevorzuge beispielsweise Eingaben über die Tastatur und
arbeite gerne mit Shortcuts. In den Nutzertests hat sich schnell gezeigt, dass
der Testkandidat am liebsten alle Formularfelder mit der Maus auswählt. Somit
ist für ihn ein grafischer Timepicker beispielsweise sehr wichtig für eine gute
User Experience. Für mich wäre es allerdings viel wichtiger die Möglichkeit zu
haben, jedes Eingabefeld auch über Tastaturbefehle zu erreichen. Dieses kleine
Beispiel soll illustrieren, dass selbst einzelne Menschen einer Nutzungsgruppe
oftmals ganz verschiedene Vorstellungen von einer \textit{angenehmen} und
\textit{schönen} Nutzungsoberfläche haben. Diese Differenzen weiter zu
untersuchen und allen Nutzenden eine adäquate Schnittstelle zu bieten wäre also
eine weitere spannende Richtung um die Ergebnisse dieser Arbeit noch weiter zu
vertiefen.