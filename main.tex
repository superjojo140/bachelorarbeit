%Standard imports für Artikel
\documentclass[12pt]{article}
\usepackage[utf8]{inputenc}
\usepackage[ngerman]{babel}
\usepackage[T1]{fontenc}
\usepackage{cite}

%Bilder importieren
\usepackage{graphicx}
\graphicspath{ {./img/} }

\usepackage{hyperref}
\hypersetup{
    colorlinks,
    citecolor=black,
    filecolor=black,
    linkcolor=black,
    urlcolor=black
}

\setlength{\parskip}{1em}

\title{Human Centered Design - Software zur Terminvereinbarung}
\author{Johannes Schnirring}

\begin{document}

\begin{titlepage}
    \centering
    {\scshape\LARGE Human Centered Design - Software zur Terminvereinbarung \par}
    \vspace{1cm}

    {\scshape\Large Bachelorarbeit\par}
    \vspace{1.5cm}

    \begin{tabular}{r l}
        {\Large Studierender:} & {\Large Johannes Schnirring}             \\ \\
        {\Large Betreuung:}    & {\Large Prof. Dr. Claude Draude        } \\ \\
        {\Large Semester:}     & {\Large Wintersemester 2022}             \\ \\
        {\Large Datum:}        & {\Large \today}                          \\ \\
    \end{tabular}
    \vfill
    {\large Universität Kassel}

    % Bottom of the page

\end{titlepage}

%\maketitle
\tableofcontents
\newpage

\section{Einleitung}

\subsection{Motivation}

% Hier wird die Problemstellung im Kontext
% einer Anwendung dargestellt und der Inhalt der Arbeit kurz vorweggenommen. Welches Problem wurde
% gelöst, warum ist das relevant? Wie ist die Vorgehensweise? Was wurde thematisch eingegrenzt, was
% wurde ausgegrenzt? Wichtig ist es, den eigenen Beitrag in wenigen prägnanten Sätzen herauszuarbeiten. (Das Fazit soll sich zum Schluss auf diese Beiträge beziehen, um der Arbeit eine erzählerische
% Klammer zu geben.) Die Einleitung endet mit einem Überblick über die Arbeit – hierbei werden die
% Inhalte der einzelnen Kapitel knapp umschrieben

Der Ansatz des Human Centered Design bietet einen nutzungsfokussierten
Blickwinkel auf den Entwurf und den Designprozess von Software. In dieser
Ausarbeitung werden die grundlegenden Methoden des Human Centered Design
vorgestellt, praktisch angewendet und abschließend reflektiert.

Die allgemeine Studienberatung und Information Studium der Universität Kassel
nutzt zur Erleichterung und Dokumentation der täglich anfallenden Aufgaben eine
speziell für diesen Bereich entwickelte, webbasierte Software. Diese Software
setzt sich aus verschiedenen Modulen zusammen, die dazu beitragen Ordnung und
Kommunikation im Team zu erleichtern. Um allen Mitarbeitenden einen Überblick
zu geben, welche Kolleg:innen aufgrund von Urlaub, Krankheit oder Dienstreisen
am aktuellen Tag abwesend sind, gibt es in dieser Software ein Modul um
Abwesenheiten einzupflegen und somit alle Teammitglieder auf dem neuesten Stand
zu halten.\\ Um die Nutzung des Abwesenheitsmoduls für alle Mitarbeitenden der
Abteilung intuitiver und einfacher zu gestalten soll eine Überarbeitung des
Moduls mit Methoden des Human Centered Design durchgeführt werden. Ziel ist die
Implementierung eines übersichtlichen und intuitiven Managements von
Abwesenheiten des Teams in der bestehenden Callcenter Software.

Das Ziel dieser Seminararbeit ist es den Designprozess strukturiert zu
begleiten und zu dokumentieren. Am Beispiel des Abwesenheitsmoduls sollen
Verfahren zur Entwicklung und Einführung intuitiv zu bedienender Software in
enger Zusammenarbeit mit den Anwender:innen diskutiert und praktisch erprobt
werden. Hierfür wird zunächst der Begriff des Human Centered Design näher
erläutert. Mit dem \textit{Interview im Kontext} wird eine grundlegende Methode
dieses Designansatzes vorgestellt. Der Ablauf des Interviews in der Praxis wird
im Hauptteil beschrieben. Die darauf aufbauenden Prozesse der Erarbeitung von
Optimierungen der Software werden weitergehend dokumentiert. Schließlich werden
die ersten Entwürfe der neu umgesetzten Veränderungen präsentiert und kritisch
reflektiert. Abschließend wird der Erfolg der verwendeten Methoden beurteilt
und ein Ausblick auf die weiteren Schritte der Implementierung gegeben.

\section{Methoden}

\subsection{Human Centered Design}
Human-Centered Design ist eine Methode zur Entwicklung interaktiver Systeme,
wie beispielsweise Software. Der wichtigste Aspekt der Methode ist es, diese
Systeme benutzerfreundlich und möglichst nützlich zu gestalten. Wie Alan Dix
klarstellt hat sich die Interaktion zwischen Menschen und Computern in den
letzten Jahrzehnten stark verändert. Während Computer anfangs die meiste Zeit
einfach vor sich hin gerechnet haben, sind Softwaresysteme heutzutage höchst
interaktiv uns sollen ohne Hürden von allen Teilen der Gesellschaft genutzt
werden können. \cite{hci}[S. 234] Beim Human Centered Design werden die
Nutzenden der Systeme in den Mittelpunkt gestellt. Martin Ludwig Hofmann
betont: \glqq [Es geht nicht darum] vom Gerät her zu denken, sondern vom
Menschen und der Art und Weise, wie er die Welt wahrnimmt\grqq{}. \cite{hcd}[S.
    134] Die Bedürfnisse, Prozessabläufe und Erwartungen der Nutzenden sind beim
Human Centered Design der wichtigste Aspekt beim Entwurf von Schnittstellen
zwischen Systemen und den Menschen, die sie benutzen.

Die Nutzenden sollen während der gesamten Design- und Entwicklungsphase
kontinuierlich in den Prozess der Produktentwicklung eingebunden werden. Statt
sie eine fertige Idee oder einen fertig entwickelten Prototypen bewerten zu
lassen, sollen ihre Bedürfnisse erforscht und direkt in die Ausgestaltung des
Produkts integriert werden. In \textit{Human-Computer Interaction} wird betont
wie wichtig es ist den Fokus im gesamten Prozess auf die Nutzenden zu legen.
Bei allen Aspekten des Softwaredesigns ist es wichtig, das direkte Feedback der
Nutzenden einzuholen und in den kompletten Entwicklungs- und Lebenszyklus einer
Software einfließen zu lassen \cite{hci}[S. 226]

Der Designprozess beinhaltet den intensiven Austausch mit den Nutzenden. In
\textit{The human-computer interaction handbook} weisen die Autoren auf die
Relevanz von Beobachtung und aufmerksamer Wahrnehmung hin. Wenn man Nutzende
fragt, wie sie eine Software benutzen, werden sie viele Dinge nicht erwähnen,
weil sie vergessen werden, nicht relevant erscheinen oder die Nutzenden nicht
genau wissen, wie sie darüber sprechen können. \cite{hciHandbook}[S. 970]
Resultierend ist es also wichtig die Bedürfnisse, Fähigkeiten und Strategien
der Nutzenden direkt im Nutzungsalltag zu beobachten und zu analysieren.
Entwickler und Designer sollen aber auch das Umfeld, die Arbeitsabläufe und den
Kontext des zu entwerfenden Systems genau verstehen. So soll ein System
entwickelt werden, dass den Nutzenden in ihrer Situation genau den Mehrwert
bieten kann, der für sie wichtig ist. Das System soll sich an die Abläufe und
Prozesse der Nutzenden anpassen und nicht umgekehrt.

\begin{figure}[h]
    \caption{Iteratives Vorgehen im Human Centered Design nach ISO 9241 \cite{iso9241}}
    \centering
    \includegraphics[width=10cm]{HCD.png}
\end{figure}

Alan Dix stellt des weiteren ganz klar heraus, dass dieser Prozess in mehreren
Iterationen ablaufen muss. Nach dem ausgiebigen Beobachten und Diskutieren der
Anforderungen gemeinsam mit den Nutzenden, können erste Prototypen und
Beta-Versionen entwickelt werden. Diese müssen nun unbedingt erneut mit den
Nutzenden ausprobiert und diskutiert werden. Dieser Prozess des Ausprobierens,
Beobachtens, Analysierens und Entwickelns neuer Lösungsansätze sowie deren
praktische Umsetzung muss oftmals in vielen Iterationen wiederholt und mit
jedem Mal weiter optimiert werden.\cite{hci}[S. 234-237]

\subsection{Interview im Kontext}
Eines der wichtigsten Kernkonzepte im Human Centered Design ist das Verstehen
der Nutzenden und ihrer Umfelder. Hierfür gibt es verschiedene Methoden, die
Entwicklern und Designern diesen Prozess erleichtern. An dieser Stelle wird die
Methode des\textbf{ Interviews im Kontext} gewählt und kurz vorgestellt.

Ziel des \textit{Interviews im Kontext} ist es, die Anforderungen und
Bedürfnisse der Nutzenden im realen Nutzungskontext zu erleben und zu
dokumentieren. Wie in \textit{Contextual Design} klargestellt wird, machen
klassische Techniken der Marktforschung für eine nutzerfokussierte
Softwareentwicklung oftmals keinen Sinn. Wichtig ist es die Nutzenden im
tatsächlichen Arbeitsumfeld zu beobachten und somit den Kontext der Interaktion
mit der Software als essentiellen Teil in die Beobachtungen einfließen zu
lassen.\cite{contextualDesign}[S. 36ff] Hierbei werden die Nutzenden direkt an
Ihrem Arbeitsplatz, während ihrer Arbeit beobachtet und begleitet. Der
Interviewende begibt sich hauptsächlich in eine zurückhaltende Rolle eines
Beobachters. Wie in \textit{The human-computer interaction handbook}
vorgeschlagen, wird der Rahmen des Interviews möglichst locker gehalten. Die
Nutzenden des Systems sollen möglichst natürlich und frei zeigen, wie sie das
System benutzen.\cite{hciHandbook}[S. 972] Wichtig ist den Nutzenden
Möglichkeiten zum Erzählen zu bieten. Auch Dinge die im ersten Moment trivial
oder nicht relevant erscheinen, sollen Nutzende in Ruhe ausführen und dem
Interviewer somit die Möglichkeit geben, einen umfassenden Einblick in die
Interaktionen mit dem System zu gewähren. Der Interviewende verbringt also die
meiste Zeit damit zuzuhören und Impulsen des Interviewten zu folgen. Zusätzlich
stellt er Nachfragen zum besseren Verständnis der beobachteten Situationen,
Arbeitsabläufe und Handlungen. Der Interviewende hält das Feedback der
Nutzenden zusammen mit seinen eigenen Beobachtungen meist schriftlich fest.
Ziel dieser Methode ist es kontextabhängige Nutzungsszenarien der Systeme
mitzuerleben, zu dokumentieren und daraus Ideen für die Entwicklung bzw.
Verbesserung der Systeme zu gewinnen.

\subsection{Weitere Methode (vllt Prototypentwicklung/Feedbackmethode)}


\section{Durchführung}

Struktur abarbeiten und verschiedene inhaltliche Aspekte immer wieder aufgreifen. Oder inhaltliche Aspekte abarbeiten und Struktur aufgreifen...?

\subsection{Softwarebedarf in der Abteilung}
\subsection{Aktuelle Softwarelösung}
\subsection{Grund für Veränderung}

\subsection{Erste Planungen - Interview im Kontext}
\subsubsection{Kontext des Interviews}
\subsubsection{Spannende Erkentnisse}

\begin{itemize}
    \item Datum / Uhrzeit mit Maus statt Tastatur
    \item Suche nach Teilnehmer
    \item Kompakte Ansicht Kalender (mit Farben)
    \item Telefonnummeranzeige ("Silbentrennung")
    \item Format / Beispielwerte bei Variablen in Templates
    \item Modal reset oder nicht?
\end{itemize}

\subsubsection{Erarbeitung von Verbesserungsvorschlägen}

\subsection{Prototyping}
\subsubsection{Methode ???}
\subsubsection{Erste Prototypen}
\subsubsection{Feedback zu Prototypen}

\subsection{Implementierung}
\subsubsection{Methode ???}
\subsubsection{Beschreibung des Prozess}
\subsubsection{Technische Umsetzung}
\subsubsection{Präsentation erster Ergebnisse}

\subsection{Testen / User-feedback}
\subsubsection{Methode ???}
\subsubsection{Feedback der Nutzenden}
\subsubsection{Ausblick auf weitere Iterationen}

\section{Reflektion und Fazit}
\subsection{Beschreibung des Ergebnis}
\subsection{Beurteilung der Umsetzungsphase}
\subsection{Beurteilung der eingesetzten Methoden}
\subsection{Ausblick}

\newpage

\bibliographystyle{plain}
\bibliography{refs}

\end{document}