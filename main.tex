\documentclass[12pt]{report}
\usepackage[utf8]{inputenc}
\usepackage[ngerman]{babel} %Deutsche Formate benutzen
\usepackage[T1]{fontenc}
\usepackage{todonotes} %Todo Notizen am Rand
\usepackage{fancyhdr} %Kopfzeile mit Chaptername

%Literatur 
\usepackage[backend=bibtex,style=ieee,natbib=true]{biblatex}
\addbibresource{refs.bib}


%Bilder da anzeigen wo sie eingebunden werden, Rahmen um Bilder
\usepackage{float}
\floatstyle{plain} 
\restylefloat{figure}

%Listings
\usepackage{listings} %Codeschnipsel
\lstset{
    basicstyle=\ttfamily,
    columns=fullflexible,
    frame=single,
    breaklines=true,
    postbreak=\mbox{\textcolor{red}{$\hookrightarrow$}\space},
}

%Bilder importieren
\usepackage{graphicx}
\graphicspath{ {./img/} }

%Referenzen
\usepackage{hyperref}
\hypersetup{
    colorlinks,
    citecolor=black,
    filecolor=black,
    linkcolor=black,
    urlcolor=black
}

\usepackage[nopostdot,section=section]{glossaries} %Glossar (muss nach hyperref eingebunden werden)
\renewcommand{\glossarysection}[2][]{}
\renewcommand{\glstextformat}[1]{\textbf{\em #1}}
\makeglossaries

\newglossaryentry{Bootstrap}
{
    name=Bootstrap,
    description={Bootstrap ist ein ursprünglich von Twitter entwickeltes \gls{Css} Framework, das standardisierte \gls{Html} Elemente definiert. Somit können Softwaredesignende häufig benötigte Gestaltungselemente direkt verwenden und können gleichzeitig eine einheitliche und vertraute User-Experience bieten.\\ \\
            \textit{In dieser Arbeit verwendete Version: \textbf{Bootstrap v3.3.4}}
            \cite{Bootstrap}}
    \\\textit{--Begriff kommt vor auf Seite: }
}

\newglossaryentry{Tech-Stack}
{
    name=Tech Stack,
    description={Ein Tech Stack beschreibt die Technologien, die zur Entwicklung einer spezifischen Software zum Einsatz kommen. Dazu zählen beispielsweise die verwendeten Programmiersprachen, Bibliotheken, Frameworks oder auch Entwicklungswerkzeuge.}
    \\\textit{--Begriff kommt vor auf Seite: }
}

\newglossaryentry{Stubegru}
{
    name=Stubegru,
    description={Stubegru ist ein Softwarepaket für akademische Beratungsstellen. Als webbasierte Groupware kann Stubegru typische Workflows in einer Studienberatung begleiten: Beratungstermine, Wissensdatenbank, Abwesenheitsmanagement und vieles mehr
            \cite{stubegruWebsite}. Die Software Stubegru wurde gemeinsam von Johannes Schnirring und der Universität Kassel entwickelt. In Abschnitt \ref{section:stubegru-refs} finden sich weitere Links und Referenzen zur Software.}
    \\\textit{--Begriff kommt vor auf Seite: }
}

\newglossaryentry{Usertest}
{
    name=Usertest,
    description={Fertige Software oder Prototypen werden den Nutzenden vorgestellt. Diese sollen vorgegebene Prozesse in der Software durchlaufen. Softwareentwickelnde und Gestaltende beobachten wie die Nutzenden mit dem System interagieren. Anschließend kann das Feedback der Nutzenden eingeholt werden.}
    \\\textit{--Begriff kommt vor auf Seite: }
}

\newglossaryentry{Timepicker}
{
    name=Timepicker,
    description={Kleines grafisches Popup, um Uhrzeiten elegant mit der Maus eingeben zu können. Ähnlich wie ein Datepicker, der allerdings zur grafischen Eingabe eines Datum genutzt werden kann.
            \cite{datepicker}}
    \\\textit{--Begriff kommt vor auf Seite: }
}

\newglossaryentry{UML}
{
    name=UML,
    description={Abkürzung für Unified Modeling Language (vereinheitlichte Modellierungssprache).UML beschreibt Konventionen um Datenmodelle und Workflows von Software und technischen Systemen grafische zu modellieren.
            \cite{UML}}
    \\\textit{--Begriff kommt vor auf Seite: }
}

\newglossaryentry{View}
{
    name=View,
    description={Ein View beschreibt eine Ansicht, die Nutzende zu einem Zeitpunkt beim Verwenden einer Software auf dem Bildschirm grafisch angezeigt bekommen. Eine tabellarische Übersicht von Datensätzen oder eine Eingabemaske für einen neuen Datensatz sind Beispiele für einzelne Views.}
    \\\textit{--Begriff kommt vor auf Seite: }
}

\newglossaryentry{Modal}
{
    name=Modal,
    description={Im \gls{Css} Framework \gls{Bootstrap} bezeichnet ein Modal eine Art Popup Fenster. Dieses Fenster legt sich über den bisherigen Bildschirminhalt und zieht somit die Aufmerksamkeit der Nutzenden auf sich. Aufforderungen für Nutzereingaben, oder Detailansichten eines Datensatzes werden oftmals mit Modals umgesetzt.}
    \\\textit{--Begriff kommt vor auf Seite: }
}

\newglossaryentry{Javascript}
{
    name=Javascript,
    description={JavaScript (Abk. JS) ist eine Skriptsprache, welche für interaktive Inhalte in Webbrowsern entwickelt wurde. Mit Javascript können Inhalte von \gls{Html} Seiten angepasst oder dynamisch nachgeladen werden. Außerdem können Interaktionen vom Nutzenden (z.B. Klicken, Scrollen, Tastatureingaben) verarbeitet und ausgewertet werden.\\ \\
            \textit{In dieser Arbeit verwendete Version: \textbf{ECMAScript 2022 (ES13)}}
            \cite{Javascript}}
    \\\textit{--Begriff kommt vor auf Seite: }
}

\newglossaryentry{Html}
{
    name=Html,
    description={Abkürzung für Hypertext Markup Language. Html ist eine textbasierte Sprache um Dokumente, insbesondere Internetseiten zu Strukturieren. Über Html kann der Text auf einer Internetseite dargestellt, formatiert und positioniert werden. Außerdem können grafische Elemente wie Hyperlinks, Bilder oder Videos eingebunden werden.\\ \\
            \textit{In dieser Arbeit verwendete Version: \textbf{HTML 5}}
            \cite{Html}}
    \\\textit{--Begriff kommt vor auf Seite: }
}

\newglossaryentry{Php}
{
    name=Php,
    description={Php ist eine Programmiersprache um serverseitige Befehle auszuführen. Php Skripte werden häufig verwendet um Datenbankoperationen auf einem Webserver auszuführen und die Ergebnisse an einen Webbrowser zu schicken.\\ \\
            \textit{In dieser Arbeit verwendete Version: \textbf{PHP 8}}
            \cite{Php}}
    \\\textit{--Begriff kommt vor auf Seite: }
}

\newglossaryentry{Http}
{
    name=Http,
    description={Die Abkürzung Http steht für Hypertext Transfer Protocol. Dieses Protokoll wurde von Tim Berners-Lee entwickelt um \gls{Html} Dokumente über das World Wide Web verfügbar zu machen. Das Hypertext Transfer Protocol definiert Standards, wie eine Website vom Server zum Client übertragen wird.
            \cite{http}}
    \\\textit{--Begriff kommt vor auf Seite: }
}

\newglossaryentry{MySQL}
{
    name=MySQL,
    description={MySQL ist ein Datenbankverwaltungssystem, das von Oracle entwickelt und maintained wird. MySQL ist OpenSource und kann somit kostenfrei von Softwarentwickelnden genutzt werden um relationale Datenbanken aufzusetzen.\\ \\
            \textit{In dieser Arbeit verwendete Version: \textbf{10.9.4-MariaDB}}
            \cite{mariadb}}
    \\\textit{--Begriff kommt vor auf Seite: }
}

\newglossaryentry{OpenSource}
{
    name=OpenSource,
    description={Der Begriff OpenSource beschreibt Softwareprojekte deren Code für alle Menschen frei zugänglich ist. Somit kann die Software nicht nur kostenfrei genutzt werden, sondern auch von jedem Einzelnen verändert, weiterentwickelt und verteilt werden.}
    \\\textit{--Begriff kommt vor auf Seite: }
}

\newglossaryentry{JSON}
{
    name=JSON,
    description={JSON steht für JavaScript Object Notation und ist eine Beschreibungssprache. Mithilfe von JSON können Datensätze in textueller Form dargestellt werden. Diese Darstellung kann sowohl von Menschen, als auch von Computern einfach gelesen werden. Daher wird JSON inzwischen nicht nur im Zusammenhang mit \gls{Javascript} verwendet, sondern hat sich als Standard für den Datenaustausch im Internet durchgesetzt.
            \cite{json}}
    \\\textit{--Begriff kommt vor auf Seite: }
}

\newglossaryentry{Document-Object-Model}
{
    name=Document Object Model,
    description={Das Document Object Model (kurz DOM) definiert eine Programmierschnittstelle um \gls{Html} Dokumente als Baumstruktur darzustellen. Über das DOM kann zum Beispiel mit \gls{Javascript} auf einzelne Html Elemente zugegriffen werden, um diese zu verändern oder auszulesen,}
    \\\textit{--Begriff kommt vor auf Seite: }
}

\newglossaryentry{PDO}
{
    name=Php Data Objects,
    description={Die Php Data Objects Erweiterung (PDO) definiert eine Schnittstelle, um in \gls{Php} Skripten auf Datenbanken zuzugreifen. \gls{MySQL} definiert beispielsweise eine PDO-Schnittstelle, die es ermöglicht aus Php Skripten heraus auf Datensätze in der Datenbank zuzugreifen und diese zu bearbeiten.}
    \\\textit{--Begriff kommt vor auf Seite: }
}

\newglossaryentry{fetch-api}
{
    name=fetch API,
    description={Die fetch API ist eine Schnittstelle zum dynamischen Abrufen und Versenden von Daten über \gls{Http} Requests und kann in \gls{Javascript} standardmäßig verwendet werden. Im Vergleich zu den früher gebräuchlichen XMLHttpRequests bietet die fetch API eine deutliche einfachere Syntax sowie mächtige Funktionalitäten um dem asynchronen Charakter von Anfragen an einen entfernten Server gerecht zu werden
            \cite{fetchAPI}.}
    \\\textit{--Begriff kommt vor auf Seite: }
}

\newglossaryentry{Css}
{
    name=Css,
    description={Abkürzung für Cascading Style Sheets. Css ist eine textbasierte Stylesheet-Sprache und wird in Verbindung mit \gls{Html} genutzt um die Position, Farbe und Größe der einzelnen Html Elemente zu definieren.\\ \\
            \textit{In dieser Arbeit verwendete Version: \textbf{CSS 3}}
            \cite{Css}}
    \\\textit{--Begriff kommt vor auf Seite: }
}

\newglossaryentry{fullcalendar}
{
    name=fullcalendar,
    description={Fullcalendar ist eine \gls{Javascript} Bibliothek und stellt fertige Funktionen zur Verfügung um Kalenderansichten und Termindaten einfach und übersichtlich in einem \gls{Html} Dokument anzuzeigen.\\ \\
            \textit{In dieser Arbeit verwendete Version: \textbf{FullCalendar v5.8.0}}
            \cite{fullCalendarWeb}}
    \\\textit{--Begriff kommt vor auf Seite: }
}

\newglossaryentry{jQuery}
{
    name=jQuery,
    description={jQuery ist eine \gls{Javascript} Bibliothek und stellt fertige Funktionen zur Verfügung um \gls{Html} Elemente dynamisch und effizient auszulesen, zu verändern oder einzublenden.\\ \\
            \textit{In dieser Arbeit verwendete Version: \textbf{jQuery v3.6.0}}
            \cite{jQuery}}
    \\\textit{--Begriff kommt vor auf Seite: }
}

%Keine Einrückung neuer Absätze
\setlength\parindent{0pt}

%Variable Textbausteine
\newcommand{\ipName}{Herr Maier }
\newcommand{\thesisTitle}{Menschzentrierte Gestaltung einer Terminvereinbarungssoftware}
\newcommand{\code}[1]{\texttt{#1}} % Inline Code Formatierung

%Metadaten
\title{\thesisTitle}
\author{Johannes Schnirring}



%Hier beginnt der Inhalt
\begin{document}

%Kopfzeile mit Kapitel und Seitennummer
\pagestyle{fancy}
\renewcommand{\chaptermark}[1]{\markboth{#1}{#1}}
\fancyhead[R]{\thepage}
\fancyhead[L]{\chaptername\ \thechapter\ --\ \leftmark}



%Einbinden der einzelnen Kapitel
\begin{titlepage}
    \centering
    {\scshape\Huge \thesisTitle \par}
    \vspace{1cm}
    {\scshape\Large \textbf{Bachelorarbeit}\par}
    \vspace{1.5cm}
    \begin{tabular}{r l}
        {\Large Studierender:}   & {\Large Johannes Schnirring}                     \\ \\
        {\Large Matrikelnummer:} & {\Large 33422833}                                \\ \\
        {\Large Studiengang:}    & {\Large Informatik (PO 2018) }                   \\ \\ \\ \\
        {\Large Erstprüferin:}   & {\Large Prof. Dr. Claude Draude }                \\
                                 & {\Large Gender/Diversity in Informatiksystemen } \\ \\
        {\Large Zweitprüfer:}    & {\Large Prof. Dr. Albert Zündorf }               \\
                                 & {\Large Software Engineering }                   \\ \\ \\ \\
        {\Large Semester:}       & {\Large Wintersemester 2022}                     \\ \\
        {\Large Datum:}          & {\Large \today}                                  \\ \\
    \end{tabular}
    \vfill
    {\large Universität Kassel}
\end{titlepage} %Titelseite
\thispagestyle{empty}
\begin{center}
    \Large\scshape Eidesstattliche Erklärung
\end{center}
Ich erkläre an Eides statt, dass ich die vorliegende Arbeit selbständig
verfasst, andere als die angegebenen Quellen/Hilfsmittel nicht benutzt, und die
den benutzten Quellen wörtlich und inhaltlich entnommenen Stellen als solche
kenntlich gemacht habe.\\ \\ \\ \\ \\ \\
\vspace{10cm}
\begin{minipage}[h]{0.4\linewidth}
    Kassel, am\dotfill\\
    \vspace*{2.5mm}
\end{minipage}
\hspace*{0.1\linewidth}
\begin{minipage}[h]{0.5\linewidth}
    \begin{center}
        \dotfill\\
        (Unterschrift)
    \end{center}
\end{minipage}
\newpage %Eidesstattliche Erklärung
\setcounter{tocdepth}{1}\tableofcontents %Inhaltsverzeichnis
\chapter{Einleitung}

\section{Zusammenfassung}

Diese Arbeit beschäftigt sich mit dem Entwicklungsprozess einer Software zur
Terminvergabe von Beratungsterminen. Diese Software soll in der Studienberatung
an der Universität Kassel eingesetzt werden. Der Entwicklungsprozess wird mit
Methoden des \textit{Human Centered Design} begleitet und umgesetzt. Diese
Arbeit erklärt die verwendet Methoden, sowie die dahinterliegende Theorie. Im
Rahmen dieser Arbeit wird die Software zur Terminvereinbarung vollständig
implementiert. Dabei werden alle Schritte des Gestaltungsprozesses beschrieben,
dokumentiert und ausgewertet.

Die allgemeine Studienberatung der Universität Kassel setzt die Software
Stubegru bereits seit 6 Jahren zum Management ihrer Beratungstermine,
Abwesenheiten und als Wissensdatenbank ein. Das Modul zur Vergabe der
Beratungstermine soll nun grundlegend neu implementiert werden und noch besser
an die Bedürfnisse der Mitarbeitenden angepasst werden. In Zusammenarbeit mit
einem Ansprechpartner aus der Abteilung Studium und Lehre, wird dieses neue
Modul entwickelt, designt und implementiert.\\ Die Bachelorarbeit begleitet
diesen praktischen Prozess, indem theoretische Grundlagen erläutert werden und
alle Entwicklungsschritte beschrieben werden. Die Theorie des Human Centered
Design gibt vor, wie ein Gestaltungsprozess von Software aussehen kann. Neben
grundlegenden Herangehensweisen bringt diese Theorie einige Methoden mit, die
im Laufe des Gestaltungsprozesses angewandt werden. Hierzu gehört
beispielsweise das \textit{Interview im Kontext}, das in dieser Arbeit
ebenfalls erklärt wird und dessen praktische Umsetzung dokumentiert,
ausgewertet und kritisch hinterfragt wird.

\section{Motivation}

\subsubsection{Historische Entwicklung}
Wir leben in einem Zeitalter, in dem immer mehr Prozesse des täglichen Lebens
durch technische Systeme wie Software begleitet werden. In dieser Situation ist
es von hoher Relevanz, dass alle Menschen unserer Gesellschaft die Möglichkeit
haben, diese Systeme gleichermaßen zu nutzen und zu verstehen. In den Anfängen
der Softwareentwicklung wurde Software in der Regel aus Sicht der
verarbeitenden Maschine gedacht: Wie kann die Maschine die gegebenen Aufgaben
möglichst schnell, fehlerfrei und effizient abarbeiten? Die Nutzenden dieser
Maschinen war meist ein eingeschränkter, speziell dafür ausgebildeter
Personenkreis. Diese Situation hat sich gänzlich verändert. Fast jeder Mensch
agiert heute viele Male am Tag mit technischen, softwaregetriebenen Systemen:
Bankautomaten, Smartphones, Supermarktkassen oder Computer. Außerdem wird der
Umgang mit solchen Systemen durch neue Entwicklungen wie Sprachsteuerung oder
Touchscreens immer interaktiver und unmittelbarer. Somit ist es enorm wichtig,
dass alle diese Systeme eine klare und nutzungsfreundliche Schnittstelle für
die Nutzenden bieten\cite{moserTesting}.

\subsubsection{Rolle des Human Centered Design}
Der Ansatz des Human Centered Design setzt genau an dieser Stelle an, indem er
Grundkonzepte und Methoden vermittelt, die das Design von
Benutzerschnittstellen thematisieren\cite{hcd}. Das Kernkonzept des Human
Centered Design ist es, den Blickwinkel aus der Perspektive der Menschen zu
wählen, die mit den technischen Systemen täglich interagieren müssen. Die zu
entwickelnden Systeme sollen Schnittstellen anbieten, die intuitiv und einfach
zu verstehen sind. Prozessabläufe sollen aus Sicht des Nutzenden abgebildet
werden und möglichst wenig durch technische Hintergründe eingeschränkt
sein\cite{HMI-HCD}.

\subsubsection{Praktische Anwendung}
Der Gestaltungsprozess nach den Methoden des Human Centered Design soll in
dieser Bachelorarbeit genauer untersucht werden. Zum einen, indem die
theoretischen Hintergründe genauer erläutert und Methoden präzise erklärt
werden. Den wesentlichen Teil dieser Arbeit stellt aber die Dokumentation der
praktischen Anwendung all dieser Methoden und Konzepte dar. Dies soll am
Beispiel einer Software zur Terminvergabe in der Studienberatung an der
Universität Kassel durchgeführt werden. Das entsprechende Modul für diese
Software soll im Rahmen dieser Arbeit geplant, gestaltet, implementiert und
getestet werden. All diese Schritte werden schriftlich aufgearbeitet,
festgehalten und hinterfragt. Das Ziel dieser Arbeit soll es also sein, den
Designzyklus des Human Centered Design am praktischen Beispiel einer
Terminvergabe-Software anzuwenden. Die bestehenden wissenschaftlichen
Erkenntnisse im Bereich der menschzentrierten Gestaltung sollen somit
erarbeitet, vertieft und auf Praxistauglichkeit geprüft werden. Ein weiterer
wichtiger Bestandteil ist es schließlich die theoretischen und methodischen
Grundlagen im Hinblick auf die gewonnenen praktischen Erfahrungen zu
hinterfragen und zu diskutieren.

\section{Aufbau}
Zu Beginn der Arbeit werden in Kapitel \ref{chapter:methods} zunächst die
verwendeten Grundlagen und Methoden des Human Centered Design genauer erläutert
und klar definiert. Exemplarisch für eine Methode wird hier das Konzept des
Interviews im Kontext genauer betrachtet und eingeführt. Die
Kapitel \ref{chapter:user-context} bis \ref{chapter:evaluation} beschäftigen sich
mit der praktischen Entwicklung des Softwaremoduls zur Terminvereinbarung
anhand des vorgestellten Designzyklus. Um dem Lesenden einen Eindruck über den
praktischen Einsatz der betrachteten Software an der Universität zu geben, wird
in Kapitel \ref{chapter:user-context} zunächst die Situation in der allgemeinen
Studienberatung erläutert. Anschließend wird die eingesetzte Software genauer
vorgestellt und deren Entwicklungsgeschichte aufgegriffen. Im weiteren Verlauf
soll thematisiert werden, aus welchen Gründen eine Überarbeitung dieser
Software notwendig ist und welches Ziel die allgemeine Studienberatung damit
beabsichtigt. Im Kapitel  \ref{chapter:user-requirements} wird ein Interview im
Kontext mit einem Mitarbeitenden der Studienberatung durchgeführt, dokumentiert
und ausgewertet. Die aus den Wünschen und Bedürfnissen der Nutzenden
entstehenden Anforderungen sollen ebenfalls zusammengefasst und dargestellt
werden. Als nächster Schritt soll in Kapitel \ref{chapter:implementation} die
praktische Umsetzung der technischen Anforderungen an die Software skizziert
werden. Implementierungsdetails, Screenshots und Codeschnipsel sollen in diesem
Teil die Dokumentation der Umsetzung veranschaulichen. Als letzter Aspekt der
praktischen Durchführung soll in Kapitel \ref{chapter:evaluation} das Feedback
der Nutzenden bei ersten Usertests festgehalten und aufbereitet werden. In
Kapitel \ref{chapter:conclusion} wird schließlich das Ergebnis des
durchgeführten Entwicklungsprozesses dargestellt. Die eingesetzten Methoden
werden nochmals aufgegriffen und kritisch hinterfragt: An welchen Stellen
können die theoretischen Grundlagen des Human Centered Design den
Entwicklungsprozess in der Praxis tatsächlich sinnvoll unterstützen? Gab es
eventuell auch Methoden, die in der praktischen Umsetzung problematisch waren
oder noch optimiert werden könnten? Der letzte Teil dieser Arbeit gewährt
schlussendlich einen Ausblick auf weitere Forschungsfelder des Human Centered
Design und die weitere Entwicklung der Software für die Studienberatung der
Universität Kassel.
\chapter{Methoden}
\label{chapter:methods}

\section{Human Centered Design}
Human Centered Design ist eine Methode, die den Entwicklungsprozess von
interaktiven Systemen und Software beschreibt. Die Kernthese ist hierbei, dass
die Nutzenden der Systeme und ihr Umfeld im Mittelpunkt aller Entwicklungs\hyphen\space und
Designprozesse stehen. Inhaltlich lässt sich die Theorie des Human Centered
Design in das Themenfeld der Mensch-Maschine-Interaktion
einordnen \cite{HMI-HCD}.

Die Entwicklung von Software war in den Anfängen geprägt von den technisch sehr
eingeschränkte Möglichkeiten des Computers. Software wurde größtenteils so
entwickelt, dass sie möglichst problemlos und effizient auf dem entsprechenden
Computer ausgeführt werden kann. Alle Entwicklungsprozesse wurden aus
Perspektive der ausführenden Maschine gedacht. Der Ansatz des Human Centered
Design setzt dieser maschinenzentrierten Vorgehensweise eine klare Alternative
entgegen. Martin Ludwig Hofmann beschreibt den Grundgedanken in \textit{Human
    Centered Design} so: \glqq{}[Es geht] nicht darum, mit Geräten zu denken,
sondern mit Menschen und ihren Weltanschauungen\grqq{} \cite{hcd}. Wie in
Kapitel \ref{chapter:introduction} erläutert, gewinnt dieser Ansatz in der
heutigen Zeit immer größere Bedeutung. Alan Dix erklärt in \textit{Human Computer
    Interaction}, dass sich die Art und Weise, wie Menschen und Computer
interagieren, in den letzten Jahrzehnten stark verändert hat. Während Computer
früher meist nur mathematische Berechnungen durchführten, sind Softwaresysteme
heute hochgradig interaktiv und sollen von allen Teilen der Gesellschaft
ungehindert genutzt werden können \cite{hci}.

Um die Methodik des Human Centered Design weiter zu spezifizieren, ist zunächst
festzuhalten, dass dem Designprozess an sich eine wichtige Rolle zugeschrieben
wird. Es reicht nicht aus, dass eine Software korrekt funktioniert, sich also
alle Berechnungen und Prozesse fehlerfrei ausführen lassen. Im Human Centered
Design liegt der Fokus auf der Interaktion zwischen dem Menschen und der
Maschine. In Abgrenzung zum \textit{User Centered Design} werden beim Human
Centered Design nicht nur die unmittelbaren Nutzenden mit einbezogen. Alle
Personen, die von der Gestaltung der Systeme betroffen sind, sollen während des
Designprozesses beachtet werden \cite{sequenceDiagrams}. Man betrachtet, wie die
Menschen mit dem Computer interagieren. Wie stoßen sie Prozesse an? Welche
Reaktionen erwarten sie von dem System? Die Betroffenen sollen mit den Systemen
möglichst intuitiv interagieren können. Die Benutzeroberfläche der Software
soll selbst erklären, welche Funktionen auf welchem Weg erreicht werden können.
Ein essentieller Bestandteil hierfür ist zu verstehen, wie die Systeme genutzt
werden. Relevante Fragestellung sind: In welchem Kontext werden die
Systeme eingesetzt? Welche Menschen arbeiten mit den Systemen? Welche
Informationen müssen die Nutzenden schnell erfassen können? Um diese Fragen
sinnvoll beantworten zu können, müssen die Betroffenen während der Designphase
kontinuierlich in den Entwicklungsprozess einbezogen werden. Ihre gesamte
Situation sowie ihre Bedürfnisse müssen im Designprozess untersucht und direkt
in das Produktdesign integriert werden \cite{hci}.

Masaaki Kurosu hält fest, dass ein weiterer wichtiger Aspekt des Human Centered
Design das Durchlaufen mehrerer Iterationen ist. Ein Produkt ist nach einem ersten
Entwicklungszyklus selten schon perfekt. Oftmals ist es wichtig, den Betroffenen
erste Ergebnisse zu präsentieren oder ihnen Prototypen zu zeigen. Das Feedback,
das Nutzende hierbei geben, muss dokumentiert und für weitere Iterationen der
Entwicklung aufgearbeitet werden. Wie Kurosu weiter ausführt, steht der
zyklische Ablauf des Human Centered Design auch in der entsprechende ISO-Norm
im Fokus \cite{kurosuHCI}.

\begin{figure}[H]
    \caption{Iteratives Vorgehen im Human Centered Design nach ISO 9241  \cite{ISO9241Graphic}.}
    \label{fig:iso-diagram}
    \centering
    \includegraphics[width=10cm]{HCD.png}
\end{figure}

Abbildung \ref{fig:iso-diagram} verdeutlicht den Ablauf eines Designprozesses in
mehreren Phasen. Nachdem geplant ist, welches Projekt umgesetzt werden soll,
beginnt der zyklische Kreislauf des Entwicklungsprozesses. Im ersten Schritt
liegt der Fokus auf der Analyse der Situation, in der das zu entwickelnde
System später eingesetzt werden soll. Nur wer den Kontext eines interaktiven
Systems kennt, kann die Schnittstelle zwischen Maschine und Mensch adäquat
gestalten. In Schritt zwei werden aus den geplanten Erwartungen an das System und der Analyse des Nutzungskontextes konkrete Anforderungen formuliert. Dabei steht noch nicht im Fokus, wie diese Anforderungen
technisch umgesetzt werden könnten. Vielmehr geht es darum, welche
Anforderungen das fertige System überhaupt erfüllen soll. Im dritten Schritt
geht es darum, diese Anforderungen tatsächlich umzusetzen und technische
Lösungen zu implementieren. Gibt es erste Ergebnisse, wird der vierte Schritt
relevant: Die umgesetzten Lösungen müssen im tatsächlichen Nutzungskontext
evaluiert werden. Das setzt wieder eine intensiven Austausch mit den
Betroffenen des Systems voraus. Mit dem so gewonnen Feedback kann der Prozess
wieder von vorne beginnen. Fehlende Features können nachgebessert,
missverständliche Schnittstellen klarer gestaltet und Probleme im praktischen
Einsatz minimiert werden \cite{ISO9241}.

G.A. Boy berichtet in \textit{The Handbook of Human-Machine Interaction}, wie
wichtig es ist, dem späteren Nutzenden der Systeme erste Prototypen und Ideen zu
präsentieren. Oftmals wissen die Nutzenden gar nicht, welche technischen
Möglichkeiten es gibt oder welche alternativen Bedienkonzepte in ihrem Kontext
besonders gut funktionieren könnten. Im Mittelpunkt steht also das Zuhören und Eingehen auf die
Nutzenden und ihr Umfeld. Allerdings kann auch das Einbringen neuer, für die Nutzenden
unbekannter Lösungsansätze von hoher Bedeutung sein \cite{HMI-HCD}.

\section{Interview im Kontext}

Ein wesentlicher Bestandteil im Entwicklungsprozess nach den Methoden des Human
Centered Design ist der intensive Austausch mit den Betroffenen der Systeme.
Dafür ist es wichtig, dass Softwareentwickler und Designer von
Benutzerschnittstellen in den direkten Austausch mit den Nutzenden der Systeme
gehen. Um diesen Prozess strukturiert und zielführend zu durchlaufen, kann
die Methode des \textit{Interviews im Kontext} benutzt werden.

Beim Interview im Kontext geht es darum, die Nutzenden der Systeme in dem
Umfeld zu beobachten, in dem die Software tatsächlich auch im praktischen
Alltag eingesetzt wird. Der Fokus liegt also auf dem Umfeld der Interaktion mit
technischen Systemen. In herkömmlichen Interviews begegnen sich Interviewender und
Interviewter oftmals in einer neutralen Umgebung, wie beispielsweise einem
Konferenzraum und sprechen über die Prozesse, für die sich der Interviewende
interessiert. Beim Interview im Kontext steht nicht das Sprechen über, sondern das direkte Miterleben des Prozesses im Mittelpunkt \cite{contextualDesign}. In \textit{Manual on Human-Computer Interaction} legen die Autoren den Fokus auf die genaue Beobachtung der
Nutzenden in der Interaktion mit den Systemen. Mit dieser Methode könne man
noch viel mehr praxisnahe Details erfassen, als wenn man mit den Nutzenden nur
über das System spreche. Spricht man mit Nutzenden beispielsweise darüber, wie
sie eine Software bedienen, gibt es möglicherweise viele Dinge, die sie nicht
erwähnen, weil sie vergessen wurden, nicht relevant erscheinen oder Nutzende
nicht wissen, wie sie genau darüber sprechen sollen \cite{hciHandbook}.

Dies kann in der Praxis bedeuten, dass man sich mit den Nutzenden der Systeme
in ihren Büroräumen trifft und für mehrere Stunden mit dabei ist, wenn sie
ihrer Arbeit nachgehen und mit den technischen Systemen interagieren. Die Rolle
des Interviewenden ist dabei aus einer beobachtenden Perspektive zu verstehen.
Der Interviewte soll die Richtung bestimmen, in die sich das Interview
entwickelt. Der Interviewende hört hauptsächlich zu und beobachtet, wie die
Personen in ihrer Umgebung, ihrem Kontext agieren \cite{hciHandbook}. Dazu kann
der Interviewende Notizen machen, um eine spätere Auswertung des Interviews zu
erleichtern. An passenden Stellen kann der Interviewende gezielte Nachfragen
stellen, um beispielsweise Hürden bei der Interaktion mit den Systemen zu
provozieren. Eine weitere Intention für Nachfragen kann aber auch sein, die
Gedanken und Emotionen des Nutzenden zu erfragen, welche er beim Benutzen des
Systems empfindet.

\begin{figure}[h]
    \caption{Während eines Interviews im Kontext können viele Aspekte beobachtet werden, die über das benutzte System selbst hinausgehen \cite{johannesGrafik}.}
    \centering
    \includegraphics[width=0.9\textwidth]{Interview im Kontext Schaubild.pdf}
\end{figure}

Das Ergebnis eines Interviews im Kontext ist also eine ausführliche Beobachtung
der Interaktion von Nutzenden mit den entsprechenden Systemen im alltäglichen
Kontext. Diese Beobachtungen können schriftlich festgehalten werden oder auch
durch Video- und Tonaufzeichnungen ergänzt werden. Im Nachgang des Interviews
müssen die Beobachtungen ausgewertet und analysiert werden. Ziel ist es, aus den Beobachtungen des Interviews konkrete Nutzungsanforderung an das zu
entwickelnde System zu formulieren \cite{HMI-HCD}. Hier können beispielsweise Featurelisten,
Sequenzdiagramme oder User-Scenarios zum Einsatz kommen \cite{sequenceDiagrams}.
\chapter{Nutzungskontext}
\label{chapter:user-context}

Um den Bedarf und Entstehungsprozess der Software besser einordnen zu können,
werden nun sowohl der Kontext als auch die Prozessabläufe der Nutzergruppe kurz
skizziert. Die Beschreibung der Situation im Team hilft zu erkennen, wie sich
die aktuelle Softwarelösung in den Arbeitsalltag eingliedert, welche
Prozessabläufe bereits gut durch Software begleitet werden, und an welchen
Stellen noch Optimierungsbedarf besteht.

\section{Studienberatung der Universität Kassel}
Als Nutzergruppe in dieser Bachelorarbeit werden die Mitarbeitenden der
Abteilung \textit{Studium und Lehre} der Hochschulverwaltung an der Universität
Kassel dienen. Zu deren alltäglichen Aufgaben gehört es, alle erdenklichen
Organisationen zu übernehmen, die Studierenden und Lehrenden ein erfolgreiches
Zusammenarbeiten an der Universität ermöglichen. Hierzu gehören beispielsweise
das Einschreiben und Exmatrikulieren von Studierenden, die Durchführung des
Bewerbungsverfahrens, das Betreiben der Information Studium und die allgemeine
Studienberatung. In dieser Arbeit wird der Fokus auf die Mitarbeitenden der
Studienberatung als Teilgruppe der Abteilung Studium und Lehre gesetzt.

Die allgemeine Studienberatung der Universität Kassel berät Studierende zu
allen Fragen rund um das Studium. Insbesondere bei persönlichen Problemen mit
der Fertigstellung des eigenen Studiums hilft die Studienberatung mit einem
persönlichen Lösungsgespräch und kann an weitere fachspezifische
Beratungsstellen weiter vermitteln. Des Weiteren bietet die allgemeine
Studienberatung verschiedene Workshops und Seminare an. Hierbei können sich
Studierende mit Fokus auf bestimmten Fragestellungen austauschen und
Qualifikationen im Umgang mit herausfordernden Studiensituationen erlangen.
Auch Schnupperkurse für Schüler werden von der allgemeinen Studienberatung
angeboten um den Studieninteressierten einen möglichst unmittelbaren Einblick
in den Studienalltag zu gewähren \cite{studBeratungKsWeb}.

Eines der zentralen Themen im Alltag der Studienberatung sind Beratungstermine.
Studienberatende müssen Termine mit den Ratsuchenden vereinbaren und abstimmen.
Beratungstermine können über verschiedene Kontaktkanäle stattfinden: Es ist
eine telefonische Beratung oder auch eine Beratung über eine Videokonferenz
möglich. Ebenso ist es auch möglich einen persönlichen Beratungskontakt vor Ort
zu vereinbaren. Über all diese Termine muss jeder Studienberatende den
Überblick behalten und gleichzeitig neue Terminanfragen schnell beantworten
können. Um diesen Prozess zu erleichtern und mögliche Fehler, wie
beispielsweise Terminüberschneidungen, zu minimieren, wird hierfür die Software
\gls{Stubegru} eingesetzt.

\section{Aktuelle Softwarelösung: Stubegru}
Stubegru ist ein umfangreiches Softwarepaket für akademische Beratungsstellen.
Die webbasierte Groupware begleitet viele Arbeitsabläufe im Alltag einer
Beratungsstelle an einer Hochschule. In einem Softwaresystem vereint Stubegru
eine Wissensdatenbank in Form eines Wikis, sowie ein Dashboard mit vielen
Modulen für spezifische Workflows. So können über Stubegru an der Universität
Kassel beispielsweise Abwesenheiten der Abteilung, Telefonnotizen und
Beratungskontakte verwaltet werden. Jeder Mitarbeitende der Abteilung hat über
einen eigenen Account Zugriff auf die Software, die er im Browser aufrufen
kann. Die Software hilft dabei, tagesaktuelle Informationen schnell und
übersichtlich allen Mitarbeitenden zur Verfügung zu stellen und bei Bedarf
langfristige und ausführliche Informationen mit wenigen Klicks zur Verfügung zu
stellen \cite{stubegruWebsite}.

Das wichtigste Modul der eingesetzten Software für die Studienberatung ist der
Kalender zur Terminvereinbarung von Beratungsterminen. Über dieses Modul können
in einem zweistufigen Prozess Termine für Ratsuchende freigegeben und an die
entsprechenden Studierenden und Studieninteressierten vermittelt werden.

\subsection*{Zeitslots erstellen}
Im ersten Schritt können die Studienberatenden freie Zeitslots für ihre
Beratungstermine anlegen. Diese Zeitslots zeigen an, dass der entsprechende
Beratende in der eingestellten Zeitspanne potenziell Zeit für ein
Beratungsgespräch hat. Bei der Erstellung der Zeitslots können weitere
Attribute wie der Beratungskanal (Online Meeting, Telefongespräch oder
Präsenztermin) konfiguriert werden. Außerdem können Mail Templates verknüpft
werden, die im Falle einer Terminvergabe den Ratsuchenden per Email über alle
wichtigen Informationen zum Termin informieren.

\subsection*{Terminvergabe durch Erstinformation}
Im zweiten Schritt werden die eingestellten Zeitslots durch Hilfskräfte der
Erstinformation an Ratsuchende vergeben. Die Erstinformation der Universität
Kassel berät Studierende und Studieninteressierte zu allen Fragen rund ums
Studium übers Telefon, Email und an einer Servicetheke vor Ort. Bei
tiefgehenden Fragen und spezifischen Anliegen verweisen die Mitarbeitenden an
die entsprechenden Sachbearbeitenden oder Beratungsstellen. Die Erstinformation
ist auch für das Vereinbaren von Beratungsterminen mit der allgemeinen
Studienberatung verantwortlich. Sind die Mitarbeitenden der Erstinformation in
Kontakt mit einem Kunden, der einen Termin in Anspruch nehmen möchte, können
sie in der Software alle freien Zeitslots der Beratenden einsehen und einen
passenden Termin mit dem Kunden vereinbaren. Wenn ein freier Zeitslot vergeben
wird und fest mit einem Kunden verknüpft ist, wird eine Email an den Beratenden
versendet, die über alle Details wie Adresse, Kontaktinformationen und Anliegen
der Ratsuchenden informiert. Des Weiteren wird eine Mail an den Kunden
versendet, die auf dem zuvor verknüpften Mailtemplate aufbaut und dynamisch
terminrelevante Informationen einsetzt, wie beispielsweise Datum und Uhrzeit
des Termins, oder eine Wegbeschreibung zum Beratungsraum.

\subsection*{Auskunft bei Terminabsage}
Eine weitere Verwendung des Kalendermoduls tritt ein, wenn Kunden der
Erstinformationen Fragen zu einem bereits vereinbarten Termin haben oder diesen
absagen möchten. In diesem Fall können die Hilfskräfte der Erstinformation über
eine Suchfunktion gezielt nach den vereinbarten Terminen des Kunden suchen und
weitere Auskünfte geben.

\subsection*{Datenschutz}
Da Datenschutz in Beratungsszenarien eine wichtige Rolle einnimmt, kann
lediglich der verantwortliche Beratende das Anliegen der ratsuchenden Person
einsehen. Zu Auskunftszwecken können aber alle Mitarbeitenden der Abteilung
sehen, wann ein Beratungstermin mit welchem Beratenden vereinbart wurde.
Datensätze zu vergangenen Beratungsterminen werden täglich gelöscht, sodass
möglichst wenig personenbezogenen Daten in der Datenbank gespeichert werden
müssen. Über ein differenziertes Berechtigungssystem der Software Stubegru kann genau gesteuert werden, welche Nutzungsgruppen Beratungstermine anlegen
und vergeben dürfen.

\section{Historische Entwicklung der Software Stubegru}
Die Software Stubegru wurde ursprünglich von mir als Hilfskraft der Abteilung
Studium und Lehre an der Universität Kassel erstellt und betreut. Da die
Software nun langfristig an der Uni Kassel und auch an anderen deutschen
Hochschulen eingesetzt werden soll, wurde ein Prozess gestartet, um eine
Professionalisierung und nachhaltige Betreuung der Software zu gewährleisten.
In diesem Rahmen wurde die Software auch für andere Hochschulen zur Verfügung
gestellt und unter einer OpenSource Lizenz veröffentlicht. Im Zuge dieser
Veröffentlichung wurde in Zusammenarbeit mit der Hochschule Bremen eine
grundlegend überarbeitete Variante der Software Stubegru erstellt, die im
Vergleich zu der bisherigen Version deutlich flexibler ist und mehr
Anpassungsmöglichkeiten bietet. Wie Erich Gamma in \textit{Elemente
    wiederverwendbarer objektorientierter Software} betont, bieten wiederverwendbar
gestaltete Softwarestrukturen und abstrakte Implementierungsansätze die Option,
Softwaremodule in verschiedenen Kontexten zu verwenden. Dies erfordert im
Softwareentwurf allerdings eine vorausschauende Planung und einen hohen
Abstraktionsgrad \cite{wiederverwSoftware}. Somit ist der Einsatz der Software
Stubegru nun an verschiedenen Hochschulen mit verschiedenen Arbeitsabläufen
realisierbar. Um von dieser neuen, überarbeiteten Softwareversion auch an der
Universität Kassel zu profitieren, ist wiederum eine grundlegende
Überarbeitung des Moduls zur Terminvergabe der Beratungstermine für die
allgemeinen Studienberatung notwendig. Dieser Prozess soll im Rahmen dieser
Bachelorarbeit begleitet, wissenschaftlich untermauert und evaluiert werden.

\subsection*{Fehlende Features}
In der bisherigen Softwareversion gibt es einige Features, die noch nicht
vollständig funktionieren oder nicht optimal auf den tatsächlichen
Arbeitsalltag zugeschnitten sind. An diesen Stellen soll das neue Modul zur
Terminvereinbarung verbessert und noch weiter an die Bedürfnisse der Nutzenden
angepasst werden. Im Rahmen der Softwareüberarbeitung mit der Hochschule Bremen
wurde für die neue Version von Stubegru bereits ein Kalendermodul zur Vergabe
von Beratungsterminen entwickelt. Dieses Modul weist allerdings noch einige
Probleme auf, um reibungslos im Arbeitsalltag der allgemeinen Studienberatung
an der Universität Kassel eingesetzt zu werden. In der \textit{Bremer Version} läuft
das Erstellen und Vergeben eines Beratungstermins in einem einzigen Schritt ab.
Ein zentraler Punkt, um die überarbeitete Software auch in der allgemeinen
Studienberatung der Universität Kassel nutzen zu können, ist der zweistufige
Prozess der Terminvergabe. Hier müssen Beratende die Möglichkeit haben, zuerst
freie Terminslots freizugeben, die dann in einem getrennten zweiten Schritt
durch Mitarbeitende der Erstinformation an Ratsuchende vergeben werden können.

Welche Anpassungen im Detail notwendig sind, um die Software optimal in der
Studienberatung einsetzen zu können, soll in den folgenden Kapiteln methodisch
herausgearbeitet werden und durch Dokumentation von praktisch durchgeführten
Nutzerstudien und Gesprächen mit den verantwortlichen Personen ergänzt werden.
Diese Bachelorarbeit soll insbesondere den Designprozess strukturiert begleiten
und wissenschaftliche Methoden aufzeigen, um Softwareentwicklern und Nutzenden
eine möglichst gute Zusammenarbeit zu ermöglichen.
\chapter{Nutzungsanforderungen}

\section{Interview im Kontext}
\label{subsection:IIK}

Das Modul zur Terminvereinbarung der Software Stubegru soll auf den
Arbeitsalltag der allgemeinen Studienberatung der Universität Kassel angepasst
werden. Dies soll mit Methoden des Human Centered Design umgesetzt werden. Ein
zentraler Bestandteil des Human Centered Design ist der enge und stetige
Austausch mit den Nutzenden des Softwaresystems \cite{hci}. Um den Änderungsbedarf eines bestehenden Softwaresystems einschätzen
zu können wird im Human Centered Design häufig die Methode des
\textit{Interviews im Kontext} gewählt.\cite{contextualDesign} Diese Methode
eignet sich besonders zu Beginn des Entwicklungsprozesses, da wenig
Vorkenntnisse über die eingesetzte Software und das Umfeld, in dem Software
eingesetzt wird, bekannt sein muss. Die Softwareentwickler können so einen
guten Einstieg finden, um einen Überblick zu gewinnen, welche Funktionen die
fertige Software am Ende unterstützen muss. Auch lässt sich durch ein genaues
Beobachten beim Interview herausarbeiten, in welchem Kontext die Software im
tatsächlichen Arbeitsalltag genutzt wird und welche weiteren Faktoren die
Nutzenden der Systeme beeinflussen.


\subsection*{Rahmenbedingungen IiK}
Als erster Schritt wurde ein Termin für ein Interview im Kontext mit \ipName vereinbart. \ipName ist einer von drei
Mitarbeitenden der allgemeinen Studienberatung der Universität Kassel. Zu
seinen Aufgaben gehört die Betreuung der Software Stubegru und deren Einsatz in
der Abteilung Studium und Lehre. Seit über sechs Jahren arbeitet \ipName
bereits gemeinsam mit Hilfskräften an dem Aufbau und der Optimierung der
Software Stubegru um den täglichen Arbeitsalltag seines Teams optimal zu
unterstützen. Ich habe mich persönlich mit \ipName in seinem Büro im Campus
Center der Universität getroffen. Dort hat er mir an seinem Schreibtisch
gezeigt, wie er mit der alten Version der Software Beratungstermine erstellt
und vergeben kann. \ipName saß vor mir und hatte Maus und Tastatur in der Hand.
Ich saß hinter ihm auf einem Stuhl und habe auf einem iPad Notizen
mitgeschrieben. Für die Dauer von einer Stunde hat \ipName mir gezeigt, wie er
die Software aktuelle nutzt, welche Features für ihn sehr wichtig sind und an
welchen Stellen noch Verbesserungspotenzial besteht.

\subsection*{Detaillierter Ablauf IiK}
Am Anfang habe ich \ipName gebeten, mir einmal zu zeigen, wie er einen
Beratungstermin in der Software anlegen und vergeben kann. Dies ist der
Workflow, der im Arbeitsalltag am häufigsten vorkommt und daher eine hohe
Priorität im Designprozess hat. \ipName klickte sich durch die verschiedenen
Eingabefelder um einen freien Zeitslot für einen Beratungstermin anzulegen.
Hierbei erwähnte er, dass es ganz wichtig ist, dass Datum und Uhrzeit des
Beratungstermins mit wenigen Klicks über ein Date-/Timepicker mit der Maus
eingeben werden können. Eine Datumseingabe über die Tastatur würde er nicht
bevorzugen.

\begin{figure}[H]
    \caption{Datepicker im Formular zur Erstellung eines Zeitslots}
    \centering
    \includegraphics[width=0.9\textwidth]{screen_old_datepicker.png}
\end{figure}

Beim Eintragen mehrere Termine wäre es auch besonders praktisch, dass das zuvor
eingegebene Datum stehen bleibt und direkt ein weiterer Zeitslot für den
gleichen Tag angelegt werden kann, ohne dass er nochmal extra das Datum
auswählen muss. Diem meisten der weiteren Felder sind Dropdown Menüs, mit
wenigen Elemente., Die Auswahl der richtigen Werte kann \ipName schnell
vornehmen. Bei der Auswahl der verknüpften Räume werden beispielsweise die
Räume, die mit seinem Nutzeraccount verknüpft sind, ganz oben in der
Auswahlliste angezeigt. Da eine Beratung in der Regel in den eigen Räumen
stattfindet, ist hier eine schnelle Auswahl für den Normalfall möglich. In
einer Spezialsituation, in der ein größerer Beratungstermin beispielsweise in
einem gemeinsamen Gruppenraum stattfinden, ist aber auch solch eine Auswahl
möglich.

\begin{figure}[H]
    \caption{Dropdown zur Auswahl des Beratungsraums. Der eigene Raum wird immer als oberstes angezeigt}
    \centering
    \includegraphics[width=0.9\textwidth]{screen_old_roomdropdown.png}
\end{figure}

Nachdem der Zeitslot für den Termin angelegt ist, wird der entsprechende Tag in
der Kalenderübersicht nun grün hinterlegt. Dies ist ein Zeichen für die
Hilfskräfte der Erstinformation, dass an diesem Tag noch freie Zeitslots
verfügbar sind.

\begin{figure}[H]
    \caption{Kalenderübersicht. Grüne gefärbte Tage zeigen noch freie Zeitslots an. Rot gefärbte Tage weisen auf vergeben Zeitslots hin}
    \centering
    \includegraphics[width=0.9\textwidth]{screen_old_module.png}
\end{figure}

Durch ein Mouseover über den entsprechenden Tag in der Monatsübersicht kann man
die genauen Termine mit Informationen über die Uhrzeit, den zuständigen
Beratenden und die Anzahl der freien Plätze sehen. \ipName erklärt mir, dass
die kompakte Monatsansicht mit den farblich hervorgehobenen Terminslots bereits
eine sehr gute Lösung ist, damit die Hilfskräfte auf einen Blick erfassen
können, an welche Tagen sie den Kunden noch Beratungsgespräche anbieten können.
Sobald alle Plätze der Beratungstermine an einem Tag vergeben sind, wird dieser
im Kalender rot markiert. \glqq So sehen Hilfskräfte mit einem Blick sofort,
dass sie hier keinen Termin mehr vergeben werden können\grqq, erklärt \ipName
\cite{claves}.

\begin{figure}[H]
    \caption{Bewegt man den Mauszeiger über einen Tag, erscheinen weiteren Informationen zu den Zeitslots an diesem Tag}
    \centering
    \includegraphics[width=0.9\textwidth]{screen_old_hover.png}
\end{figure}

Soll nun ein Zeitslot tatsächlich vergeben werden, klickt man auf den
entsprechenden Tag in der Monatsansicht und es öffnet sich ein Modal. Dies ist
ein Fenster, welches sich über den anderen Bildschirminhalt legt und dem Nutzer
somit deutlich anzeigt, dass hier eine Aktion im neu geöffnet Fenster notwendig
ist. \ipName zeigt mir, wie die Mitarbeitenden der Erstinformation in diesem
Detail-View die freien Zeitslots an die ratsuchenden Personen vergeben können.
In einer Liste werden, nach Uhrzeit sortiert, alle Termine untereinander
angezeigt. Neben jedem freien Termin steht ein Button zum Vergabe dieses
Zeitslots zur Verfügung.

\begin{figure}[H]
    \caption{Der Detail-View: Eine Liste mit drei freien Zeitslots am entsprechenden Datum}
    \centering
    \includegraphics[width=0.9\textwidth]{screen_old_daylist.png}
\end{figure}

\ipName zeigt mir wie eine Hilfskraft der Erstinformation nun einen solchen
Zeitslot vergeben könnte. Nach Klick auf den "Vergabe-Button" klappt ein
Formular auf, indem Name, Kontaktdaten und Anliegen der Ratsuchenden erfasst
werden können.

\begin{figure}[H]
    \caption{Formular zum vergeben eines Zeitslots an eine ratsuchende Person}
    \centering
    \includegraphics[width=0.9\textwidth]{screen_old_clientdata.png}
\end{figure}

Nachdem alle personenbezogenen Daten korrekt erfasst wurden kann der Termin nun
endgültig gebucht werden. Hierzu klicken die Hilfskräfte auf den Button
"Bestätigen". \ipName erklärt mir, dass dies ein sehr wichtiger Schritt ist:
Solange eine Mitarbeitender der Erstinformation das Formular zum Erfassen der
persönlichen Daten des Ratsuchenden geöffnet hat, wird dieser Zeitslot mit
einer Sperre versehen. So wird verhindert, dass dieser Zeitslot von einem
Kollegen vergeben werden kann, während man selbst gerade mit dem Ratsuchenden
beispielsweise am Telefon die persönlichen Daten und das Anliegen bespricht.
Sollte nach dem Aufklappen des Formulars der entsprechende Zeitslot doch nicht
vergeben werden, ist es deshalb notwendig, dass die terminvergebende Person auf
"Abbrechen" klickt, um die Sperre dieses Zeitslots aufzuheben und ihn somit für
die Kollegen wieder freizugeben. \ipName betont, dass dieser Schritt manchmal
nicht ganz intuitiv ist, und für die Hilfskräfte daher in Einführungsschulungen
immer besonders hervorgehoben wird. Es wäre allerdings deutlich schlimmer einen
Termin doppelt zu vergeben und somit mindestens einer ratsuchenden Person
wieder absagen zu müssen, als einen Zeitslot versehentlich zu sperren.

\begin{figure}[H]
    \caption{Detail-View: Ein Zeitslot wurde nun vergeben und ist für den entsprechenden Kunden reserviert. Hilfskräfte können nur den Namen des Ratsuchenden einsehen}
    \centering
    \includegraphics[width=0.9\textwidth]{screen_old_assigned_hiwi.png}
\end{figure}

Ist der Termin nun erfolgreich vergeben, können alle Nutzenden der Software
einsehen an welche Person dieser Termin vergeben wurde. Meldet sich ein
Ratsuchender beispielsweise einige Tage später noch einmal bei der
Erstinformation und möchte wissen, wann sein Beratungstermin stattfindet,
können die Mitarbeitenden der Erstinformation diese Auskunft aus der Software
ablesen. Aus Datenschutzgründen können allerdings keine weiteren
personenbezogenen Daten des Beratungstermins ausgelesen werden. Lediglich der
Studienberatende, bei dem der Termin stattfindet, bekommt beim Aufruf des
Detail-Views weitere Details wie Kontaktdaten und Anliegen der ratsuchenden
Person angezeigt.

\begin{figure}[H]
    \caption{Detail-View: Der verantwortliche Beratende kann weitere personenbezogene Details einsehen}
    \centering
    \includegraphics[width=0.9\textwidth]{screen_old_assigned.png}
\end{figure}

\ipName hat nun den zweistufigen Workflow zur Terminvergabe einmal komplett
durchgespielt und mich auf viele Details hingewiesen. Während \ipName mir
gezeigt und erzählt hat, wie die Terminvergabe in der aktuellen Softwareversion
abläuft, habe ich in Stichworten mitgeschrieben, welche Bemerkungen und
Auffälligkeiten er besonders betont hat.

\section{Auswertung des Interviews}

\subsection*{Einleitung Auswertung}
Während bisher der detaillierte Ablauf des Interviews im Kontext geschildert
wurde, sollen im Folgenden die wesentlichen Kernaspekte nochmals
zusammengefasst werden, die während des Interviews notiert wurden. Das
Augenmerk liegt hierbei auf Beobachtungen, die Konsequenzen für den
Designprozess des überarbeiteten Kalendermoduls zur Terminvergabe
hervorbringen.

\subsection*{Methode der Auswertung}
Während dem Interview habe ich mir alle relevant erscheinenden Aussagen von
\ipName auf einem iPad notiert. Wurden im weiteren Gesprächsverlauf noch
ergänzenden Informationen zu den einzelnen Punkten deutlich, habe ich diese in
den Notizen stichpunktartig an die entsprechenden Themen angefügt. Im Nachgang
des Interviews mussten diese Notizen nun sorgfältig analysiert und ausgewertet
werden. Hierzu bin ich die einzelnen Themen durchgegangen und habe die
entsprechenden Ansichten und Klickpfade in der Software nochmals nachgespielt.
In einem neuen Dokument habe ich nun die herausgearbeiteten Problematiken
zusammengefasst um die zu Grunde liegenden Zusammenhänge klarzustellen und zu
spezifizieren. Dies entspricht dem zweiten Schritt des iterativen Design Zyklus
des Human Centered Design nach ISO 9241 \cite{iso9241}

\subsection*{Spannende Erkenntnisse}
\label{subsection:SpannendeErkenntnisse}

Im Folgenden werden nun drei Punkte exemplarisch vorgestellt, die während des
Interviews aufgefallen sind. Anhand dieser drei verschiedenen bestehenden
Probleme wird der Designprozess des Human centered Design beispielhaft
durchlaufen.

\subsubsection{Kompakte Ansicht Kalender (mit Farben)}
In der alten Softwareversion, die an der Uni Kassel bisher zum Einsatz kam,
werden alle freien und vergeben Zeitslots der Beratungstermine in einer
tabellarischen Monatsansicht dargestellt.

\begin{figure}[H]
    \caption{Tabellarische Ansicht der Zeitslots mit Einfärbungen der einzelnen Tage}
    \centering
    \includegraphics[width=0.9\textwidth]{screen_old_module.png}
\end{figure}

Durch die farblichen Markierungen der einzelnen Tage können Nutzenden auf einen
Blick erfassen, ob an diesem Tag Beratungsslots eingetragen wurden und ob unter
den eingetragenen Zeitslots noch freie Termine vorhanden sind. Ein grün
markierter Tag bedeutet, dass an diesem Tag noch mindestens ein freier
Beratungsslot vorhanden ist. Ein rot markierter Tag bedeutet, dass an diesem
Tag Beratungstermine stattfinden, diese allerdings bereits alle an ratsuchende
Personen vergeben sind. In der überarbeiteten Version der Stubegru Software,
die in Zusammenarbeit mit der Hochschule Bremen entstanden ist, wurde diese
kompakte tabellarische Übersicht durch eine größere umfangreiche Ansicht
ausgetauscht, die durch die Javascript Bibliothek \textit{full
    calendar}\cite{fullCalendarWeb} gerendert wird.

\begin{figure}[H]
    \caption{Monatsübersicht der Beratungstermine in der Bremer Version}
    \centering
    \includegraphics[width=0.9\textwidth]{screen_bremen_month_view.png}
\end{figure}

Dieses neue Ansicht ermöglicht auf den ersten Blick zu sehen, zu welcher
Uhrzeit die Termine stattfinden und einzelne Termine aus der Monatsübersicht
direkt anzuklicken. Allerdings bietet diese Ansicht keine Möglichkeit, Tage je
nach freien Plätzen rot oder grün darzustellen. Dies ist jedoch ein wichtiges
Feature für die zweistufige Terminvergabe an der zentralen Studienberatung der
Universität Kassel. An dieser Stelle braucht es eine Idee um den Hilfskräften
der Erstinformation auf den ersten Blick anzuzeigen, ob sie an diesem Tag noch
freie Terminslots vergeben können.

\subsubsection{Suche nach Teilnehmern}
Manchmal kommt es vor, dass Ratsuchende, die bereits einen Beratungstermin
vereinbart haben, nochmals in Kontakt mit der Erstinformation treten, um
weitere Fragen zum Termin zu stellen. Auch kommt es vor, dass das genaue Datum
oder die Uhrzeit vergessen wurden. In diesem Fall sollen die Hilfskräfte der
Erstinformation möglichst schnell Auskunft über die angefragten Details geben
können. Hierfür immer alle vergebenen Beratungstermine manuell durchzulesen,
ist zeitlich ein großer Aufwand. Es braucht also ein Feature, sodass die
Mitarbeitenden der Erstinformationen direkt nach Terminen und weiteren
organisatorischen Daten dieser Termine suchen können. Wenn Ratsuchende
beispielsweise am Telefon ihren Namen nennen, werden sie manchmal nicht
einwandfrei verstanden. Ein Suche nach Teilnehmernamen der Termine sollte also
auch funktionieren, wenn der Name nicht exakt in der gleichen Schreibweise
eingegeben wird, wie er im Datensatz des Beratungstermins in der Datenbank
hinterlegt ist.

\subsubsection{Telefonnummer Anzeige ("Silbentrennung")}
In der Regel wird bei einer Terminvergabe die Telefonnummer der ratsuchenden
Person erfasst. Der zuständige Studienberatende kann den Datensatz bei Bedarf
aufrufen und diese Telefonnummer einsehen. Dies passiert in der Regel, wenn der
Berater vor einem Beratungstermin nochmals telefonisch Details mit der
ratsuchenden Person abklären möchte. Der Berater wählt also die angezeigt
Telefonnummer in seinem Telefon. Während des Interviews im Kontext zeigt sich,
dass die Eingabe längerer Telefonnummern manchmal Fehler mit sich bringt, da
Ziffern vertauscht oder vergessen werden. Den Beratenden wäre hier eine
wertvolle Hilfe an die Hand gegeben, wenn eine Darstellung langer
Telefonnummern möglich wäre, die ein direktes und intuitives eintippen in die
Telefontastatur erleichtern.

\section{Gestaltungslösungen entwickeln}

Nachdem nun die Problematiken und Herausforderungen des neuen Softwaremoduls
verdeutlicht wurden, sollen im nächsten Schritt konkrete Ideen entwickelt
werden, wie die erkannten Problematiken und Anforderungen in der Praxis
umgesetzt werden können. Alan Dix betitelt diese Phase in "Human Computer
Interaction" als "Requirements specification" und betont, dass der Fokus in
diesem Schritt darauf liegt, die notwendigen Funktionalitäten und Features der
Software grob zu beschreiben. Von besonderer Bedeutung in diesem Schritt des
Designzyklus sind Zusammenhänge und Abhängigkeiten zwischen einzelnen
Komponenten. Exakte Implementierungsdetails hingegen sind in dieser Phase noch
nicht von großer Bedeutung und sollten erst im nächste Schritt genauer
betrachtet werden.

\subsection*{Methode der Erarbeitung}
Durch die Auswertung des Interviews im Kontext sind Nutzungsanforderungen an
das neue Modul zur Terminvereinbarung entstanden. Um diese lose formulierten
Nutzungsanforderungen später implementieren zu können, werden sie in diesem
Schritt weiter konkretisiert. Es sollen erste Ideen entstehen, wie die
Bedürfnisse der Nutzenden durch einzelne Komponenten der Software umgesetzt
werden können. In diesem Fall wird mit Skizzen der einzelnen Views und
Formulare gearbeitet. Für jedes Szenario, dass Nutzende beim späteren Verwenden
der Software durchlaufen, wird eine digital gezeichnete Skizze erstellt.
Hierbei werden bereits wichtige Elemente wie Buttons, Formularfelder und
Hinweisboxen skizziert. Durch Markierungen und Notizen an der Skizze werden die
Funktionen dieser Elemente definiert.

\subsubsection{Kompakte Ansicht Kalender (mit Farben)}

Die Übersicht aller Termine eines Monats ist die Ansicht, die Nutzende beim
Aufruf der Software als erstes sehen. Den größten Raum nimmt die tabellarische
Ansicht der einzelnen Tage des Monats ein. In den einzelnen Feldern werden
Terminslots, nach Uhrzeit sortiert, aufgelistet. Neben der Uhrzeit des Termins
wird der Titel eines jeden Termins angezeigt. Die einzelnen Termine werden
farblich entweder grün oder rot eingefärbt, um auf den ersten Blick zu
kennzeichnen, ob es sich um einen freien Terminslot (grün) oder um einen
bereits vergebene Termin (rot) handelt. Wenn an einem Tag viele Zeitslots
angelegt werden, wird das Feld für diesen Tag automatisch größer, sodass alle
Termine Platz finden. Sollten an jedem Tag sehr viel Termine angelegt werden,
könnte die tabellarische Monatsansicht so lang werden, dass sie unter Umständen
nicht mehr vollständig auf den Bildschirm passt. Dies wäre unpraktisch, da dann
nicht mehr alle Termine eines Monats auf einen Blick erfasst werden könnten. In
der Phase der Evaluation sollte Diese Problematik berücksichtigt werden und
eine Abschätzung getroffen werden, wie viele Termine im praktische Einsatz
tatsächlich pro Tag angelegt werden.

\begin{figure}[H]
    \caption{Monatsübersicht der Beratungstermine mit farblichen Markierungen}
    \centering
    \includegraphics[width=0.9\textwidth]{doodle_month_view.jpeg}
\end{figure}

Über der tabellarischen Ansicht der Tage befindet sich eine horizontale Leiste, die den aktuell angezeigten Monat betitelt und Kontrollelemente beinhaltet um in den vorherigen bzw nächsten Monat zu wechseln. Ein Button, um nach einigem hin- und herblättern wieder den aktuelle Monat anzuzeigen, könnte in einigen Anwendungsszenarien viele Klicks ersparen. Über der Leiste mit dem Monat befindet sich eine weitere Kontrollleiste. Diese enthält einen Button um einen neuen Zeitslot anzulegen. Dieser Button sollte nur für Nutzeraccounts von Beratenden sichtbar sein. Hilfskräfte der Erstinformation sollen Zeitslots nur vergeben, aber nicht selbst anlegen können. Daneben befindet sich eine Suchleiste um schnell nach Namen von ratsuchenden Personen suchen zu können. Ganz rechts gibt es schließlich noch einen Button um weitere Einstellungen vorzunehmen. Durch einen Klick auf diesen Button mit einem Zahnrad Symbol soll ein Dropdown-Menü aufklappen, in dem Filter für die Ansicht der Termine gesetzt werden können.

\begin{figure}[H]
    \caption{Filtereinstellungen der Kalenderansicht. Das Dropdown Menü öffnet sich durch Klick auf den Zahnrad Button}
    \centering
    \includegraphics[width=0.9\textwidth]{example-image-a}
\end{figure}

In diesem Menü kann über Toggles eingestellt werden, ob nur eigene Termine oder
auch fremde Termine in der Monatsansicht dargestellt werden sollen. Mit
\textit{eigenen Terminen} sind Termine gemeint, die den eigenen Benutzeraccount
als zuständigen Beratenden hinterlegt haben. Außerdem kann ein Filter gesetzt
werden um ausschließlich freie Termine anzuzeigen. Dies kann besonders für
Hilfskräfte bei der Vergabe freier Termine relevant sein, da bereits vergeben
Zeitslots in diesem Fall irrelevante Informationen sind, die von freien
Zeitslots ablenken.

\subsubsection{Suche nach Teilnehmern}

Die Suchfunktion ist ein weiterer Aspekt, dem in dieser Ausarbeitung besonderer
Aufmerksamkeit gewidmet ist. Über das Freitextfeld in der oberen Kontrollleiste
können Nutzende nach Namen von Ratsuchenden suchen, an die bereits Termine
vergeben wurden. Tippt man einige Buchstaben in das Suchfeld ein, klappt eine
Box mit Ergebnisvorschlägen unter der Suchleiste auf und schiebt den restlichen
Inhalt (die tabellarische Monatsansicht) nach unten. In diese Box werden zur
Suchanfrage passenden Termine dargestellt. Für jeden Termin wird in einer Zeile
der Titel, der Name des Ratsuchenden, der Name des Beratenden sowie Datum und
Uhrzeit aufgelistet. Neben jedem Datensatz erscheint ein Button mit einem
Augensymbol. Durch einen Klick darauf wird der entsprechende Termin in der
Detailansicht geöffnet.

\begin{figure}[H]
    \caption{Suche nach Terminen eines Ratsuchenden mit Ergebnisliste}
    \centering
    \includegraphics[width=0.9\textwidth]{doodle_search_view.jpeg}
\end{figure}

Wichtig für die Suchfunktion ist, dass passende Ergebnisse auch angezeigt
werden, wenn die Eingabe in der Suchleiste eventuell Fehler enthält oder noch
nicht vollständig ist. Durch solche automatischen Ergebnisvorschläge wird das
Suchen für die Nutzenden erleichtert und Fehlerquellen minimiert. Dadurch, dass
Nutzenden schon während dem Tippen der ersten Buchstaben ein aktives und
konstruktives Feedback erhalten, fühlt sich die Nutzung der Software
dynamischer und flüssiger an. \cite{autoCompletion} Wenn Mitarbeitende der
Erstinformation ihre Kunden am Telefon beispielsweise nicht ganz genau
verstehen, können sie mit diesen automatischen Ergebnisvorschlägen trotzdem den
passenden Termin finden. Allerdings muss bei solchen automatisiertenVorschlägen
darauf geachtet werden, dass nicht zu viele unnötige oder unpassende Vorschläge
angezeigt werden. Diese würden Nutzende von den eigentlich gesuchten
Ergebnissen ablenken und sich somit nachteilig auf die User-Experience
auswirken. \cite{autosuggModeration}

\subsubsection{Telefonnummeranzeige ("Silbentrennung")}

Durch einen Klick auf den Termin in der Monatsübersicht öffnet sich die
Detailansicht des zugehörigen Termins und weitere Eigenschaften des Datensatzes
werden angezeigt. Alternativ kann ein Termin auch über die Suchfunktion
gefunden und dann über den Button mit dem Augensymbol in der Detailansicht
aufgerufen werden. In dieser Ansicht können Nutzeraccounts mit der
entsprechenden Berechtigung nochmals Details des Termins bearbeiten oder den
Termin löschen.

\begin{figure}[H]
    \caption{Detailansicht eines Termins, der bereits an eine ratsuchende Person vergeben wurde}
    \centering
    \includegraphics[width=0.9\textwidth]{doodle_client_details.jpeg}
\end{figure}

Wenn Beratende nach der Vereinbarung eines Termins nochmals auf telefonischem
Weg Absprachen oder Vorgespräche mit den Ratsuchenden erledigen möchten, können
sie die Telefonnummer der entsprechenden Person in der Detailansicht eines
Beratungstermins einsehen. Die Beobachtung des Nutzungsverhaltens während des
Interviews im Kontext hat gezeigt, dass es umständlich ist, lange
Telefonnummern zu erkennen und korrekt in die Tastatur des Telefons einzugeben.
Im Gespräch mit \ipName kam der Wunsch auf, Telefonnummern an dieser Stelle so
zu formatieren, dass sie intuitiver erfasst und abgetippt werden können. Der
Standard für das Formatieren von Telefonnummern in Deutschland wird durch DIN
5008 geregelt. Diese Norm beschäftigt sich mit Formatierungsstandards für
Briefe und Anschreiben. Hier wird das Trennen der Vorwahl vom Rest der Nummer
durch ein Leerzeichen vorgeschrieben. Weitere Formatierung, wie beispielsweise
das aufteilen der Ziffern in kleinere Blöcke wird hier nicht thematisiert.
\cite{din5008}

\begin{figure}[H]
    caption{So werden nationale Festnetz- und Mobilfunknummern nach DIN 5008 richtig geschrieben. Quelle: \cite{phoneFormatBlog}}
    \centering
    \includegraphics[width=0.9\textwidth]{grafik-telefonnummer-national.png}
\end{figure}

Wissenschaftliche Untersuchungen zeigen, dass das Eingeben und Ablesen von
Telefonnummern in Interaktion mit den entsprechenden Maschinen ein relevantes
Details ist. Dieser Prozess sollte durch die technischen Systeme möglichst
intuitiv und nutzerfreundlich gestaltet werden. \cite{humCompPhoneNumbers}.
Eine Unterteilung der Ziffern in kleinere Blöcke, von beispielsweise vier
Ziffern pro Block, erhört die Lesbarkeit deutlich und ermöglicht es dem
menschlichen Gehirn einen Ziffernblock in einem Blick direkt zu erfassen und
auf die Telefontastatur zu übertragen. \cite{phoneFormatBlog}
\cite{numberRecognition} \cite{numberRepres}
\chapter{Gestaltungslösungen implementieren}
\label{chapter:implementation}

Im vorherigen Schritt wurden konkrete Nutzungsanforderungen herausgearbeitet
und Ideen für passende Gestaltungslösungen formuliert. Nun soll das folgende
Unterkapitel sich mit der praktischen Umsetzung der Anforderungen beschäftigen.
Dieser Prozess entspricht dem dritten Schritt \glqq{}Gestaltungslösungen
entwickeln, die die Nutzungsanforderungen erfüllen\grqq{} des Designzyklus im
Human Centered Design nach ISO 9241\cite{iso9241}.

\section{Kundenwünsche und technische Machbarkeit}
Auch wenn im Human Centered Design immer die Nutzenden und nicht das technische
System im Fokus stehen sollte,kommt man nicht drumherum die technischen Details
genauer zu betrachten. Dadurch ist es ganz besonders wichtig in dieser Phase
die Bedürfnisse und Erwartungen der Nutzenden nicht aus den Augen zu verlieren.
Eine klare und intuitive Schnittstelle für die Nutzenden sollte in diesem
Prozess eine höhere Priorität einnehmen als eine Lösung, die technisch am
einfachsten umzusetzen ist. K. Holtzblatt verwendet hier den Begriff
\textbf{Kohärenz} und meint damit, das sich ein System für den Nutzer so
zusammenhängend und natürlich wie möglich anfühlen soll: \glqq{}The challenge
is to keep the system work model coherent, so that it supports the users and
fits with their expectations while extending and transforming their work
practice as prescribed by the vision.\grqq{}\cite{contextualDesign}.
Gleichzeitig müssen alle Lösungsansätze natürlich auch tatsächlich programmiert
werden können. Hierbei kann nicht ignoriert werden, dass viele externe Faktoren
die Machbarkeit bestimmter Lösungskonzepte in der Praxis einschränken. G.A. Boy
erwähnt in \textit{The Handbook of Human-Machine Interaction} einige dieser
Faktoren: \glqq{}Design work is often constrained by various external factors
in the development organization and the marketplace (policies, standards,
competitive products, past and planned products, schedules, resource
budgets).\grqq\cite{HMI-HCD} So ermöglichen verwendete Programmiersprachen,
eingesetzte Frameworks und bereits existierende Softwareteile es, bestimmte
Konzepte sehr elegant umzusetzen. Andere Gestaltungsideen sind, beschränkt
durch äußere Faktoren, unter Umständen gar nicht realisierbar.\cite{HMI-HCD}

\subsubsection{Überleitung konkreter Anwendungsfall Stubegru}
In dem konkreten Fall des erarbeiteten Moduls zur Terminvergabe für die
Studienberatung ist hier ein wichtiger Gesichtspunkt, dass es sich um ein Modul
innerhalb eines bereits bestehenden Softwarepakets handelt. Tech-Stack,
Programmiersprache und Schnittstellen zu anderen Modulen sind hier bereits
vorgeben und können nicht allein durch die Wünsche der Nutzenden geformt
werden.

\subsection*{Techstack Stubegru}
Im Folgenden sollen nun die technischen Gegebenheiten und Grundlagen der
Software \textbf{Stubegru} vorgestellt werden. Das neue System zur
Terminvereinbarung muss sich als Modul in das Softwarepaket Stubegru einbinden
lassen und somit einige Standards und Schnittstellen der Software zur Verfügung
implementieren.

Die Software Stubegru ist webbasiert und wird über einen Browser aufgerufen.
Dementsprechend sind die verwendeten Technologien im Frontend Html, Css und
Javascript. Die Kommunikation mit dem Backend wird durch asynchrone
Http-Requests umgesetzt, die durch Javascript Methoden initiiert werden. Das
Backend besteht aus vielen einzelnen Php Dateien, welche die per Http
übermittelten Daten auslesen und aufarbeiten. Als Datenbank kommt eine
relationale mySQL Datenbank zum Einsatz, die von den Php Skripten über
Php-Database-Objects (PDO) angesprochen wird. Die abgerufenen Daten der Php
Skripte werden, als JSON codiert zurück an das Frontend geschickt und dort von
Javascript Methoden weiterverarbeitet. Mithilfe der Javascript Library jQuery
werden die entsprechende Element im Document Object Model (DOM) angepasst und
somit für den Nutzenden grafisch dargestellt. Das Softwarepaket Stubegru ist
grundlegend modular aufgebaut, sodass für jede Funktionalität oder jeden
Prozess ein eigenes Modul programmiert wird, das genau diese Aufgabe übernimmt.
Jedes Modul besteht aus den entsprechenden Anzeigeelementen, die in einer Html
Datei formuliert werden. Des weiteren grafische Designregeln, die als CSS Datei
angelegt werden und der eigentlichen Logik, die in Javascript implementiert
werden muss. Im Modul zur Terminvereinbarung kommt zusätzlich die Javascript
Library \textbf{fullcalendar} zum Einsatz. Diese bietet eine einfache
Schnittstelle um Termindaten in einer monatlichen Übersicht darzustellen und
bietet dem Nutzenden standardisierte Kontrollelemente zum Interagieren mit der
Kalenderansicht. Funktionalitäten wie das Rendern einer Monatsübersicht, oder
das Wechseln zwischen den einzelnen Monaten sind über diese Bibliothek bereits
abgedeckt und müssen nicht eigenständig implementiert werden. Gleichzeitig
schränkt die Verwendung dieser Bibliothek die Möglichkeiten in der Darstellung
der Termine auch ein, sodass lediglich die angebotenen Ansichten
(Monatsansicht, Wochenansicht, Tagesansicht, Terminliste) eingesetzt werden
können.

\section{Nutzungsanforderungen formalisieren}
\label{subsection:sequenceDiagrams}

\subsubsection{Einführung Sequenzdiagramme}
Um die erarbeiteten Nutzungsanforderungen in praktische Programmierung
umzusetzen wurden zunächst Sequenzdiagramme erstellt. Diese Diagramme werden
jeweils für einen zusammenhängenden Workflow gezeichnet und skizzieren alle
kleinere Teilschritte, die in diesem Workflow nacheinander abgearbeitet werden
sollen. Ziel eines Sequenzdiagramms ist es, die einzelnen Aktivitäten, die ein
Nutzender beim Verwenden der Software durchläuft,
darzustellen.\cite{holtzblattCDEvolved} An solchen Diagrammen können auch die
Zusammenhänge dieser Aktivitäten untereinander abgelesen werden. Karen L.
McGraw verweist in \textit{User-centered Requirements} darauf, dass diese Art
von Diagrammen sehr hilfreich sein können, um primäre Workflows und
zusammenhängende Prozesse zu identifizieren\cite{sequenceDiagrams}. Ian
Alexander führt in \textit{Scenarios, Stories, Use Cases} ganz ähnliche
Diagramm ein, die er Scenario Process Models nennt. Auch hier geht es darum ein
Abbild der Teilprozesse zu schaffen, die ein Nutzender beim Verwenden eine
Software durchläuft. I. Alexander nennt als weiteren Vorteil dieser Diagramme,
das noch unklare oder fehlende Daten innerhalb eines Workflows schnell zu
erkennen sind. Über eingehende und ausgehende Pfeile kann an den
Sequenzdiagrammen abgelesen werden durch welche Aktionen der jeweilige Workflow
angestoßen wird, beziehungsweise welche anderen Workflows durch
Nutzerinteraktionen angestoßen werden können. Teile einzelner Elemente der
grafischen Oberfläche werden zkizzenhaft dargestellt, um einen intuitiven
Eindruck festzuhalten, wie die Nutzenden durch diesen Workflow navigieren und
welche Steuerelemente sie auf dem Bildschirm verwenden können. Durch farbige
Anmerkungen an einzelnen Elementen oder Arbeitsschritten wird auf bestimmte
Details hingewiesen, die bei der Implementierung zu beachten sind. So kann
beispielsweise angemerkt werden, dass nach einem Klick auf einen Button, ein
Popup mit einer Bestätigungsaufforderung angezeigt werden soll. Die
Sequenzdiagramme sind bewusst formlos gehalten und nur grob skizziert. Dies
spiegelt die schnelle und intuitive Erstellung dieser Diagramme wieder. Es geht
noch nicht darum den Prozess in einer sehr strukturierten Darstellungsform
aufzuzeichnen, die man direkt in Code übersetzen könnte. Vielmehr soll intuitiv
und spontan festgehalten werden, welche Schritte ein Nutzender durchlaufen
könnte, wenn er jenen Workflow anstößt.

\subsubsection{Beispiel Sequenzdiagramm [Vergebenen Termin aufrufen]}
Beispielhaft für den Entstehungsprozess dieser Sequenzdiagramme wird im
Folgenden ein Diagramm gezeigt und näher erläutert. Dieses Diagramm beschreibt
den Workflow wenn ein Nutzender die Detailansicht eines bereits vergebenen
Termins aufruft.

\begin{figure}[H]
    \caption{Sequenzdiagramm, Laden der Detailansicht für einen vergebenen Termin}
    \centering
    \includegraphics[width=0.9\textwidth]{flow_termin_aufrufen_unvergeben.jpeg}
\end{figure}

Die grünen Pfeile links oben beschrieben, auf welchem Weg Nutzenden diesen
Workflow anstoßen können. Die Detailansicht eines vergebenen Termins kann in
diesem Fall auf zwei verschiedenen Wegen geöffnet werden. Entweder klickt ein
Nutzender auf einen Termin in der Monatsübersicht oder er klickt auf ein
Suchergebnis in der Ergebnisliste der Suche nach Teilnehmenden. Im rechten
oberen Teil des Diagramms wird stichpunktartig festgehalten, welche
Teilschritte notwendig sind um die Termindetails sinnvoll darstellen zu können:
Zunächst muss das Modal\todo{Modal im Glossar} über dem bestehenden
Bildschirminhalt eingeblendet werden. Im nächsten Schritt muss der Titel des
Modals angepasst werden. Der Standardtitel \textit{Termin erstellen} macht in
diesem Kontext keinen Sinn, da der Termin bereits existiert. Daher wird der
Titel des Modals in diesem Fall auf \textit{Termindetails} geändert. Im
weiteren Verlauf müssen die Daten des Termins (Datum, Uhrzeit, Titel) in die
entsprechenden Input Felder geladen werden. Diese Input Felder sollen als
\textit{disabled} dargestellt werden. Das bedeutet, dass man den eingetragenen
Inhalt lesen, ihn jedoch nicht bearbeiten kann. Ein bereits an einen Kunden
vergebener Termin kann nicht bearbeitet werden, solange der Datensatz des
Kunden nicht entfernt wurde. So soll vermieden werden, dass beispielsweise das
Datum des Beratungstermins nachträglich geändert wird, ohne dass der Kunde
darüber informiert wird. Hier wird den Nutzenden also bewusst die
Funktionalität des nachträglichen Änderns verboten, um keine Missverständnisse
mit der Kundschaft aufkommen zu lassen. Sollte tatsächlich einmal das Datum
eines Beratungstermins verändert werden, muss der Kundendatensatz zunächst
entfernt und dann, nach dem Anpassen des Datums neu hinzugefügt werden, Dies
hat den Vorteil, dass die entsprechenden Informationsmails (Terminabsage, Neue
Terminbestätigung) korrekt an den Kunden und den Beratenden gesendet werden und
somit für beide Parteien nachvollziehbar ist, an welchem Datum der Termin nun
tatsächlich stattfinden soll.

\begin{figure}[H]
    \caption{Verschiedene Bereiche des Modals mit Kontrollelementen}
    \centering
    \includegraphics[width=0.9\textwidth]{doodle_modal_overview.jpeg}
\end{figure}

Im unteren Teil des Diagramms wird durch kleine Skizzen festgehalten, welche
Kontrollelemente den Nutzenden im Workflow zur Verfügung stehen. Die
Textit{Vergabestelle} bezeichnet den Bereich, indem Buttons zum Interagieren
mit den Kundendaten dargestellt werden. In diesem Fall ist lediglich der Button
\textit{Kundendaten löschen} sichtbar. Ein nachträgliches Bearbeiten der
Kundendaten soll nicht möglich sein, da auch hier inkonsistente Status beim
Versand der Mails an den Kunden entstehen könnten. Wird beispielsweise die
Mailadresse des Kunden geändert, nachdem er für den Versand eines
Feedback-Fragebogens eingetragen wurde, kann der Link für den Fragebogen nicht
mehr an die korrekte Adresse verschickt werden. Diese Problematik wird im
Diagramm grafisch durch die orangene Infobox am rechten Rand verdeutlicht.
Schlussendlich wird im Diagramm dargestellt welche Buttons in der Fußleiste des
Modals angezeigt werden sollen. Hier soll lediglich der Button
\textit{Abbrechen} aktiv sein. Über diesen Button wird das Modal ohne Übernahme
von Änderungen geschlossen und ein anderer Termin kann aufgerufen werden. Die
Buttons zum Speichern und Löschen des Termins sollten nicht angeklickt werden
können. Wie bereits erwähnt, soll ein Ändern oder Löschen der Termindaten nur
möglich sein, wenn die verknüpften Kundendaten entfernt wurden. Die Buttons
sollen jedoch nicht vollkommen unsichtbar sein, sondern lediglich ausgegraut
dargestellt werden. Die blaue Box im Diagramm ergänzt, das bei einem
Mouse-Hover über diese ausgegrauten Buttons ein Hinweis angezeigt werden soll.
Dass die Buttons nicht vollständig ausgeblendet werden, sondern stattdessen
deaktiviert mit einem Hinweis dargestellt werden, soll dazu führen, dass
Nutzende eine einheitliche Benutzeroberfläche vorfinden und bei dem Versuch mit
den deaktivierten Buttons zu interagieren, automatisch darauf hingewiesen
werden, welche Schritte notwendig sind, um die gewünschte Funktion nutzen zu
können. Dieser Ansatz soll dazu beitragen, dass auch unerfahrene Nutzende
schnell und intuitiv verstehen, wie eine Interaktion mit dem System gedacht
ist.

Im Anhang (Kapitel \ref{section:anhang:Sequenzdiagramme}) finden sich weitere
drei Sequenzdiagramme die folgende Prozesse beschreiben: \textit{Vergebenen
    Termin aufrufen}, \textit{Neuen Termin erstellen} und \textit{Termin an Kunden
    vergeben}.

\section{Technische Umsetzung}
Im folgenden Abschnitt soll der Fokus nun auf die technischen Details der
Implementierung gelegt werden. Wesentliche Fragestellungen sind hier: Wie
werden die Nutzungsanforderung in der Praxis umgesetzt? Welche Designpattern
und Softwarekonzepte werden verwendet? Wie werden Schnittstellen zwischen
Frontend und Backend gestaltet? Und über welche Protokolle werden Daten
ausgetauscht? Wichtig ist es, an dieser Stelle wieder im Blick zu behalten,
dass es sich um eine Modul in einer bereits bestehenden Software handelt. Somit
sollten bereits existierende Konzepte und Schnittstellen nach Möglichkeit
wieder verwendet werden. Durch das Aufgreifen bestehender Lösungsstrategien und
Entwurfsmuster wir das gesamte Softwarepaket einheitlich strukturiert und ist
somit leichter zu warten. Gleichzeitig müssen einige Funktionen nicht von Grund
auf neu implementiert werden, somit kann Zeit und Arbeit in diesem Prozess
erspart werden.\cite{wiederverwSoftware} Im Folgenden werden einige
Designentscheidung, Grafiken und Codeschnipsel präsentiert um einen groben
Einblick in die technische Umsetzung zu erlauben. Der vollständige Sourcecode,
sowie alle Commits zur Programmierung des Moduls sind im entsprechenden GitHub
Repository einzusehen.\cite{stubegruRepo}

\subsubsection{Datenmodell}
Die eigentlichen Datenmodelle mit denen in dem Modul zur Terminvereinbarung
gearbeitet wird sind relativ simpel gehalten. Es gibt drei relevante Typen von
Datensätzen: Termine, Mailtemplates und Beratungsräume. Die Datensätze, die
einen Termin beschreiben, enthalten außerdem auch Informationen über den
Ratsuchenden, der diesen Termin gebucht hat. Diese Datensätze werden im
Folgenden als Meeting bezeichnet und sollen anhand eines UML Diagramms
detailliert veranschaulicht werden.\todo{UML erklären?}

\begin{figure}[H]
    \caption{UML Diagramm der Klasse Meeting. Stellt alle Methoden und Eigenschaften eines Objektes dar, das einen Termin repräsentiert}
    \centering
    \includegraphics[width=0.9\textwidth]{uml_meeting.jpeg}
\end{figure}

Die Klasse Meeting stellt zunächst einige statische Eigenschaften und Methoden
zur Verfügung. Das Array \textit{meetingList} enthält zur Laufzeit alle aktuell
bekannten Meetings in einer unsortierten Liste. Durch die statische Funktion
\textit{fetchMeetings()} werden alle Datensätze aus der Datenbank geladen und
in der \textit{meetingList} gespeichert. Über die Funktion \textit{getById()}
kann ein bestimmtes Meeting aus der \textit{meetingList} angesprochen werden.
Die weiteren Funktionen dienen dazu, die Eigenschaften eines Meetings an das
Backend zu senden, um die neuen Werte in der Datenbank zu speichern oder zu
löschen. Die statische Methode\textit{createOnServer(properties)} sendet die
Daten eines neuen Meetings an den Server. Diese Methode erstellt bewusst keine
neue Instanz vom Typ Meeting, da es keinen Sinn macht ein Meeting Objekt zu
verwenden, solange der Server dieses Meeting noch gar nicht kennt. Im Anschluss
an die \textit{createOnServer(properties)} Methode sollte stets die Methode
\textit{fetchMeetings()} aufgerufen werden, um auf das neu erstellte Meeting
zugreifen zu können. Die Funktion\textit{updateServer()} wird auf einem
bestehenden Objekt vom Typ Meeting aufgerufen und sendet die aktuellen lokalen
Eigenschaften an den Server. Hierbei wird die Funktion\textit{toFormData()}
verwendet, um die Eigenschaften des Objektes in das passende Format zu bringen.
Die so aufbereiteten Daten werden über den asynchronen Aufruf der
\textit{fetchApi}\cite{fetchAPI} mit einer HTTP Request an das entsprechenden
PHP Skript auf dem Server gesendet.

\lstinputlisting[language=Java, caption=Senden von lokalen Änderungen eines Meetings an den Server]{listings/updateOnServer.js}

Über die Methode \textit{deleteOnServer()} kann ein bestehendes Meeting auf dem Server gelöscht werden. Durch einen anschließenden Aufruf von \textit{fetchMeetings()} wird das Meeting dann auch in der lokalen \textit{meetingList} gelöscht. Außerdem stellt ein Objekt vom Typ Meeting noch zwei Methoden zur Verfügung um Kundendaten zu verknüpfen beziehungsweise zu löschen. Diese Funktionen stoßen in den jeweiligen PHP Skripten auf dem Server weitere Workflows an, wie beispielsweise das Versenden einer Bestätigungsmail an die ratsuchende Person.

\lstinputlisting[language=PHP, caption=PHP Skript zum Erstellen eines neuen Meetings in der Datenbank]{listings/save_meeting.php}

\subsubsection{Workflow: Vergebenen Beratungstermin aufrufen}

In Kapitel \ref{subsection:sequenceDiagrams} wurden exemplarisch alle
Teilschritte erklärt, die nötig sind um einen bereits vergebenen
Beratungstermin aufzurufen und in der Detailansicht darzustellen. Für jeden
dieser einzelnen Schritte wurde eine entsprechende Funktion in der Klasse
\textit{CalendarModal} implementiert. Die Funktion
\textit{setModalVisible(isVisible)} kann beispielsweise aufgerufen werden um
das Modal ein- oder auszublenden. Die Klasse \textit{CalendarController}
verwendet nun all diese Funktionen und bildet so den kompletten beschriebenen
Workflow Schritt für Schritt ab. Das folgende Listing beschreibt den Ablauf
aller einzelnen Schritte durch den sequenziellen Aufruf der einzelnen
Funktionen:

\lstinputlisting[language=Java, caption=Öffnen der Detailansicht eines vergebenen Beratungstermins]{listings/openAssignedMeeting.js}

\subsubsection{Monatsansicht mit fullcalendar darstellen}

Als abschließendes Beispiel der konkreten Implementierung wird die Klasse
\textit{CalendarView} vorgestellt. Diese ist für das Darstellen der
Terminübersicht zuständig. An dieser Stelle wird die Javascript Bibliothek
\textit{fullcalendar}\cite{fullCalendarWeb} verwendet um den Aufbau der
Monatsansicht deutlich zu vereinfachen. Die Klasse \textit{CalendarView}
referenziert auf eine Instanz vom Typ \textit{FullCalendar} und kann darüber
die wichtigsten Funktionen zum Löschen und Hinzufügen weiterer Termine
aufrufen. Im folgenden Listing wird die Funktion
\textit{addMeetings(meetingList)} gezeigt. Diese Funktion wird vom
\textit{CalendarController} aufgerufen und bekommt als Parameter eine Liste von
Meetings übergeben. Aus den übergebenen Objekten vom Typ \textit{Meeting}
werden zunächst Objekte generiert, die alle Eigenschaften erhalten, welche die
\textit{fullcalendar} Bibliothek benötigt um die Termine in der Übersicht
korrekt anzeigen zu können. Die so erstellten Datensätze werden anschließend in
\textbf{eigene Termine} und \textbf{fremde Termine} sortiert. Über die Funktion
\textit{fullCalendar.addEventSource(eventSource)} werden die Listen mit den
eigenen und den fremden Terminen hinzugefügt. Über den weiteren Parameter
\textit{color} wird in diesem Fall die farbliche Darstellung der Termine in der
Monatsübersicht gesteuert.

\lstinputlisting[language=Java, caption=Hinzufügen von Terminen in die Monatsübersicht von fullcalendar]{listings/addMeetings.js}
\chapter{Gestaltungslösungen evaluieren}
\label{chapter:evaluation}

\subsubsection{Einleitung Usertest}
Im vorherigen Kapitel wurden einzelne Aspekte der konkreten Implementierung der
Gestaltungslösungen vorgestellt. Im nächsten Schritt sollen die so entwickelten
Prototypen mit einem Nutzer getestet werden. Dies entspricht dem vierten
Schritt \textit{Gestaltungslösungen aus der Benutzerperspektive evaluieren} im
menschzentrierten Gestaltungsprozess der ISO Norm \cite{ISO9241}. Christian
Moser erklärt den möglichen Ablauf eines solchen Usertests wie folgt:
\glqq{}Bei Tests mit Benutzern wird den Testteilnehmern der Kontext anhand des
Szenarios erklärt und dann werden die Aufgaben gestellt. Dabei wird beobachtet,
wie gut sie diese mit Hilfe des Prototyps lösen können.\grqq{}\cite{moserTesting}. In diesem Fall wird der Test gemeinsam mit \ipName von der allgemeinen Studienberatung der Universität Kassel durchgeführt. Gemeinsam haben wir uns eine Stunde Zeit genommen um das neu entwickelte Modul zur Terminvereinbarung zu testen. Wir sitzen gemeinsam im Büro von \ipName und können die Software so direkt an seinem Dienstrechner testen um den Nutzungskontext möglichst authentisch zu gestalten.

\section{Usertest}
\subsubsection{Monatsübersicht}
Der Usertest beginnt damit, dass \ipName die Software Stubegru im Browser
aufruft und den View mit dem neuen Kalendermodul zur Terminvergabe öffnet. Sein
erster Blick fällt auf die Monatsübersicht der Termine im November.

\begin{figure}[H]
    \caption{Monatsübersicht der Termine im November.}
    \centering
    \includegraphics[width=0.9\textwidth]{screen_new_month_view.png}
\end{figure}

In dieser Übersicht kann er einsehen, welche Termine für den aktuellen Monat
eingetragen sind. Anhand der farbigen Punkte \todo{Listing mit color attr
    anpassen} kann \ipName auf einen Blick erkennen, welche Termine noch frei sind
und welche bereits an ratsuchende Kunden vergeben wurden. Neben dem farbigen
Punkt kann für jeden Termin eingesehen werden, bei welchem Studienberatenden
dieser Termin stattfindet. \ipName merkt an, dass es im Vergleich zur alten
Version möglich ist direkt in der Übersicht zu sehen, zu welchem Zeitpunkt die
Termine stattfinden und wie viele Termine an einem Tag angeboten werden. Er
testet was passiert, wenn viele Termine am gleichen Tag angelegt werden.
Zufrieden stellt er fest, dass die Zelle für den entsprechenden Tag automatisch
größer wird, wenn viele Termine eingetragen werden. \ipName überlegt weiterhin,
dass die Monatsübersicht dadurch eventuell so lang werden könnte, dass nicht
mehr alle Zeilen gleichzeitig auf den Bildschirm passen könnten. Wir
diskutieren wie viele Termine üblicherweise pro Tag angeboten werden und
einigen uns darauf die aktuelle Ansicht zunächst beizubehalten und in der
Praxis zu testen.

\begin{figure}[H]
    \caption{Toggles im Dropdown Menü um die angezeigten Termine zu filtern. Hier werden nur freie Termine dargestellt}
    \centering
    \includegraphics[width=0.9\textwidth]{screen_new_filter.png}
\end{figure}

Ich weise \ipName darauf in, dass er über den Button mit dem Zahnrad weitere
Filtermöglichkeiten hat, um eine überfüllte Monatsübersicht auf die
wesentlichen Termine zu reduzieren. Den Button hätte er zunächst nicht bemerkt,
nachdem er ihn einmal gefunden hat versteht er die Bedienung über die beiden
Toggles aber schnell. Er merkt positiv an, dass Änderungen an den
Filtereinstellungen direkt übernommen werden und die Auswirkung der Filter
somit leicht zu begreifen ist. Ihm fällt auf, dass der Filter für eigene
Termine für die Hilfskräfte wenig Sinn ergibt, da diese keine Berechtigung
haben um selber Termine zu erstellen. Sie können lediglich eingestellte
Zeitslots an Ratsuchende vergeben. Wir diskutieren über die Möglichkeit diesen
Filter für Accounts von Hilfskräften auszublenden, entscheiden uns aber dafür,
dass dies nicht notwendig ist, da der Filter für diese Nutzergruppe zwar
überflüssig, aber auch nicht störend ist.

\subsubsection{Detailansicht}

Als nächstes gebe ich \ipName die Aufgabe einen neuen Zeitslot zu erstellen.
Ich verrate nicht weiter, wo er diese Funktion finden kann und lasse ihn
ausprobieren. Nach kurzer Zeit entdeckt \ipName den Button \textit{Neuer
    Termin} und es öffnet sich das Modal zum Erstellen eines freien Zeitslots.
\ipName erklärt seine Gedanken: \glqq{}Ich kenne es von anderer Software so,
dass man auf den entsprechenden Tag in der Monatsübersicht klickt um einen
neuen Termin anzulegen. Aber aus der alten Version von Stubegru erinnere ich
mich noch an den grünen Knopf zum Erstellen eines Termins. Früher war der Knopf
unter der Monatsübersicht, aber so fügt er sich elegant in die Leiste mit der
Suche und den neuen Filtern ein.\grqq{}

\begin{figure}[H]
    \caption{Navigationsleiste über der Monatsansicht. Über den grünen Button kann eine neuer Zeitslot erstellt werden.}
    \centering
    \includegraphics[width=0.9\textwidth]{screen_new_create_button.png}
\end{figure}

In dem leeren Formular füllt \ipName Schritt für Schritt die Input Felder aus.
Nach einem Klick auf das Feld für die Startzeit ist er enttäuscht. In der alten
Version öffnete sich hier ein Timepicker mit dessen Hilfe man die Uhrzeit ohne
Tastatur nur mit der Maus eingeben konnte. Auch ich bin überrascht, denn in
meinen Tests hat sich hier auch ein Timepicker geöffnet. Nach kurzer REcherche
stellt sich heraus, dass das verwendete Html5 Input Feld für Uhrzeiten in jedem
Browser unterschiedlich dargestellt wird. Während der Implementierung habe ich
die Ergebnisse immer in Google Chrome getestet, dort erscheint nach dem Klick
in ein Uhrzeit-Input ein Timepicker \todo{Timepicker ins Glossar}. Die
Abteilung der Studienberatung verwendet in der Regel Firefox als Browser, da
dieser vom ITS standardmäßig vorinstalliert ist. Ich verspreche mich dem
Problem anzunehmen und einen zusätzlichen Timepicker einzubinden, der
unabhängig vom verwendeten Browser funktioniert.

\begin{figure}[H]
    \caption{Erstellen eines neuen Termins und Ausfüllen der Eingabefelder. In diese Fall wird in Google Chrome ein Timepicker angezeigt.}
    \centering
    \includegraphics[width=0.9\textwidth]{screen_new_create_with_picker.png}
\end{figure}

Alle anderen Formularfelder füllt \ipName zügig aus. Da die abgefragten Daten
und die Anordnung der Eingabefelder genauso wie in der alten Version gestaltet
sind, merkt man hier, dass \ipName bereits viel Erfahrung mit der alten
Softwareversion von Stubegru gesammelt hat. Den unten rechts platzierten grünen
Button zum Speichern der Daten findet er sofort. Der Datensatz wird in der
Datenbank gespeichert und eine kleine grüne Meldung am linken unteren Rand
verkündet, dass der Termin erfolgreich gespeichert wurde. \ipName ist trotzdem
verunsichert, ob der Vorgang erfolgreich war. In der alten Version, hat sich
das Modal nach dem Speichern automatisch geschlossen. \ipName merkt an, dass in
der neuen Version nun jedes Mal ein zusätzlicher Klick auf den Button
\textit{Schließen} notwendig ist. Wenn das Modal sich nach dem Speichern direkt
schließen würde, wäre es für ihn einfacher. Ich erkläre, dass das Modal
geöffnet bleibt, damit ein neu erstellter Termin direkt an einen Kunden
vergeben werden kann. Nach dem Speichern erscheint hierfür der große blaue
Button mit der Aufschrift \textit{Termin vergeben}. \ipName erklärt, dass die
Termine eigentlich immer von den Hilfskräften der Erstinformation vergeben
werden und es für die Studienberatenden nicht nötig sei, den erstellten Termin
direkt vergeben zu können. Wir einigen uns darauf, dass ein Klick auf den
Speichern Button das Modal, nach erfolgreichem Sichern der Daten, schließt.

Und noch eine weitere Sache fällt \ipName auf: Wenn er bereits einen Termin
eingetragen hat und dann einen zweiten Termin erstellen möchte, werden alle
Eingabefelder zurückgesetzt und sind wieder leer. \glqq{}Das war in der alten
Version praktischer\grqq{}, sagt \ipName \glqq{}Oftmals trage ich viele Termine
fpr den gleichen Tag mit den gleichen Attributen ein. Das einzige was ich
bisher anpassen musste war dann die Uhrzeit. Nun brauche ich viel mehr Klicks
weil ich jedes Mal wieder den Raum und das Mailtemplate auswählen muss.\grqq{}.
Es stellt sich heraus, das es in der alten Softwareversion einen Bug gab,
wodurch in eingen Fällen das zurücksetzen des Formulars nicht aufgerufen wurde.
Dieser Bug hat sich in der Praxis dann allerdings als sehr praktisch etabliert.
Die Formularfelder nie zurückzusetzen kann allerdings inkonsistente Status
erzeugen, wenn zuvor beispielsweise ein bereits vergebener Termin im Modal
angezeigt wurde. Wir einigen uns auf eine gemeinsame Idee: Wenn das Modal zum
Erstellen eines Termins angezeigt wird, erscheint in der Fußleiste ein weiterer
Button mit der Aufschrift: \textit{Speichern und nächster}. Dieser Button
speichert den aktuellen Termin und lässt das Modal ohne zurücksetzen der
Formularfelder geöffnet. Somit kann direkt ein neuer Termin mit ähnlichen
Attributen eingetragen werden. \todo{Direkte Zitate von \ipName kennzeichnen?}

\begin{figure}[H]
    \caption{Ein freier Termin. Studienberatende können Änderungen am Termin vornehmen oder einen Kunden einbuchen.}
    \centering
    \includegraphics[width=0.9\textwidth]{screen_new_free_meeting.png}
\end{figure}

Als nächstes soll \ipName einen erstellten Zeitslot an einen Kunden vergeben.
Intuitiv klickt er hierfür auf den entsprechenden Termin in der
Monatsübersicht. Es öffnet sich das Modal mit der Detailansicht der
Termindaten. Sehr schnell findet \ipName den großen blauen Knopf zur
Terminvergabe. Wie er es bereits aus der alten Softwareversion kennt, öffnet
sich das Formular um die weiteren Kundendaten einzutragen. \ipName prüft was
passiert, wenn er hier relevante Felder einfach leer lässt. Beruhigt stellt er
fest, das beim Versuch die Kundendaten zu speichern ein Hinweis am
entsprechenden leeren Eingabefeld auftaucht und das Speichern unvollständiger
Daten verhindert.

Nachdem alle Felder vollständig ausgefüllt sind, speichert \ipName die
Kundendaten und sieht eine leicht veränderte Detailansicht des Termins:

\begin{figure}[H]
    \caption{Ein vergebener Termin. Terminattribute und neue Kundendaten können erst nach Löschen der aktuellen Kundendaten hinterlegt werden.}
    \centering
    \includegraphics[width=0.9\textwidth]{screen_new_assigned.png}
\end{figure}

Die Terminattribute und Kundendaten können nicht weiter bearbeitet werden. Dies
erkennt \ipName schnell daran, dass die entsprechenden Eingabefelder ausgegraut
angezeigt werden. Zusätzlich verrät der Hinweis in der blauen Infobox, dass
Änderungen nur vorgenommen werden können, wenn zuvor die Kundendaten gelöscht
werden. \ipName realisiert, dass diese Einschränkung Sinn macht, da somit keine
Änderungen am Termin vorgenommen werden können, ohne dass der Kunde durch ein
erneutes Einbuchen darüber informiert wird.

\ipName freut sich, dass die Telefonnummer nun in übersichtlichen Blöcken von jeweils vier Ziffern angezeigt wird. Das Eintippen ins Telefon sei somit deutlich einfacher. Allerdings merkt er an, dass die Telefonnummer nur in dem Account des zugeordneten Beratenden angezeigt werden soll. So kann ein höherer Datenschutz gewährleistet werden und Hilfskräfte haben keinen Zugriff auf die Kontaktdaten aller ratsuchenden Kunden. Ich erkläre, dass ich diese Funktion bereits für das Beratungsanliegen umgesetzt habe. Wir loggen uns mit einem anderen Account ein und sehen, dass das Beratungsanliegen zensiert dargestellt wird. Dies Funktionalität soll in Zukunft auch für die Telefonnummer und Mailadresse der Kunden eingesetzt werden.

Abschließend betrachtet \ipName nochmals die Monatsübersicht. \glqq{} Wäre es
technisch aufwendig, bei den vergebenen, roten Terminen statt dem Namen des
Beratenden den Namen des Kunden anzuzeigen? So könnten die Hilfskräfte den
entsprechenden Termin eines Kunden schneller finden.\grqq{}, fragt er mich. Ich
antworte, dass dies technisch kein großer Aufwand wäre. Allerdings könnte es
Verwirrung stiften, wenn bei manchen Terminen der Name des Beratenden und bei
anderen der Name des Kunden angezeigt wird. Für den Fall, dass der Termin eines
bestimmten Kunden schnell gefunden werden muss, wurde außerdem die Suchfunktion
entwickelt. \ipName erinnert sich an die Suchfunktion und will diese direkt
ausprobieren.

\begin{figure}[H]
    \caption{Suchfunktion um bereits vergebene Termine anhand des eingebuchten Kunden zu finden. Die Suchanfrage \glqq{}jo\grqq{} ergibt drei Treffer.}
    \centering
    \includegraphics[width=0.9\textwidth]{screen_new_search.png}
\end{figure}

Das Eingabefeld für die Suchanfrage findet er schnell in der oberen
Navigationsleiste. Er gibt einige Buchstaben ein und direkt werden passende
Suchvorschläge angezeigt. \ipName ist überzeugt, dass dies eine gute Lösung
ist, um direkt nach Namen von ratsuchenden Kunden zu suchen. Er betont, dass es
wichtig ist, dass passenden Suchvorschläge bereits beim Tippen der ersten
Buchstaben angezeigt werden und kein Bestätigen der Suche mit Enter notwendig
ist.

\subsubsection{Abschluss Usertest}
Der Usertest ist nun beendet und \ipName bedankt sich für die
entgegengebrachte Aufmerksamkeit. Er ist sichtlich begeistert, dass die
Nutzenden des Systems in diesem Designprozess im Fokus stehen und freut sich
auf die Einführung der neuen Softwareversion: \glqq{} Wir müssen dann
sicherlich eine Schulung für die Hilfskräfte machen um sie mit den neuen
Features vertraut zu machen. Aber eigentlich ist ja fast alles so, wie sie es
schon gewohnt sind.\grqq{}, beendet \ipName unser Gespräch. Ich bedanke mich
ebenfalls für die Zeit und die konstruktive Kritik, die eine Diskussion über
Verbesserungspotenziale erst möglich gemacht hat.
\chapter{Reflexion und Fazit}
\label{chapter:conclusion}

Nachdem in den vorherigen Kapiteln die Durchführung des praktischen
Designprozesses ausführlich beschrieben wurde, soll dieses Kapitel die
Ausarbeitung abschließen, indem es die Ergebnisse zusammenfasst und
hinterfragt. Zunächst wird im Abschnitt~\ref{subsection:resultDescription} das
Ergebnis des Gestaltungs- und Entwicklungsprozesses resümiert. Hier wird
nochmals darauf eingegangen, wie der Nutzungskontext erfasst wurde, die
Nutzungsanforderungen spezifiziert, die entsprechenden Gestaltungslösungen
umgesetzt und evaluiert wurden. Somit wird der in den Kapiteln
\ref{chapter:user-context} bis~\ref{chapter:evaluation} ausführlich
dokumentierte Entwicklungsprozess nochmals anhand der vier Phasen des Human
Centered Design aufgegriffen und eingeordnet~\cite{ISO9241}. Zusätzlich wird
kurz präsentiert, welche Änderungen an der Software nach der Auswertung des
Usertests noch vorgenommen wurden. Als nächstes werden die eingesetzten
Methoden im Abschnitt~\ref{subsection:reflection} aufgegriffen und kritisch
hinterfragt. Der Fokus liegt dabei auf der Methode des Interviews im Kontext
und auf dem Designzyklus des Human Centered Design nach ISO9241. Hierbei soll an
die Fragestellung der Einleitung angeknüpft werden und eine Verbindung zu den
gewonnenen Resultaten hergestellt werden. Als Letztes wird in Abschnitt
\ref{subsection:outlook} ein Ausblick gegeben, wie die entwickelte Software nun
in der Praxis eingesetzt werden kann und welche Schritte dazu noch notwendig
sind. Außerdem wird auch skizziert, welche weiteren wissenschaftlichen Methoden
und Prozesse man noch anwenden könnte, um die hier gewonnen Ergebnisse zu
vertiefen und weiter zu verwenden.

\section{Zusammenfassung der Ergebnisse}
\label{subsection:resultDescription}

\subsection*{Was war geplant, was wurde umgesetzt?}
Ziel dieses Abschnittes soll es sein, den Bogen zu spannen zwischen den
anfänglich erarbeiteten Nutzungsanforderungen und den am Ende tatsächlich
umgesetzten Ergebnissen.

Wenn man sich die Auswertung des Interviews im Kontext in Abschnitt
\ref{subsection:IIK} ansieht, wird klar, dass eines der wichtigsten Bedürfnisse
der Nutzenden stets das schnelle und intuitive Nutzen der Software ist. Es
sollten möglichst viele Informationen auf einen Blick sichtbar sein, ohne die
Nutzenden von den wesentlichen Dingen abzulenken. Wenn Workflows durch die
Nutzenden abzuarbeiten sind, sollten diese möglichst schnell und mit möglichst
wenigen Klicks zu erledigen sein. Ein großer Detailgrad der Auswahl- und
Konfigurationsmöglichkeiten sind dagegen in diesem Kontext nicht relevant. Wo
die Software eine halbwegs sinnvolle Vorauswahl treffen kann, sollte dies
automatisch passieren und den Nutzenden somit unnötige Interaktion ersparen.

Um hierfür konkrete Anhaltspunkte zu geben, werden im Folgenden die drei
exemplarisch vorgestellten Nutzungsanforderungen aus Abschnitt
\ref{subsection:SpannendeErkenntnisse} nochmals aufgegriffen.

Gefordert wurde da eine kompakte Kalenderansicht, die durch farbliche
Markierungen auf den ersten Blick darstellt, an welchen Tagen noch freie
Termine verfügbar sind. Diese Anforderung konnte durch die tabellarische
Monatsansicht mithilfe der Javascript Bibliothek \textit{fullcalendar}
umgesetzt werden. Durch die roten und grünen Punkte neben jedem angezeigten
Termin kann direkt auf den ersten Blick erkannt werden, ob es sich um einen
bereits vergebenen oder freien Termin handelt. Durch die verwendete Bibliothek
zum Darstellen der Monatsübersicht ist dieser tabellarische Überblick nicht
mehr ganz so kompakt wie in der alten Softwareversion. Dies ist auch \ipName
beim Usertest negativ aufgefallen. Ein positiver Ausgleich hierfür ist
allerdings, das in der neuen Übersicht direkt die einzelnen Termine eines Tages
dargestellt werden. In der alten Version war es nötig, zunächst den Tag
anzuklicken und erst in dem danach erscheinenden Modal konnten die einzelnen
Termine und die entsprechenden Uhrzeiten eigensehen werden. Der etwas
ausführlichere neue View nimmt also mehr Platz auf dem Bildschirm ein und kann
bei sehr vielen Terminen an einzelnen Tagen sehr lang werden, bietet dafür aber
eine umfassende Übersicht über freie und vergebene Termine und stellt
weiterführende Informationen mit nur einem Klick in der Detailansicht zur
Verfügung.

\begin{figure}[H]
    \caption{Vergleich der Monatsübersicht in der alten und neuen Softwareversion.}
    \centering
    \includegraphics[width=\textwidth]{screen_past_now_month_view.pdf}
\end{figure}

Als nächstes wurde beim Interview im Kontext deutlich, dass eine Funktion
notwendig ist, um im Nachhinein den Termin eines Ratsuchenden schnell und
unkompliziert finden zu können. Hierfür wurde die Suchfunktion neu
implementiert. Bereits nach dem Eingeben einiger Buchstaben aus dem Namen der
ratsuchenden Person schlägt diese neue Suchfunktion passende Ergebnisse vor und
bietet in einer tabellarischen Übersicht direkt die wichtigsten Daten zum
zugehörigen Beratungstermin. Durch einen Klick auf ein Suchergebnis kann der
Termin schnell und einfach in der Detailansicht geöffnet werden. In der
Planungsphase war hierfür ein extra Button mit einem Augensymbol vorgesehen.
Ein einfacher Klick auf die entsprechende Zeile in der Tabelle hat sich aber
als praktischer erwiesen. Im Vergleich zur alten Softwareversion sind die
Suchergebnisse sehr viel ausführlicher und strukturierter dargestellt. Dies
wurde auch im Usertest positiv wahrgenommen. In weiteren Tests entsteht der
Eindruck, dass die alte Suchfunktion etwas schneller gearbeitet hat. Dies kann
aber erst objektiv bewertet werden, wenn die neue Implementierung auch auf den
gleichen Produktiv-Servern installiert und im Einsatz ist.

\begin{figure}[H]
    \caption{Vergleich der Suchfunktion in der alten und neuen Softwareversion.}
    \centering
    \includegraphics[width=\textwidth]{screen_past_now_search.pdf}
\end{figure}

Als letztes wurde im Abschnitt~\ref{subsection:SpannendeErkenntnisse} das
elegante Darstellen der Telefonnummern gefordert. Dieses Feature ließ sich in
der Praxis leicht implementieren. In die aus der Datenbank abgerufene 
Telefonnummer wird an jeder vierten Stelle ein Leerzeichen eingefügt. Somit
kann die Telefonnummer von den Studienberatenden bei Bedarf in leicht zu
merkenden Vierer-Blöcken in das Telefon eingeben werden. Für andere
Nutzeraccounts als den zuständigen Beratenden werden persönliche Details der
Ratsuchenden nicht angezeigt. Diese Funktion gab es bereits in der alten Version, allerdings wird in der neuen Version statt den eigentlichen Daten ein
Hinweis angezeigt. In der alten Version wurde an dieser Stelle gar nichts
angezeigt, was manchmal zu Unsicherheiten geführt hat, ob hier überhaupt Daten
eingetragen und korrekt gespeichert wurden. Das Feedback aus dem Usertest
zeigt: Die neue Variante braucht etwas mehr Platz auf dem Bildschirm, ist dafür
aber klarer zu verstehen und intuitiver zu erfassen.

\begin{figure}[H]
    \caption{Vergleich der Detailansicht eines vergebenen Termins in der alten und neuen Softwareversion.}
    \centering
    \includegraphics[width=\textwidth]{screen_past_now_client_details.pdf}
\end{figure}

\subsection*{Umsetzung von Feedback des Usertests}
\label{paragraph:weitereIteration}

In den Kapiteln~\ref{chapter:user-context} bis~\ref{chapter:evaluation} wurde
das Durchlaufen des Designzyklus anhand einer Iteration ausführlich
beschrieben. Ein essentielles Grundkonzept des Human Centered Design beinhaltet
allerdings auch das Durchlaufen weiterer Iterationen~\cite{hcd}. K. Holtzblatt
betont in \textit{Contextual Design: Evolved}, dass der Entwicklungsprozess mit
einem durchgeführten Usertest noch nicht als abgeschlossen gelten sollte. Die
herausgearbeiteten Kritikpunkte und Verbesserungsvorschläge, die sich aus dem
Usertest ergeben haben, sollten auch in den Gestaltungsprozess der Software mit
einfließen~\cite{holtzblattCDEvolved}. Dies ist im Schaubild nach ISO9241 durch
die gestrichelten Pfeile erkennbar. Nach der Evaluierung sollte der Prozess
erneut durchlaufen werden. Der Einstieg erfolgt dabei je nach Feedback der
Nutzenden in Phase eins bis drei~\cite{ISO9241}. Diese weiteren Iterationen
ausführlich zu beschreiben würde den Rahmen dieser Arbeit überstrapazieren.
Daher werden im Folgenden lediglich die Ergebnisse eines zweiten Durchlaufes
nach dem Usertest exemplarisch präsentiert:

Ein neuer Button \textit{Speichern und Nächster} wird beim Erstellen eines
neuen Termins angezeigt. Somit kann mit einem Klick der aktuelle Termin
gespeichert werden und direkt ein neuer Datensatz eingetragen werden. Die
Besonderheit ist, dass die Formularfelder in diesem Fall nicht auf
Standardwerte zurückgesetzt werden. Somit können viele Termine mit ähnlichen
Attributen besonders schnell hintereinander eingetragen werden.

\begin{figure}[H]
    \caption{Neuer Button: Mit \textit{Speichern und Nächster} kann direkt der nächste Termin eingetragen werden.}
    \centering
    \includegraphics[width=0.9\textwidth]{screen_feedback_save_next.png}
\end{figure}

\ipName hat während des Usertests angemerkt, dass nicht nur das Beratungsanliegen der ratsuchenden Personen besonderem Datenschutz unterliegt. Auch persönliche Kontaktdaten wie Mailadresse und Telefonnummer sollten nur dem zuständigen Beratenden angezeigt werden. In der überarbeiteten Version werden somit nun auch die Telefonnummer und die Mailadresse des Kunden zensiert.

\begin{figure}[H]
    \caption{Details eines vergebenen Termins aus Ansicht einer Hilfskraft.}
    \centering
    \includegraphics[width=0.9\textwidth]{screen_feedback_client_censorship.png}
\end{figure}

\section{Reflexion der eingesetzen Methoden}
\label{subsection:reflection}

Um dem wissenschaftlichen Anspruch dieser Arbeit Genüge zu tragen, sollen die
vorgestellten Methoden und Theorien nicht nur in der Praxis erprobt, sondern
auch kritisch hinterfragt und eingeordnet werden. Im folgenden Abschnitt werden
die Methoden des Human Centered Design und des Interviews im Kontext nochmals
aufgegriffen und deren Praxistauglichkeit reflektiert. Des Weiteren wird auch
die Umsetzung und Implementierung der Gestaltungslösungen in Frage gestellt und
unter Einbeziehung des Feedbacks aus dem Usertest analysiert.

\subsection*{Human Centered Design und Interview im Kontext}
Die Theorie des Human Centered Design wurde zu Beginn dieser Arbeit motiviert
mit der technischen Entwicklung von interaktiven Systemen. Software ist im
Laufe der letzten Jahrzehnte immer interaktiver geworden. Der Schnittstelle
zwischen Mensch und Maschine kommt dabei eine immer größer werdenden Bedeutung
zu~\cite{hci}. Der Ansatz des Human Centered Design versucht an dieser Stelle
alle Beteiligten einzubeziehen und eine einfache, intuitive Bedienung der
Systeme zu ermöglichen, die ganz bewusst aus Perspektive der Nutzenden
gestaltet wurde~\cite{sequenceDiagrams}. Diese Theorie wurde in diesem Fall
anhand des Designzyklus nach ISO9241 in die Praxis umgesetzt. Für die
Durchführung der ersten Phase \textit{Den Nutzungskontext verstehen und
    Beschreiben} wurde hier die Methode des Interviews im Kontext verwendet.
Hierbei ging es darum, alle Menschen, die mit dem System zu tun haben
einzubeziehen und ihre Arbeitsabläufe zu verstehen. Wie der Name der Methode
bereits impliziert, liegt der Fokus hierbei auch auf dem Kontext, in dem diese
Arbeitsabläufe stattfinden~\cite{hciHandbook}. Das Interview im Kontext wurde
deshalb mit \ipName in seinem Büro an seinem Dienstrechner durchgeführt, um eine
möglichst realistische Alltagssituation abzubilden.

Eine zentrale Fragestellung dieser Arbeit war folgendermaßen beschrieben
worden: \textit{An welchen Stellen können die theoretischen Grundlagen des
    Human Centered Design den Entwicklungsprozess in der Praxis tatsächlich
    sinnvoll unterstützen?} Der Einsatz dieser Methoden lässt sich rückblickend als
sehr ertragreich beschreiben. Durch den engen Austausch mit \ipName konnte der
Nutzungskontext sehr gut verstanden und aufgearbeitet werden. Das Interview im
Kontext hat überraschend viele neue Ideen hervorgebracht, die wohl keiner der
Beteiligten im Vorhinein hätte formulieren können. Durch das gemeinsame
Benutzen der alten Softwareversion vor Ort wurde ganz klar deutlich, welche
Features wirklich wichtig sind für den täglichen Einsatz, an welchen Stellen es
schnell gehen muss und wo detaillierte Einstellungsmöglichkeiten unbedingt
notwendig sind. Die Erkenntnisse und Beobachtungen aus dem Interview im Kontext
wurden unstrukturiert schriftlich festgehalten. Beim Auswerten dieser Notizen
ist aufgefallen, dass viele Details aus dem Gespräch nicht so schnell
aufgeschrieben werden konnten. Die Notizen waren somit nicht ganz vollständig.
Daher war es sehr wichtig, die Auswertung des Interviews im Kontext zeitlich
sehr nah an der Durchführung zu planen. Somit konnten viele Details aus der
Erinnerung an das Interview selbst bei der Auswertung ergänzt werden. Der
Austausch mit Menschen aus der allgemeinen Studienberatung hat sich auf den Kontakt
mit \ipName beschränkt. Dieser Austausch war sehr ertragreich, dennoch wäre es
interessant gewesen, auch noch Nutzungsanforderungen anderer Nutzergruppen wie
beispielsweise der Hilfskräfte der Erstinformation mit einzubeziehen. Die
Erfahrungen und Ideen anderer Nutzenden hätten noch weitere hilfreiche Aspekte
in den Gestaltungsprozess einbringen können. Somit hätte ein System entstehen
können, das noch inklusiver und reibungsloser alle beteiligten Personen mit
einbezieht. In der Praxis war es allerdings schon sehr aufwendig, Termine mit
\ipName zu organisieren. Wären noch weitere Nutzungsgruppen mit einbezogen
worden, wäre der organisatorische und zeitliche Aufwand der Software deutlich
höher gewesen. Dies kann gerade in gewinnorientierten Softwarehäusern einen
hemmenden Faktor für den Einsatz von Human Centered Design mit sich bringen.

Um noch einmal Bezug auf die vier Phasen des Gestaltungsprozesses nach ISO9241
zu nehmen, kann man festhalten, dass die klare Zuordnung der Arbeitsschritte in
eine dieser Phasen sehr gut funktioniert hat. Somit wurde jede Arbeitsphase
klar strukturiert und war einem definierten Ergebnis zugeordnet. Ein
zielführendes und effizientes Arbeiten wurde somit erleichtert.

\subsection*{Implementierung und Usertests}

Im Folgenden soll nun auch die Implementierung der Software kritisch
hinterfragt werden. Die Ergebnisse der Usertests sollen hierbei mit einbezogen
werden und das Meinungsbild mit ersten Testerfahrungen bestärken.

% Einfache Umsetzung vs Nutzerfreundliche Lösung
Zunächst fällt allgemein auf, dass besonders nutzungsfreundliche Lösungen
oftmals technisch verhältnismäßig aufwendig umzusetzen sind. Wird Software von
Seite der technischen Möglichkeiten und Datenstrukturen her gestaltet, drängen
sich bestimme Herangehensweisen oftmals fast unumgänglich auf. Diese
Herangehensweisen sind dann meist sehr technisch geprägt und für unerfahrene
Nutzende manchmal schwierig zu verstehen. Wenn man diesen Prozess im Sinne des
Human Centered Design bewusst umdreht und die Gestaltung bei den Nutzenden und
ihrem Umfeld beginnt, können oftmals deutlich passendere und einfachere
Workflows und Views geschaffen werden. Diese sind in der Implementierung
allerdings oftmals aufwendig zu realisieren.

% Problem: Designpattern bei fertigem Softwareökosystem sehr eingeschränkt
% Objektorientierter Ansatz sinnvoll für start serverlastige Anwendung?
Als nächstes soll nochmal in Erinnerung gerufen werden, dass es sich bei dieser
Umsetzung um ein Modul in einer bereits bestehenden Software handelt. Es
mussten also alle bestehenden Schnittstellen und Programmierkonzepte der
bereits existierenden Module eingehalten werden. Beispielsweise wurde für die
Termindatensätze, sowie für Beratungsräume und Mailtemplates, ein
objektorientierter Ansatz gewählt. Ziel war es, damit die Beziehungen dieser
Datentypen untereinander klarer modellieren und elegant in Code umsetzen zu
können. Im Rahmen der gesamten Software Stubegru hat sich allerdings gezeigt,
das es eher wenig Sinn macht, lokale Instanzen von Objekten zu erstellen und zu
referenzieren. Die restlichen Module funktionieren so, dass Daten meist nur vom
Server abgerufen und angezeigt werden. Weitere Referenzen oder Methodenaufrufe
auf diesen Daten sind nicht vorgesehen und machen in diesem Anwendungsfall auch
wenig Sinn. Die erhofften Vorteile der objektorientierten Umsetzung in diesem
Modul konnten somit nicht vollständig erreicht werden. Viele Methoden wurden
letztendlich als \textit{static} implementiert und werden somit unabhängig von
der Instanz einer Klasse aufgerufen.

% Prototypen wichtig für Feedback, Vorstellung
% Evtl frühere Tests
Im Usertest hat sich herausgestellt, dass es extrem hilfreich ist, wenn man
eine Gestaltungslösung vorzeigen und gemeinsam durchspielen kann. Beim
gemeinsamen Nutzen der implementierten Funktionen fallen oft wichtige Details
auf, die im Vorhinein bei der Planung noch nicht berücksichtigt wurden. Somit
war es also extrem hilfreich, für den Usertest bereits das fast vollständig
fertig implementierte Modul zur Terminvereinbarung präsentieren zu können und
somit eine greifbare Gesprächsgrundlage zu bieten. Vielleicht wäre es hilfreich
gewesen, schon früher im Gestaltungsprozess erste Prototypen zu entwickeln, die
noch nicht voll funktionstüchtig sind, allerdings schon die wichtigsten
Nutzungsinteraktionen wiedergeben können. Dann hätte man den Nutzenden schon
früher im Designprozess ein anschauliches Beispiel für konkrete
Gestaltungslösungen präsentieren können. Somit hätten etwaige Mängel schon
früher erkannt und direkt korrigiert werden können. Allerdings lässt sich
auch festhalten, dass durch das ausführliche Interview im Kontext bereits ein
sehr guter Überblick entstanden ist, wie die Software aussehen sollte, um die
Nutzenden möglichst gut zu unterstützen. Somit waren nach dem Usertest gar
keine großen, grundlegenden Änderungen mehr notwendig. Dies zeigt auf, dass
eine ausführliche und gewissenhafte Durchführung der ersten Phasen im
Gestaltungsprozess die Wahrscheinlichkeit für Missverständnisse oder größere
Änderungswünsche in späteren Usertests senkt.

\section{Ausblick}
\label{subsection:outlook}

Als abschließender Abschnitt dieser Ausarbeitung soll die weitere Perspektive
des Forschungsthemas betrachtet werden. Hierbei ist es interessant, den weiteren
Einsatz des praktisch entwickelten Moduls zur Terminvereinbarung zu skizzieren.
Wie wird sich das Modul in das gesamte Softwarepaket Stubegru eingliedern? In
welchem Kontext kann es tatsächlich produktiven Einsatz finden und welche
Schritte sind dafür noch zu bedenken? Außerdem sollen die verwendeten Methoden
und Theorien nochmals aufgegriffen werden. An welchen Stellen lassen sich die
verwendeten Konzepte eventuell noch verbessern? In welcher Richtung verspricht
eine weitere wissenschaftliche Forschung spannende neue Erkenntnisse?

\subsection*{Werdegang der Software Stubegru}
Wie bereits in Abschnitt~\ref{paragraph:weitereIteration} erwähnt, sind noch
weitere Iterationen und Feedbackrunden mit den Nutzenden nötig, um das neu
entwickelte Modul zur Terminvereinbarung tatsächlich sinnvoll im Arbeitsalltag
zu verwenden. Es ist konkret geplant, diese weiteren Entwicklungszyklen zu
durchlaufen und eine neue Version der Software Stubegru in der Abteilung
Studium und Lehre produktiv einzusetzen. Ist das Modul zur Terminvereinbarung
fertig, werden noch andere Module und Workflows der Software überarbeitet
werden müssen. Wenn all diese Arbeiten abgeschlossen sind, wird es eine
intensive Einführungsphase geben. In den ersten Wochen im produktiven Betrieb
ist zu erwarten, dass noch einige Bugs entdeckt und behoben werden müssen.
Außerdem müssen die Mitarbeitenden der Abteilung auf die neuen Abläufe in der
überarbeiteten Softwareversion geschult werden. All diese Schritte sollen im
Rahmen eines Dienstleistungsvertrages mit der Universität Kassel weiterhin
betreut werden.

\subsection*{Forschungsausblick}
Die Durchführung und Auswertung dieser Arbeit hat gezeigt, dass die verwendeten
Theorien des Human Centered Design schon sehr ausgereift und vor allem sehr
praxisnah formuliert sind. Die Umsetzung der methodischen Grundkonzepte ließ
sich somit sehr gut in die Praxis übertragen. Spannend wäre es allerdings, sich
noch tiefgreifender mit den verschiedenen Nutzungsgruppen einer Software
auseinander zu setzen. In dieser Arbeit bedeutete der Austausch mit den
Nutzenden größtenteils der Kontakt mit einem Studienberater der allgemeinen
Studienberatung. Dieser repräsentierte eine Nutzungsgruppe, die bereits
jahrelange Erfahrung mit den Prozessen und Abläufen in der Abteilung vorweisen
kann. Spannend wäre es, noch weitere Nutzungsgruppen in den Fokus zu rücken.
Hilfskräfte beispielsweise sind oftmals nur für einen kurzen Zeitraum mit
geringen Wochenstunden angestellt. Eine jahrelange Einarbeitung in die Prozesse
des Arbeitsalltags kann hier nicht vorausgesetzt werden. Wie könnte man also
diese Nutzungsgruppe besser in der Bedienung der verwendeten Software
unterstützen? Könnten zum Beispiel kleine interaktive Hilfetexte helfen, die
Arbeitsprozesse spielerisch bei der Verwendung der Software zu erlernen?
Hierfür wäre sicherlich ein intensiverer und direkter Austausch mit
Hilfskräften der Abteilung notwendig gewesen.

Des Weiteren wäre es relevant, die Unterschiede in der direkten Interaktion mit
dem Computer der einzelnen Nutzenden genauer zu untersuchen. In den hier
verwendeten Grundlagen zum Human Centered Design lag der Fokus immer darauf,
eine Gestaltungslösung zu finden, die möglichst nutzungsfreundliche
Schnittstellen bietet~\cite{hcd}. Oftmals bevorzugen verschiedene Nutzende
allerdings verschiedene Schnittstellen. Ich persönlich, als technisch sehr
erfahrener Nutzer, bevorzuge beispielsweise Eingaben über die Tastatur und
arbeite gerne mit Shortcuts. In den Usertests hat sich schnell gezeigt, dass
der Testkandidat am liebsten alle Formularfelder mit der Maus ausfüllt. Somit
ist für ihn beispielsweise ein grafischer Timepicker sehr wichtig für eine gute
User Experience. Für mich wäre es allerdings viel wichtiger, die Möglichkeit zu
haben, jedes Eingabefeld auch über Tastaturbefehle zu erreichen. Dieses kleine
Beispiel soll illustrieren, dass selbst einzelne Menschen einer Nutzungsgruppe
oftmals ganz verschiedene Vorstellungen von einer \textit{angenehmen} und
\textit{schönen} Nutzungsoberfläche haben. Diese Differenzen weiter zu
untersuchen und allen Nutzenden eine adäquate Schnittstelle zu bieten, wäre also
eine weitere spannende Richtung, um die Ergebnisse dieser Arbeit noch weiter zu
vertiefen.

\chapter{Glossar}
\setglossarystyle{altlist}\printglossaries

\chapter{Literaturverzeichnis}
\printbibliography[heading=none]

\begingroup
\chapter{Abbildungsverzeichnis}
\let\clearpage\relax
\renewcommand{\listfigurename}{}
\listoffigures
\endgroup

\chapter{Anhang}

\section{Software Stubegru}
\label{section:stubegru-refs}

\subsection*{GitHub Repository}
Der komplette Quellcode der Software Stubegru ist als öffentliches OpenSource
Repository auf GitHub verfügbar:\\
\url{https://github.com/stubegru/stubegru}\\ \\
Dieses Repository umfasst den Coder des gesamten Softwarepakets. Alle Skripte, die im Rahmen dieser Bachelorarbeit entstanden sind, finden sich unter:\\
\url{https://github.com/stubegru/stubegru/tree/main/modules/calendar2}\\ \\
Alle Issues für das neue Modul zur Terminvereinbarung sind unter folgendem Milestone zusammengefasst:\\
\url{https://github.com/stubegru/stubegru/milestone/1?closed=1}

\subsection*{Website und Live-Demo}
Zusätzlich zum Repository mit allen technischen Informationen, finden sich auf der offiziellen Website weitere Infos zum Softwarepaket Stubegru:\\
\url{https://stubegru.org}\\ \\
Über die Website lässt sich ebenfalls eine Live-Demo der Software Stubegru aufrufen:\\
\url{https://stubegru.org/demo}.\\ \\
Um das neu entwickelte Modul zur zweistufigen Terminvereinbarung zu testen kann direkt der entsprechende View aufgerufen werden:\\
\url{https://stubegru.org/demo?view=caltest}

\section{Sequenzdiagramme}
\label{section:anhang:Sequenzdiagramme}

Hier finden sich die in Kapitel \ref{subsection:sequenceDiagrams} erwähnten Sequenzdiagramme für alle vier betrachteten Workflows.

\begin{figure}[H]
    \caption{Sequenzdiagramm, Erstellen eines neuen Termins.}
    \centering
    \includegraphics[width=\textwidth]{flow_neuer_termin.jpeg}
\end{figure}

\begin{figure}[H]
    \caption{Sequenzdiagramm, Laden der Detailansicht für einen freien Termin.}
    \centering
    \includegraphics[width=\textwidth]{flow_termin_aufrufen_unvergeben.jpeg}
\end{figure}

\begin{figure}[H]
    \caption{Sequenzdiagramm, Laden der Detailansicht für einen vergebenen Termin.}
    \centering
    \includegraphics[width=\textwidth]{flow_termin_vergeben.jpeg}
\end{figure}

\begin{figure}[H]
    \caption{Sequenzdiagramm, Vergeben eines Termins und Eintragen der Kundendaten.}
    \centering
    \includegraphics[width=\textwidth]{flow_termin_aufrufen_vergeben.jpeg}
\end{figure}



%Ende
\end{document}